\section{Diffusion on a shrinking domain}

In

Calculating MSD:

Assuming that after the initial dispersal, bats are uniformly distributed around a circular domain of radius $R_0$, the probability distribution is uniform,

\begin{equation}
\phi{r,t} = \frac{1}{\pi R_0^2}.
\end{equation}

If the domain is shrinking, the radius of the domain at time $t$ is given by $R(t)$.
If the rate of diffusion is large enough,
Then the expected MSD at time $t$ can be calculated:


\begin{equation}
\begin{split}
\left<r^2\right> 	&= \int_{\Omega}r^2 \phi(r,t) d\Omega \\
                 	&= \int_0^{2\pi}\int_0^{\infty} r^2 \phi(r,t) r dr d\theta \\
                	&= \int_0^{2\pi}\int_0^{\infty} \frac{r^3}{\pi R(t)^2} dr d\theta \\
\left<r^2\right>	&= \frac{R(t)^2}{2}\\
\label{eqn:shrink_domain}
\end{split}
\end{equation}

For a decreasing quadratic

\cite{woolley2011stochastic}

Diffusion is modelled using a dicretised ODE model

Discretise the domain $\Omega$ into $N$ boxes $u_i$ each of length $dx=R_0/N$. The diffusion process is described by the finite difference approximations:

\begin{equation}
\frac{du_i}{dt} = \begin{cases}
		d(u_i - u_{i+1}), & \text{for } i = 1\\
		d(u_{i-1}-2u_i +u_{i+1}), & \text{for } 2 \leq i \leq N-1\\
		d(u_{i-1}-u_i), & \text{for } i = N
		\end{cases}
\end{equation}



The PDE for diffusion on a 1D domain with apical growth $l(t)$ is
%
\begin{equation}
\D{\phi}{t} = \frac{D}{l(t)^2} \DD{\phi}{x} + x \frac{\dot{l(t)}}{l(t)} \D{\phi}{x} .
\end{equation}
%
For growth of the form $l = \mathrm{e}^{rt}$, the PDE becomes
%
\begin{equation}
\D{\phi}{t} = \frac{D}{\mathrm{e}^{2rt}} \DD{\phi}{x} + x r \D{\phi}{x} .
\label{eqn:exp_apical}
\end{equation}

Using
%
\begin{equation}
\D{}{x} (rx\phi) = r\phi + xr \D{\phi}{x},
\end{equation}
%
\eqn{eqn:exp_apical} can be written as
%
\begin{equation}
\D{\phi}{t} = \D{}{x}\left( \frac{D}{\mathrm{e}^{2rt}}\D{\phi}{x} + xr \phi \right) - ru .
\end{equation}
%
Integrating the convection term gives
\begin{equation}
\int_0^x x'r\D{\phi}{x'} dx = xr\phi - r\int_0^x \phi dx ,
\end{equation}
%
and therefore \eqn{eqn:exp_apical} can also be written as
%
\begin{equation}
\D{\phi}{t} = \D{}{x}\left( \frac{D}{\mathrm{e}^{2rt}}\D{\phi}{x} + xr \phi - r\int_0^x \phi dx \right) .
\end{equation}
