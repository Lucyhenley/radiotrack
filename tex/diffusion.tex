\documentclass{article}


\usepackage{amsmath,amssymb,mathtools,amsthm} % Math

\newcommand{\chap}[1]{Chapter \ref{#1}}
\newcommand{\eqn}[1]{equation \eqref{#1}}
\newcommand{\D}[2]{\frac{\partial #1}{\partial #2}}
\newcommand{\DD}[2]{\frac{\partial^2 #1}{\partial #2^2}}

\begin{document}

\section{A diffusion model for bat movement}

Diffusion models are widely used to model animal movement for a number of species \cite{Ovaskainen2016}. In this case, a 2D diffusion model is used to describe the movement of bats flying away from the roost.

If the roost is at $(x_0,y_0)$ and bats leave the roost at time $t =0$,
the 2D diffusion equation describes the probability density $\phi(x,y,t)$ of
finding a bat at position $(x,y)$ at time $t$,

\begin{equation}
  \D{\phi(x,y,t)}{ t} = D \nabla^2 \phi(x,y,t) ,
  \nonumber
\end{equation}
%
where $D$ is the diffusion coefficient and quantifies the speed with which bats diffuse.

The diffusion equation can also be written in polar coordinates:

\begin{equation}
\D{ \phi(r,t)}{t} = \frac{D}{r} \D{}{ r} \left( r \D{\phi(r,t)}{r} \right),
\end{equation}
%
where $r$ is the distance from the roost, given by $r=\sqrt{(x-x_0)^2 + (y-y_0)^2}$. As diffusion is symmetric, $\phi$ is only dependent on $r$ and not on the angle.

The initial condition

\begin{equation}
\phi(x_0,y_0,0) = \delta(x_0,y_0)
\label{eqn:IC}
\end{equation}
%
specifies that all bats begin the night at the roost.

Motion is determined by the parameter $D$, the diffusion coefficient. The diffusion coefficient determines the speed of motion away from the origin and can be calculated using the mean squared distance (MSD) $\mathbb{E}[r^2]$. The expected MSD at time $t$ can be calculated using the diffusion equation:

\begin{equation}
\left<r^2\right> = \int_{\infty}r^2 \phi(r,t) d\Omega ,
\end{equation}
%
where the integral is over all space. 

The rate of change of the MSD with time can be used to calculate the relationship between MSD and time:

\begin{equation}
\begin{split}
\frac{d}{dt} \left<r^2\right> &= \frac{d}{dt}\int_{\infty}r^2 \phi(r,t) d\Omega \\
                            &= \frac{d}{dt} \int_0^{2\pi}\int_0^{\infty} r^2 \phi(r,t) r dr d\theta \\
                           &= \int_0^{2\pi} \int_0^{\infty} r^3 \D{\phi}{l t} dr d\theta \\
                            &= \int_0^{2\pi} \int_0^{\infty} r^3 \frac{D}{r} \D{}{r } \left( r \D{ \phi}{ r}\right) dr d\theta \\
                            &= \int_0^{2\pi} \left( \left[ D r^2 \left( r \D{ \phi}{ r}\right) \right]_0^{\infty} - \int_0^{\infty} 2rD \left(r \frac{\partial \phi}{\partial r} \right) dr \right) d\theta \\
                            &= \int_0^{2\pi} \int_0^{\infty} -2r^2D \D{ \phi}{r}dr d\theta \\
                            &= \int_0^{2\pi} \left( \left[-2r^2D \phi \right]_0^{\infty} + \int_0^{\infty} 4rD \phi dr \right)d\theta \\
                            &= \int_0^{2\pi} \int_0^{\infty} 4rD\phi dr d\theta \\
                            &= 4D \int_{\infty} \phi d\Omega \\
\frac{d}{dt} \left<r^2\right>  &= 4D \\
\label{eqn:diffusion_1}
\end{split}
\end{equation}

Integrating with respect to time gives

\begin{equation}
\left<r^2\right> = 4Dt ,
\label{eqn:diffusion_msd}
\end{equation}

and therefore MSD is directly proportional to time.


\subsection{Advection-Diffusion model}

A simple diffusion model does not 

Bats don't diffuse for the whole night

Maximum distance away from the roost: sustenance zone

Return to roost in the morning

Motion in later portion of the night is described by an advection-diffusion model:

\begin{equation}
  \D{\phi(x,y,t)}{t} = D \nabla^2 \phi(x,y,t) - \chi \nabla \phi(x,y,t),
  \nonumber
\end{equation}

where $\chi$ is the strength of the drift back towards the roost.

The same method can be used to calculate the MSD for the advection-diffusion model in terms of $D$ and $\chi$:

\begin{equation}
\begin{split}
\frac{d}{dt} \left<r^2\right> &= \frac{d}{dt}\int_{\infty}r^2 \phi(r,t) d\Omega \\
                              &= \frac{d}{dt} \int_0^{2\pi}\int_0^{\infty} r^2 \phi(r,t) r dr d\theta \\
                              &= \int_0^{2\pi} \int_0^{\infty} r^3 \D{\phi}{t} dr d\theta \\
                              &= \int_0^{2\pi} \int_0^{\infty} r^3 \frac{D}{r} \D{}{r}\left(r \D{\phi}{r}\right)- r^3\chi\D{\phi}{r} dr d\theta \\
\end{split}
\end{equation}


Substituting the first term from \eqn{eqn:diffusion_1}:

\begin{equation}
\begin{split}
                              &= 4D - \chi \int_0^{2\pi} \int_0^{\infty} r^3 \D{\phi}{r} dr d\theta \\
                              &= 4D - \chi \int_0^{2\pi} \left[ r^3 \phi \right]_0^{\infty} - \int_0^{\infty}  3 r^2 \phi dr d\theta \\
                              &= 4D + 3\chi \int_0^{2\pi} \int_0^{\infty}  r^2 \phi dr d\theta  \\
                              &= 4D + 3\chi \int_{\infty} r \phi d\Omega \\
                              &= 4D + 3\chi \left<r\right>
\nonumber
\end{split}
\end{equation}

The mean distance $ \left<r\right>$ can be calculated in the same way:

\begin{equation}
\begin{split}
\frac{d}{dt} \left<r\right> &= \frac{d}{dt}\int_{\infty}r \phi(r,t) d\Omega \\
                              &= \frac{d}{dt} \int_0^{2\pi}\int_0^{\infty} r \phi(r,t)r dr d\theta \\
                              &= \int_0^{2\pi} \int_0^{\infty} r^2 \D{\phi}{t} dr d\theta \\
                              &= \int_0^{2\pi} \int_0^{\infty} r^2 \left[ \frac{D}{r} \D{}{r}\left(r \D{\phi}{r}\right)-\chi\D{\phi}{r} \right] dr d\theta \\
                              &= \int_0^{2\pi} \int_0^{\infty} rD  \D{}{r}\left(r \D{\phi}{r}\right) -  \chi r^2 \D{\phi}{r} dr d\theta \\
\nonumber
\end{split}
\end{equation}


For simple diffusion, $\left<r\right> = 0$ and


\begin{equation}
\int_0^{2\pi} \int_0^{\infty} rD  \D{}{r}\left(r \D{\phi}{r}\right) = 0 .
\end{equation}

\begin{equation}
\begin{split}
\frac{d}{dt} \left<r\right> &= \int_0^{2\pi} \int_0^{\infty} -  \chi r^2 \D{\phi}{r} dr d\theta \\
                            &= \int_0^{2\pi} - \left[ \chi r^2 \phi \right]_0^{\infty} + \int_0^{\infty} 2 \chi r \phi dr d\theta \\
                            &= \int_0^{2\pi} \int_0^{\infty} 2 \chi r \phi dr d\theta \\
                            &= \int_{\infty} 2 \chi \phi d\omega \\
                            &= 2\chi
\nonumber
\end{split}
\end{equation}

So then:

\begin{equation}
\left<r\right> = 2\chi t
\nonumber
\end{equation}

Substituting into the equation for $\left<r^2\right>$ gives

\begin{equation}
\begin{split}
\frac{d}{dt} \left<r^2\right> &= 4D + 6\chi^2t \\
\left<r^2\right>              &= 4Dt + 3\chi^2t^2 ,
\nonumber
\end{split}
\end{equation}
%


\section{Diffusion modelling}

Bats leave the roost after sunset and fly away from the roost in search of food.
Foraging bats are modelled as diffusive particles for the first hour after
sunset. If the roost is at the origin and bats leave the roost at time $t =0$,
the 2D diffusion equation describes the probability density $\phi(x,y,t)$ of
finding a bat at position $(x,y)$ at time $t$,

\begin{equation}
  \frac{\partial \phi(x,y,t)}{\partial t} = D \nabla^2 \phi(x,y,t) ,
  \nonumber
\end{equation}
%
where $D$ is the diffusion coefficient, calculated using the mean squared distance bats travel from the roost over time. A typical value is calculated using radio tracking surveys is $D = 75.5$ m\textsuperscript{2}s\textsuperscript{-1}. $\phi(x,y,0)$ is
the Dirac delta function.

Solving for $\phi(x,y,t)$ gives

\begin{equation}
  \phi(x,y,t) = \frac{1}{4 \pi D t} e^{-\frac{x^2 + y^2}{4Dt}}. \cite{Ovaskainen2016}
  \nonumber
\end{equation}

This model assumes infinite space and there are no boundary conditions, so that bats can travel any distance away from the roost. However, it can be shown that the probability of a bat travelling further than the maximum foraging radius $R_f =$ 3000m calculated using $\phi$ is typically small. Integrating $\phi$ using a numerical integration algorithm, \cite{Berntsen1991}, over the time we will be considering (the first hour of movement after sunset) and over space for $x^2 + y^2 > R_f^2$ gives a proability that a bat will exceed $R_f$ of $p = 4.5 \times 10^{-6}$.

Integrating $\phi(x,y,t)$ over the range $R$ of a detector $i$ at $(x_i, y_i)$ gives
the expected density of bats in range of detector $i$ at time $t$. The
integral is approximated using a quadratic Taylor expansion around $(x_i, y_i)$:
%Need to add details of Taylor expansion and integral for thesis


\begin{equation}
  N_i(t) = \frac{1}{4Dt} e^{-\frac{x_i^2 + y_i^2}{4Dt}} \left[ R^2 + \frac{R^4}{4} \left( \frac{x_i^2 + y_i^2}{8D^2t^2} - \frac{1}{2Dt} \right)\right]
\end{equation}
%
where $R$ is the detector range. The total number of hits expected at each
detector for $0 < t < T$ is calculated by integrating over time,

\begin{equation}
  \tilde{N_i}(T) = \int_{t=0}^{t=T} N_i(t) dt.
\end{equation}
%
At present, this is approximated using a Riemann sum at intervals $\Delta t$:

\begin{equation}
  \tilde{N_i}(T) \simeq \sum_{t=0}^{t=T} N_i(t) \Delta t.
  \label{eqn:expect_hits}
\end{equation}
%
In future, this approximation will be improved by using a better numerical
integration method. The number of hits expected at each detector $\tilde{N_i}(T)$ for a possible roost location $\bm{\theta'}$ will be compared to the number of hits found in bat surveys to estimate the probability that the true roost is at $\bm{\theta'}$.




\end{document}