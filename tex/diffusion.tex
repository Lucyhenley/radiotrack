\section{A diffusion model for bat movement} \label{diffusion}

Diffusion models are widely used to model animal movement, specifically dispersal, for a number of
species \cite{Ovaskainen2016}. In this case, a 2D diffusion model is used to
describe dispersal during phase 1 of movement as bats fly away from the roost. The third dimension is not
included as height is not measured in the radio tracking survey, and it is not
needed to describe landscape use. Since it is commonly accepted that bats tend
to forage within the Core Sustenance Zone and remain within a certain distance
 of the roost, a diffusion model on a bounded domain is considered here.

If the roost is at $(x_0,y_0)$ and bats leave the roost at time $t =0$,
the 2D diffusion equation describes the probability density $\phi(x,y,t)$ of
finding a bat at position $(x,y)$ at time $t$,
%
\begin{equation}
  \D{\phi(x,y,t)}{t} = D \nabla^2 \phi(x,y,t) ,
  \label{eqn:diffusion_cartesian}
\end{equation}
%
where $D$ is the diffusion coefficient, a positive constant that quantifies the
 rate of spread. The Core Sustenance Zone is modelled as a disk of radius $R$ centred around the roost, $\Omega \subset \mathbb{R}^2$ and therefore we will consider the diffusion equation in polar coordinates,
 %
 \begin{equation}
 \D{ \phi(r,t)}{t} = \frac{D}{r} \D{}{ r} \left( r \D{\phi(r,t)}{r} \right),
 \label{eqn:diffusion_polar}
 \end{equation}
 %
 where $r$ is the distance from the roost, given by $r=\sqrt{(x-x_0)^2 +
 (y-y_0)^2}$. Since diffusion is on a symmetric circular domain, $\phi$ is only dependent on $r$ and not
  on the angle. The boundary condition,
%
\begin{equation}
  \D{\phi(r=R,t)}{r} = 0,
\end{equation}
%
specifies zero-flux across the boundary. The initial condition,
%
 \begin{equation}
 \phi(r = 0) = \delta(0),
 \label{eqn:IC}
 \end{equation}
%
specifies that all bats begin the night at the roost before moving away to begin foraging.

The relationship between the expected mean squared displacement (MSD) and time $t$
can be calculated using the probability density $\phi$,
%
\begin{equation}
\left<r^2\right> = \int_{\Omega}r^2 \phi(r,t) d\omega ,
\label{eqn:MSD_expectation}
\end{equation}
%
where $\omega = (r,\theta) \subset \Omega$. Taking the time derivative of both
sides and substituting
\begin{equation}
    \D{\phi(r,t)}{t}
\end{equation} from \eqn{eqn:diffusion_polar},
%
\begin{align}
\frac{d}{dt} \left<r^2\right> &= \frac{d}{dt}\int_{\Omega}r^2 \phi(r,t) d\omega ,\\
                           &= \int_0^{2\pi} \int_0^{R} r^3 \D{\phi}{t} dr d\theta ,\\
                            &= \int_0^{2\pi} \int_0^{R} r^3 \frac{D}{r} \D{}{r } \left( r \D{ \phi}{ r}\right) dr d\theta , \\
                            &= \int_0^{2\pi} \left( \left[ D r^2 \left( r \D{ \phi}{ r}\right) \right]_0^{R} - \int_0^{R} 2rD \left(r \frac{\partial \phi}{\partial r} \right) dr \right) d\theta , \\
                            &= \int_0^{2\pi} \int_0^{R} -2r^2D \D{ \phi}{r}dr d\theta , \\
                            &= \int_0^{2\pi} \left( \left[-2r^2D \phi \right]_0^{R} + \int_0^{R} 4rD \phi dr \right)d\theta , \\
                            &= - 4\pi R^2D \phi(R,t) + 4D \int_{\Omega} \phi d\omega ,
\label{eqn:diffusion_1}
\end{align}
%
and therefore,
\begin{equation}
\frac{d}{dt} \left<r^2\right>  = 4D( 1- \pi R^2 \phi(R,t)) .
\end{equation}
%
Integrating with respect to time gives
%
\begin{equation}
\left<r^2\right> = 4D \left( t - \pi R^2 \int_0^t \phi(R,\tau) d \tau \right).
\label{eqn:diffusion_msd}
\end{equation}
%
Over short timescales, $\phi(R,t) \approx 0$, since the probability of a reaching the boundary over a short period of time is small due to the initial condition. Therefore, over a short timescale, the expected MSD for diffusion is directly proportional to time,
%
\begin{equation}
\left<r^2\right> \approx 4Dt.
\label{eqn:diffusion_short}
\end{equation}
%
\subsection{A discretised diffusion model} \label{discretised_model}

The diffusion equation in a bounded domain can be solved using a discretised ODE
description of the diffusion equation \cite{woolley2011stochastic}. For a circular domain in polar coordinates $\Omega = [0,R] \times [0, 2\pi]$, the domain can be discretised into $N$ boxes, each of
length $\Delta x=R/N$. The probability density in each box $i$ is denoted by $\phi_i$, and evolves over time according to the diffusion process. A finite difference approximation is used to describe the movement of probability density between boxes,
%
\begin{equation}
\frac{d\phi_i}{dt} = \begin{cases}
		d(\phi_i - \phi_{i+1}), & \text{for } i = 1, \\
		d(\phi_{i-1}-2\phi_i +\phi_{i+1}), & \text{for } 2 \leq i \leq N-1, \\
		d(\phi_{i-1}-\phi_i), & \text{for } i = N ,
		\end{cases}
        \label{eqn:discrete_diffusion}
\end{equation}
%
where the discretised diffusion coefficient is given by
$d = D/(\Delta x)^2$. The initial condition corresponding to \eqn{eqn:IC} means that probability density is concentrated in the first box,
%
\begin{equation}
\phi_i(0) = \begin{cases}
		\frac{1}{\Delta x}, & \text{for } i = 1, \\
		0, & \text{for } 2 \leq i \leq N. \\
		\end{cases}
        \label{eqn:discrete_diffusion_IC}
\end{equation}
%
\begin{figure} [t]
    \centering
        \includegraphics[width=0.5\textwidth]{diffusion_diagram.pdf}
        \caption{A diagram to illustrate the movement of probability density between boxes in the discretised diffusion model. Diffusion between boxes is represented by $d$ and the probability density in each box $i$ is denoted by $\phi_i$}
    \label{fig:diffusion_diagram}
\end{figure}

%
A diagram illustrating the spread of probability density due to the diffusion process is shown in \fig{fig:diffusion_diagram}.  This generates a system of $N$ ODEs describing motion over
the domain at each timestep and which can be solved using a numerical ODE solver.
The equations for $i=1$ and $i=N$ correspond to reflective, zero-flux boundary
conditions. The discretised model was simulated using DifferentialEquations.jl, a package for solving differential equations in Julia \cite{DifferentialEquations}. The result of a simulation with $N = 100$ boxes in a domain of
length $R = 1000$m and diffusion coefficient $D = 100\mathrm{ms^{-2}}$ is shown in
\fig{fig:discretised_phi}. The diffusion process spreads the probability density out from the left side of the domain, where $\phi$ is high and tends to homogenise the probability density across the domain over time. After 5 hours, the probability density $\phi$ is evenly spread throughout the domain and the probability distribution is eventually uniform.

 \begin{figure} [t]
     \centering
         \includegraphics[width=0.8\textwidth]{discretised_phi.png}
         \caption{The value of $\phi$ for a discretised diffusion simulation with parameters $N = 100$, $R = 2000\mathrm{m}$ and $D = 100\mathrm{ms^{-2}}$ and initial condition given by \eqn{eqn:discrete_diffusion_IC} after $t = $ 1 hour, 2 hours and 5 hours.}
     \label{fig:discretised_phi}
 \end{figure}
