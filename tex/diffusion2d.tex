\section{A diffusion model in two dimensions}


Diffusion models are widely used to model animal movement, specifically dispersal, for a number of
species \cite{Ovaskainen2016}. In this case, a 2D diffusion model is used to
describe dispersal during phase 1 of movement as bats fly away from the roost. The third dimension is not
included as height is not measured in the radio tracking survey, and it is not
needed to describe landscape use. Since it is commonly accepted that bats tend
to forage within the Core Sustenance Zone and remain within a certain distance
 of the roost, a diffusion model on a bounded domain is considered here.

If the roost is at $(x_0,y_0)$ and bats leave the roost at time $t =0$,
the 2D diffusion equation describes the probability density $\phi(x,y,t)$ of
finding a bat at position $(x,y)$ at time $t$,
%
\begin{equation}
  \D{\phi(x,y,t)}{t} = D \nabla^2 \phi(x,y,t) ,
  \label{eqn:diffusion_cartesian2d}
\end{equation}
%
where $D$ is the diffusion coefficient, a positive constant that quantifies the
 rate of spread. The Core Sustenance Zone is denoted by $\Omega \subset \mathbb{R}^2$ and modelled as a disk of radius $R$ centred around the roost. Therefore we will consider the diffusion equation in polar coordinates,
 %
 \begin{equation}
 \D{ \phi(r,t)}{t} = \frac{D}{r} \D{}{ r} \left( r \D{\phi(r,t)}{r} \right),
 \label{eqn:diffusion_polar2d}
 \end{equation}
 %
 where $r$ is the distance from the roost, given by $r=\sqrt{(x-x_0)^2 +
 (y-y_0)^2}$. Since the domain is symmetric, $\phi$ is only dependent on $r$ and not
  on the angle. The initial condition,
%
 \begin{equation}
 \phi(r = 0) = \delta(0),
 \label{eqn:IC2d}
 \end{equation}
%
specifies that all bats begin the night at the roost before moving away at time $t$ to begin foraging. The boundary condition,
%
\begin{equation}
\D{\phi(r=R,t)}{r} = 0,
\label{eqn:BC}
\end{equation}
%
specifies zero-flux across the boundary.

\subsection{Discretising the diffusion model}

The diffusion equation in a bounded domain can be solved using a discretised ODE
description of the diffusion equation \cite{woolley2011stochastic}. For a circular domain in polar coordinates $\Omega = [0,R] \times [0, 2\pi]$, the domain can be discretised into $N$ annuli, each of
width $h=R/N$.The probability density in each annulus $i$ is denoted by $\phi_i$, and evolves over time according to the diffusion process.  The distance from the origin to the inner edge of annulus $i$ is given by $r_i$.  A central difference approximation to \eqn{eqn:diffusion_polar2d} is used to describe the movement of probability density between boxes.

The central difference approximations to first and second order derivatives at annulus $i$ are given by
%
\begin{align}
 \at{\D{\phi}{r}}{r=r_i}&= \frac{\phi(r_i+h)-\phi(r_i-h)}{2 h} ,\label{eqn:diffusion_discrete1} \\
 \at{\DD{\phi}{r}}{r=r_i}&= \frac{\phi(r_i-h)-2\phi(r_i)+\phi(r_i+h)}{h^2} .\label{eqn:diffusion_discretei}
\end{align}
%
The central difference approximation to \eqn{eqn:diffusion_polar2d} is then
%
\begin{equation}
 \at{\D{\phi}{t}}{r=r_i} = D \left(\frac{\phi(r-h)-2\phi(r)+\phi(r+h)}{h^2} + \frac{1}{r} \frac{\phi(r+h)-\phi(r-h)}{2 h} \right).
 \label{eqn:diffusion_polar_cd}
\end{equation}
%
The distance $r_i$ at strip $i$ is given by
\begin{equation}
r = i h ,
\end{equation}
%
and \eqn{eqn:diffusion_polar_cd} can therefore be written as
%
\begin{equation}
\D{\phi(r)}{t} = \frac{D}{h^2} \left(\phi(r-h)-2\phi(r)+\phi(r+h)\right) + \frac{D}{2ih^2}\left(\phi(r+h)-\phi(r-h) \right).
\label{eqn:diffusion_polar_final}
\end{equation}
%
Changing notation for the discretised version:
%
\begin{equation}
\frac{d\phi_i}{dt} = \frac{D}{h^2}(\phi_{i-1}-2\phi_i +\phi_{i+1}) + \frac{D}{2ih^2} (\phi_{i+1}-\phi_{i})
        \label{eqn:discrete_diffusion_i}
\end{equation}

The boundary condition in \eqn{eqn:BC} specifies that bats in strips at the edge of the domain in trajectories that would pass through the boundary are reflected back in the direction of the roost.


From \eqn{eqn:diffusion_discrete1}, the equation for strip 1 at $r=0$ is
%
\begin{equation}
\frac{d\phi_1}{dt} = \frac{D}{h^2}(\phi_{0}-2\phi_1 +\phi_{2}) + \frac{D}{2h^2} (\phi_{2}-\phi_{1}).
\end{equation}
%
However, due to the reflective boundary condition, any bats that would have pass through the boundary from strip 1 to an imaginary strip 0 are instead reflected back, and therefore the value of $\phi$ in the imaginary strip 0 is the same as in strip 1, and $\phi_0$ = $\phi_1$.
%
\begin{equation}
\frac{d\phi_1}{dt} = \frac{D}{h^2}(\phi_{2}- \phi_1) + \frac{D}{2h^2} (\phi_{2}-\phi_{1}) .
        \label{eqn:strip_1}
\end{equation}

Similarly, from \eqn{eqn:diffusion_discrete1}, for strip $n$ at $r=R$,
%
\begin{equation}
\frac{d\phi_n}{dt} = \frac{D}{h^2}(\phi_{n-1}-2\phi_n +\phi_{n+1}) + \frac{D}{2h^2} (\phi_{n+1}-\phi_{n}) .
\end{equation}
%
Due to the reflective boundary condition between strip $n$ and strip $n+1$, $\phi_n+1 = \phi_n$, and
\begin{equation}
\frac{d\phi_n}{dt} = \frac{D}{h^2}(\phi_{n-1}-\phi_n) + \frac{D}{2h^2} (\phi_{n+1}-\phi_{n-1}) .
        \label{eqn:strip_n}
\end{equation}

 A diagram of the motion is shown in \fig{fig:2d_diffusion_diag}. Gathering up \eqnto{eqn:discrete_diffusion_i}{eqn:strip_n}, the set of equations describing the full system is
%
  \begin{equation}
  \frac{d\phi_i}{dt} = \begin{cases}
  		\frac{D}{h^2}(\phi_{i+1} - \phi_{i}) + \frac{D}{2h^2} (\phi_{i+1}-\phi_i), & \text{for } i = 1, \\
  		\frac{D}{h^2}(\phi_{i-1}-2\phi_i +\phi_{i+1}) + \frac{D}{2ih^2} (\phi_{i+1}-\phi_{i}), & \text{for } 2 \leq i \leq N-1, \\
  		\frac{D}{h^2}(\phi_{i-1}-\phi_i) - \frac{D}{2ih^2} (\phi_{i}-\phi_{i-1}), & \text{for } i = N .
  		\end{cases}
          \label{eqn:discrete_diffusion2d}
  \end{equation}


 A 2D discretised diffusion simulation with $N=1000$ particles and diffusion coefficient $D=65$ms\textsuperscript{-2} was run using DifferentialEquations.jl \cite{DifferentialEquations} to validate the model. The mean squared displacement is shown in \fig{fig:2d_diffusion}. The mean squared displacement for the discretised simulation is initially linear and equal to $4Dt$ before the curve flattens and tends towards a constant value. This is consistent with the mean squared displacement expected for a bounded diffusion model, calculated in section \ref{diffusion}.

  \begin{figure}
  \centering
      \includegraphics[width=0.5\textwidth]{figs/2d_diffusion_diag.pdf}
      \caption{A diagram of the discretised diffusion process. Each box represents an annular region around the roost centre. The probability density shifts between boxes due to the diffusion process, denoted by $d$. Reflective boundary conditions ensure that probability density never leaves the domain.}
  \label{fig:2d_diffusion_diag}
  \end{figure}

\begin{figure}
\centering
    \includegraphics[width=0.8\textwidth]{figs/2d_diffusion.png}
    \caption{The mean squared displacement for a discretised diffusion simulation with $N=1000$ particles in a bounded 2D domain for $D=65$ms\textsuperscript{-2} compared to the analytical result of $4Dt$ from section \ref{diffusion}.}
\label{fig:2d_diffusion}
\end{figure}
%
\section{Results}
%

\begin{figure}
\centering
    \includegraphics[width=0.7\textwidth]{figs/2dshrink_sweep.png}
    \caption{A parameter sweep of parameters $R_0$ and $t_s$, denoting the initial domain radius and time at which the domain begins to shrink. The coefficient of determination $R^2$ is used as a measure of fit. The set of parameters with the highest coefficient of determination is marked with a cross, $R_0$ = 1450m and $t_0$ = 0s.}
\label{fig:2dshrink_sweep}
\end{figure}

\begin{figure}
\centering
    \includegraphics[width=0.6\textwidth]{figs/2dshrink_result.png}
    \caption{
    The MSD for a deterministic diffusion model on a shrinking domain of size $R(t) = \sqrt{R_0^2 - \frac{R_0^2}{t_s^2} t^2}$ with parameters $R_0$ = 1450m and $t_s$ = 0s.}
\label{fig:2dshrink_result}
\end{figure}
