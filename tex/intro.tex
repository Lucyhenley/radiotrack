\chapter{Introduction} \label{chap:intro}

  Bats play an important role in the UK ecosystem, as they help control insect
  populations \cite{Kunz2011} and act as ecological indicators \cite{Jones2009}.
  However, they are susceptible to human impacts due to their sensitivity to
  light, noise and temperature. Additionally, habitat fragmentation due to roads
  and building work can reduce foraging opportunities and lead to a significant
  risk of population decline \cite{rossiter2000genetic}.
 As a result, bat habitats are protected by law in
Europe under the EUROBATS agreement \cite{Eurobats}. Therefore, identifying
roosts and important foraging areas is an important step in ensuring that
habitats remain protected. One measure often used to identify key patches of
habitat is the Core Sustenance Zone, the area surrounding a roost within which
habitat quality has a significant influence on the resilience of the colony
using the roost \cite{CSZ}. Bats are expected to remain within this area for
the majority of the time whilst foraging, however this approach does not provide
any insight into the usage of land inside the Core Sustenance Zone. Here we will consider the movment of Greater Horseshoe Bats, a species highly dependent on linear
landscape features, such as hedgerows and waterways for movement whilst foraging \cite{froidevaux2017factors, duverge1994greater}.

  Radio tracking surveys are commonly used to identify the habitat use of bats
   by using radio transmitters to
 locate bats during foraging
   \cite{Bontadina2002,
Encarnacao2005}.
  In order to
 track bats, they must first be caught using a
humane method such as a harp trap.
 Once a bat is caught, a radio
   transmitter
is attached to its back and it is
 released to forage. The
  signal from a
transmitter is then picked up by field
 workers using scanning radio-receivers, and the position of the bat is
 estimated. Field workers then follow the bat and attempt to maintain contact
 throughout the night, taking regular recordings of location until the signal is
 lost, or the bat returns to
the roost. Due to the nature of the tracking,
 locations are not recorded at regular intervals, and instead when the signal is
 found. Although useful, radio
tracking surveys are highly labour intensive as
 they require first locating a
roost in order to catch bats, and then teams of
 workers following each bat over the entire night. However, these methods have drawbacks. Transmitters have a limited
 range, and workers therefore
must remain close to the bat in order to pick up
 the signal, which can often be
difficult in a rural environment with obstacles
 such as impassable waterways
and hedgerows. Additionally,
  transmitters have a
 limited battery life and are
often detached and lost before the end of the
 survey, meaning that surveys can
be cut short.

\section{Mathematical models for animal movement}

  Mathematical models are an
 invaluable tool in understanding
ecological
  mechanisms as they help us to
 understand the ecological mechanisms
that lead to certain patterns in behaviour
 \cite{Ovaskainen2016}. There are
many possible formalisms, such as stochastic,
 deterministic, spatial, or
discrete depending on the population and behaviour
 \cite{Murray2011}.

  will be to derive deterministic
 partial differential equation
(PDE) models to compare with data to describe
  bat movement during foraging. Various PDE models will be considered here. Diffusion models in both one and two dimensions are discussed to describe the dispersal of bats away from the roosts. Two models are used to describe movement for the remainder of the night whilst bats are foraging, a convection-diffusion model and a model describing diffusion on a shrinking domain. Convection-diffusion models are widely used in ecology to model population migration, however we will show here that a convection-diffusion model is not consistent with radio tracking data. Instead, a shrinking domain diffusion model provides a better description of bat movement whilst foraging.

 \section{Bayesian Statistics}
 Bayesian
statistics is a statistical
paradigm for updating knowledge about a
 parameter
given related data
\cite{Gelman2013}. The probability distribution of a
parameter
 $\theta$
conditioned on observations $\bm{Y} = \{ Y_1, Y_2, ..., Y_n
\}$ is
 given by
Bayes' Theorem,

 \begin{equation}
   p(\theta \mid \bm{Y})
\propto
p(\bm{Y} \mid \theta) p(\theta)
   \nonumber,
 \end{equation}
 %
 where
$p(\theta \mid \bm{Y})$ is the posterior probability distribution, formally
describing the probability that the parameter value is $\theta$ given
observations $\bm{Y}$. The likelihood is given by $p(\bm{Y} \mid \theta)$ and
is the probability of observing $\bm{Y}$ if the parameter value is $\theta$.
The
 prior distribution is $p(\theta)$, describing the initial knowledge of
possible
 parameter values.

 \subsection{Approximate Bayesian Computation
(ABC)}
 Approximate Bayesian Compuation (ABC) is an approach to Bayesian
inference using
 simulation and random sampling \cite{Beaumont2002,
Sisson2010} and is widely used for problems where the analytical form of the likelihood function is intractable.
 ABC replaces the calculation
 of the likelihood function
$p(\bm{Y}
\mid \theta)$ with simulation of a model using a specific parameter
value
$\theta'$ to produce an artificial dataset
 $\bm{X}$. Then, some
distance metric
$\rho (\bm{X}, \bm{Y}) $, usually defined
 as a distance
between summary
statistics of $\bm{X}$ and $\bm{Y}$, is used to
 compare
simulated data $\bm{X}$
to observations $\bm{Y}$. If $\rho (\bm{X},
 \bm{Y}) $
is smaller than some
threshold value $\epsilon$, the simulated data is
 close
enough to observations
that the candidate parameter $\theta'$ has some
nonzero probability of being in
the posterior distribution $ p(\theta \mid \bm{Y})$, and the sample $\theta'$
is accepted into the simulated posterior
distribution. This is repeated until
the desired sample size is reached. For
small $\epsilon$, assuming that the
model is correct, the simulated posterior
distribution produced approximates
the
 true posterior $ p(\theta \mid
\bm{Y})$.

 First, the threshold parameter
$\epsilon$ and sample size $n$
are fixed and the
 prior distribution for
parameters $\theta$ is
set, $p(\theta)$. The pseudocode is as
follows:
 %
 \begin{algorithmic}
\While {$i < n$}
     \State Sample $\theta'$ from $p(\theta)$
     \State Simulate $\bm{X}$ from $theta'$
     \State $\rho \gets \mid \bm{X} - \bm{Y} \mid$
\If {$\rho < \epsilon$}
  \State $\theta_i \gets \theta'$
  \State $i \gets i + 1$
\EndIf
\EndWhile
 \end{algorithmic}
 %
The parameters $\theta_i$ are an approximate sample from the posterior distribution $p(\theta \mid \bm{Y})$, and the posterior mean $\mathbb{E}(\theta \mid \bm{Y})$ can be estimated using

\begin{equation}
\frac{1}{n} \sum_i \theta_i .
\end{equation}
