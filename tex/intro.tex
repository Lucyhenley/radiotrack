\section{Introduction}

Bats are very cool and useful, ecosystem services etc.
Difficult to study because tiny, nocturnal, elusive.
Therefore not much known about their movement while foraging.
Radio tracking useful method!



Bats play an important role in the UK ecosystem as they control insect
populations \cite{Kunz2011} and act as ecological indicators \cite{Jones2009}.
However, they are susceptible to human impacts due to their sensitivity to
light, noise and temperature, especially during hibernation \cite{Jones2009}. As
a result, many species are under threat, and bats are now protected by law under
the EUROBATS agreement \cite{Eurobats}. The breeding sites or resting places of
bats are known as roosts, and can be found in trees and buildings as well as
underground in caves and mines \cite{Eurobats}. Some species are highly site
faithful, returning to the same roosts year after year and are therefore reliant
on the suitability of the surrounding habitat \cite{Lewis1995}. Locating roosts
is important for conservation, in ensuring the roosts as well as the surrounding
habitat is preserved. Locating roosts is also important for research: monitoring
bat populations tends to be challenging as most species are small, nocturnal and
elusive, and roost surveys provide a useful way of studying populations
\cite{Flaquer2007}.
