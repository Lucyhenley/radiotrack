\section{Radoiotracking data}


A study was conducted at Greater Horseshoe bat roosts in Devon to identify
roosts used in the area. 12 bats were fit with radio tags and studied over 24
nights. Due to a limited number of workers and limited battery life on the tags,
bats were not tracked every night. Four day roosts were used by bats in the
study, with some bats using different roosts on different days. The roost used
by each bat was identified for 2/3 of bat-nights. For this analysis, only the
data from nights when a bat's roost was known was used, since the dispersal
from the roost is important. As bats were followed each night until the signal
was lost, the frequency of recordings reduced over time when signals were lost.
A total of 322 bat locations were used for this analysis, and these are shown
along with the location of day roosts in \fig{fig:radiotrack_locations}. Since
the data is recorded at irregular time intervals locations were linearly
interpolated between recordings at intervals of $\Delta t = 200$ seconds.
The interpolated trajectories for bat 1 are shown in \fig{fig:bat1} along
with the two day roosts used by bat 1 during the survey. These trajectories
indicate that the bat travelled in different directions each night whilst
foraging and remained within 3km of the roost at all times.

%Max distance from roost = 2962m
\begin{figure} [h]
    \centering
        \includegraphics[width=\textwidth]{track_locations.pdf}
        \caption{Locations of day roosts and tracked bat locations recorded during the radio tracking survey.}
    \label{fig:radiotrack_locations}
\end{figure}
%
\begin{figure} [h]
    \centering
        \includegraphics[width=\textwidth]{bat1_locations.pdf}
        \caption{The locations recorded for bat 1 over 6 nights during the study. The locations were interpolated for each night to obtain trajectories.}
    \label{fig:bat1}
\end{figure}
%
The mean-squared distance (MSD) from the roost was calculated using the interpolated positions and is shown in \fig{fig:MSD}. The standard error decreases later in the night as the number of recordings reduces with time. The results show an initial straight line segment up to $t = 1.6$ hours from sunset, followed by a quadratic decrease for the rest of the night until a sharp decrease as the bats return to the roost before sunrise.
%
\begin{figure} [h]
    \centering
        \includegraphics[width=\textwidth]{RadioTrack_MSD.pdf}
        \caption{The mean-sqared distance (MSD) for all radio tracked bats. The standard error is shown as a ribbon.}
    \label{fig:MSD}
\end{figure}
%
\section{A diffusion model for bat movement}
%
Partial differential equations (PDEs) are often used in ecology to describe spatial processes such as animal movement \cite{Holmes1994}. The diffusion model is commonly used to model dispersal \cite{Ovaskainen2016}. In this case, a 2D diffusion model is used to describe the dispersal of bats flying away from the roost.
%
All bats are assumed to leave the roost at position $(x_0,y_0)$ at time $t=0$. The 2D diffusion equation describes the probability density $\phi(x,y,t)$ of
bats at position $(x,y)$ at a later time $t$,
%
\begin{equation}
  \D{\phi(x,y,t)}{t} = D \nabla^2 \phi(x,y,t) ,
  \label{eqn:diffusion_cartesian}
\end{equation}
%
where the diffusion coefficient $D$ quantifies the dispersal speed. Since bats are assumed to move symmetrically, the diffusion equation can also be written in polar coordinates,
%
\begin{equation}
\D{ \phi(r,t)}{t} = \frac{D}{r} \D{}{ r} \left( r \D{\phi(r,t)}{r} \right),
\label{eqn:diffusion_polar}
\end{equation}
%
where $r$ is the distance from the roost, given by $r=\sqrt{(x-x_0)^2 + (y-y_0)^2}$. As diffusion is symmetric, $\phi$ is only dependent on $r$ and not on the angle.
%
The initial condition
%
\begin{equation}
\phi(r,t=0) = \delta(r=0)
\label{eqn:IC}
\end{equation}
%
specifies that all bats begin the night at the roost. The boundary conditions
%
\begin{equation}
\phi(r=\infty,t) = 0
\label{eqn:IC}
\end{equation}
%
and
%
\begin{equation}
\D{\phi(r=\infty,t)}{t} = 0
\label{eqn:IC}
\end{equation}
%
specify that the probability density is flat at $r=\infty$.
%
Motion is determined by the parameter $D$, the diffusion coefficient. The diffusion coefficient determines the speed of dispersal away from the origin and can be calculated using the mean squared distance (MSD) $\mathbb{E}[r^2]$. The expected MSD at time $t$ can be calculated using the diffusion equation,
%
\begin{equation}
\mathbb{E}[r^2] = \int_{\infty}r^2 \phi(r,t) d\Omega ,
\label{eqn:MSD_int}
\end{equation}
%
where the integral is over all space.
%
Differentiating \eqn{eqn:MSD_int} with respect to time and substituting $\D{\phi}{t}$ from \eqn{eqn:diffusion_polar},
%
\begin{equation}
\begin{split}
\frac{d}{dt} \left<r^2\right> &= \frac{d}{dt}\int_{\infty}r^2 \phi(r,t) d\Omega \\
                            &= \frac{d}{dt} \int_0^{2\pi}\int_0^{\infty} r^2 \phi(r,t) r dr d\theta \\
                           &= \int_0^{2\pi} \int_0^{\infty} r^3 \D{\phi}{t} dr d\theta \\
                            &= \int_0^{2\pi} \int_0^{\infty} r^3 \frac{D}{r} \D{}{r } \left( r \D{ \phi}{ r}\right) dr d\theta \\
                            &= \int_0^{2\pi} \left( \left[ D r^2 \left( r \D{ \phi}{ r}\right) \right]_0^{\infty} - \int_0^{\infty} 2rD \left(r \frac{\partial \phi}{\partial r} \right) dr \right) d\theta. \\
\label{eqn:diffusion_1}
\end{split}
\end{equation}
%
$\D{\phi}{r} = 0$ and $\phi(r=\infty)=0$
%
\begin{equation}
\begin{split}
\frac{d}{dt} \left<r^2\right> &= \int_0^{2\pi} \int_0^{\infty} -2r^2D \D{ \phi}{r}dr d\theta \\
                            &= \int_0^{2\pi} \left( \left[-2r^2D \phi \right]_0^{\infty} + \int_0^{\infty} 4rD \phi dr \right)d\theta \\
                            &= \int_0^{2\pi} \int_0^{\infty} 4rD\phi dr d\theta \\
                            &= 4D \int_{\infty} \phi d\Omega \\
\frac{d}{dt} \left<r^2\right>  &= 4D \\
\label{eqn:diffusion_1}
\end{split}
\end{equation}
Integrating with respect to time gives
%
\begin{equation}
\left<r^2\right> = 4Dt ,
\label{eqn:D_msd}
\end{equation}
%
and therefore MSD is directly proportional to time for a simple diffusion model, consistent with the initial straight line segment in the radio tracking data.




\subsection{Advection-Diffusion model}

A simple diffusion model does not

Bats don't diffuse for the whole night

Maximum distance away from the roost: sustenance zone

Return to roost in the morning

Motion in later portion of the night is described by an advection-diffusion model:

\begin{equation}
  \D{\phi(x,y,t)}{t} = D \nabla^2 \phi(x,y,t) - \chi \nabla \phi(x,y,t),
  \nonumber
\end{equation}

where $\chi$ is the strength of the drift back towards the roost.

The same method can be used to calculate the MSD for the advection-diffusion model in terms of $D$ and $\chi$,

\begin{equation}
\begin{split}
\frac{d}{dt} \left<r^2\right> &= \frac{d}{dt}\int_{\infty}r^2 \phi(r,t) d\Omega \\
                              &= \frac{d}{dt} \int_0^{2\pi}\int_0^{\infty} r^2 \phi(r,t) r dr d\theta \\
                              &= \int_0^{2\pi} \int_0^{\infty} r^3 \D{\phi}{t} dr d\theta \\
                              &= \int_0^{2\pi} \int_0^{\infty} r^3 \frac{D}{r} \D{}{r}\left(r \D{\phi}{r}\right)- r^3\chi\D{\phi}{r} dr d\theta \\
\end{split}
\end{equation}


Substituting the first term from \eqn{eqn:diffusion_1},

\begin{equation}
\begin{split}
                              &= 4D - \chi \int_0^{2\pi} \int_0^{\infty} r^3 \D{\phi}{r} dr d\theta \\
                              &= 4D - \chi \int_0^{2\pi} \left[ r^3 \phi \right]_0^{\infty} - \int_0^{\infty}  3 r^2 \phi dr d\theta \\
                              &= 4D + 3\chi \int_0^{2\pi} \int_0^{\infty}  r^2 \phi dr d\theta  \\
                              &= 4D + 3\chi \int_{\infty} r \phi d\Omega \\
                              &= 4D + 3\chi \left<r\right>
\nonumber
\end{split}
\end{equation}

The mean distance $ \left<r\right>$ can be calculated in the same way:

\begin{equation}
\begin{split}
\frac{d}{dt} \left<r\right> &= \frac{d}{dt}\int_{\infty}r \phi(r,t) d\Omega \\
                              &= \frac{d}{dt} \int_0^{2\pi}\int_0^{\infty} r \phi(r,t)r dr d\theta \\
                              &= \int_0^{2\pi} \int_0^{\infty} r^2 \D{\phi}{t} dr d\theta \\
                              &= \int_0^{2\pi} \int_0^{\infty} r^2 \left[ \frac{D}{r} \D{}{r}\left(r \D{\phi}{r}\right)-\chi\D{\phi}{r} \right] dr d\theta \\
                              &= \int_0^{2\pi} \int_0^{\infty} rD  \D{}{r}\left(r \D{\phi}{r}\right) -  \chi r^2 \D{\phi}{r} dr d\theta \\
\nonumber
\end{split}
\end{equation}


For simple diffusion, $\left<r\right> = 0$ and


\begin{equation}
\int_0^{2\pi} \int_0^{\infty} rD  \D{}{r}\left(r \D{\phi}{r}\right) = 0 .
\end{equation}

\begin{equation}
\begin{split}
\frac{d}{dt} \left<r\right> &= \int_0^{2\pi} \int_0^{\infty} -  \chi r^2 \D{\phi}{r} dr d\theta \\
                            &= \int_0^{2\pi} - \left[ \chi r^2 \phi \right]_0^{\infty} + \int_0^{\infty} 2 \chi r \phi dr d\theta \\
                            &= \int_0^{2\pi} \int_0^{\infty} 2 \chi r \phi dr d\theta \\
                            &= \int_{\infty} 2 \chi \phi d\omega \\
                            &= 2\chi
\nonumber
\end{split}
\end{equation}

So then:

\begin{equation}
\left<r\right> = 2\chi t
\nonumber
\end{equation}

Substituting into the equation for $\left<r^2\right>$ gives

\begin{equation}
\begin{split}
\frac{d}{dt} \left<r^2\right> &= 4D + 6\chi^2t \\
\left<r^2\right>              &= 4Dt + 3\chi^2t^2 ,
\nonumber
\end{split}
\end{equation}
%


\section{Diffusion modelling}

Bats leave the roost after sunset and fly away from the roost in search of food.
Foraging bats are modelled as diffusive particles for the first hour after
sunset. If the roost is at the origin and bats leave the roost at time $t =0$,
the 2D diffusion equation describes the probability density $\phi(x,y,t)$ of
finding a bat at position $(x,y)$ at time $t$,

\begin{equation}
  \frac{\partial \phi(x,y,t)}{\partial t} = D \nabla^2 \phi(x,y,t) ,
  \nonumber
\end{equation}
%
where $D$ is the diffusion coefficient, calculated using the mean squared distance bats travel from the roost over time. A typical value is calculated using radio tracking surveys is $D = 75.5$ m\textsuperscript{2}s\textsuperscript{-1}. $\phi(x,y,0)$ is
the Dirac delta function.

Solving for $\phi(x,y,t)$ gives

\begin{equation}
  \phi(x,y,t) = \frac{1}{4 \pi D t} e^{-\frac{x^2 + y^2}{4Dt}}. \cite{Ovaskainen2016}
  \nonumber
\end{equation}

This model assumes infinite space and there are no boundary conditions, so that bats can travel any distance away from the roost. However, it can be shown that the probability of a bat travelling further than the maximum foraging radius $R_f =$ 3000m calculated using $\phi$ is typically small. Integrating $\phi$ using a numerical integration algorithm, \cite{Berntsen1991}, over the time we will be considering (the first hour of movement after sunset) and over space for $x^2 + y^2 > R_f^2$ gives a proability that a bat will exceed $R_f$ of $p = 4.5 \times 10^{-6}$.

Integrating $\phi(x,y,t)$ over the range $R$ of a detector $i$ at $(x_i, y_i)$ gives
the expected density of bats in range of detector $i$ at time $t$. The
integral is approximated using a quadratic Taylor expansion around $(x_i, y_i)$:
%Need to add details of Taylor expansion and integral for thesis


\begin{equation}
  N_i(t) = \frac{1}{4Dt} e^{-\frac{x_i^2 + y_i^2}{4Dt}} \left[ R^2 + \frac{R^4}{4} \left( \frac{x_i^2 + y_i^2}{8D^2t^2} - \frac{1}{2Dt} \right)\right]
\end{equation}
%
where $R$ is the detector range. The total number of hits expected at each
detector for $0 < t < T$ is calculated by integrating over time,

\begin{equation}
  \tilde{N_i}(T) = \int_{t=0}^{t=T} N_i(t) dt.
\end{equation}
%
At present, this is approximated using a Riemann sum at intervals $\Delta t$:

\begin{equation}
  \tilde{N_i}(T) \simeq \sum_{t=0}^{t=T} N_i(t) \Delta t.
  \label{eqn:expect_hits}
\end{equation}
%
In future, this approximation will be improved by using a better numerical
integration method. The number of hits expected at each detector $\tilde{N_i}(T)$ for a possible roost location $\bm{\theta'}$ will be compared to the number of hits found in bat surveys to estimate the probability that the true roost is at $\bm{\theta'}$.

\section{Diffusion on a shrinking domain}

In \cite{woolley2011stochastic}

The PDE for diffusion on a 1D domain with apical growth $l(t)$ is
%
\begin{equation}
\D{\phi}{t} = \frac{D}{l(t)^2} \DD{\phi}{x} + x \frac{\dot{l(t)}}{l(t)} \D{\phi}{x} .
\end{equation}
%
For growth of the form $l = \mathrm{e}^{rt}$, the PDE becomes
%
\begin{equation}
\D{\phi}{t} = \frac{D}{\mathrm{e}^{2rt}} \DD{\phi}{x} + x r \D{\phi}{x} .
\label{eqn:exp_apical}
\end{equation}

Using
%
\begin{equation}
\D{}{x} (rx\phi) = r\phi + xr \D{\phi}{x},
\end{equation}
%
\eqn{eqn:exp_apical} can be written as
%
\begin{equation}
\D{\phi}{t} = \D{}{x}\left( \frac{D}{\mathrm{e}^{2rt}}\D{\phi}{x} + xr \phi \right) - ru .
\end{equation}
%
Integrating the convection term gives
\begin{equation}
\int_0^x x'r\D{\phi}{x'} dx = xr\phi - r\int_0^x \phi dx ,
\end{equation}
%
and therefore \eqn{eqn:exp_apical} can also be written as
%
\begin{equation}
\D{\phi}{t} = \D{}{x}\left( \frac{D}{\mathrm{e}^{2rt}}\D{\phi}{x} + xr \phi - r\int_0^x \phi dx \right) .
\end{equation}



\subsection{The ABC Algorithm}

First, the threshold parameter $\epsilon$ and sample size $n$ are fixed and the
prior distribution $p(\theta)$ is defined as uniform over the circle of 3km
radius around the original detected bat, $p(\theta) = \frac{1}{3^2 \pi}$.

Observations $\bm{Y}$ are given by the number of hits at each detector from bat
surveys and dataset $\bm{X}$ is the expected number of hits at each detector for
a roost at $\theta'$ given by Equation \ref{eqn:expect_hits}. The posterior
distribution $p(\theta \mid \bm{Y})$ for roost locations will be given by
$\bm{\theta}$, where $\bm{\theta_i}$ is the $i$-th accepted sample.

The pseudocode is as follows:

\begin{algorithmic}
  \While {$i < n$}
    \State  Sample $\theta'$ from $p(\theta)$
    \State Calculate $\bm{X}$ using Equation \ref{eqn:expect_hits}
    \State $\rho \gets \mid \bm{X} - \bm{Y} \mid$
    \If {$\bar{\rho} < \epsilon$}
      \State $\bm{\theta_i} \gets \theta'$
    	\State $i \gets i + 1$
    \EndIf
  \EndWhile
\end{algorithmic}

The probability distribution function of the roost search area is then given by $\bm{\theta}$, and the true location of the
roost can be estimated by taking the mean of the posterior $\bm{\theta}$.
