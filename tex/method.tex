4
\section{Radoiotracking data}

During the night, bats emerge from the roost and fly around to forage. Studying this movement is difficult as bats they are nocturnal, elusive and difficult to track. GPS tracking devices are unsuitable, as they are large enough that they are likely to impede bat movement. Instead, radio transmitters are fitted to the bats. These are smaller and less likely to affect foraging behaviour. The signal from the transmitters is picked up using scanning radio-receivers, and the position of the bat is estimated. Field workers can then follow the bat and attempt to maintain contact, taking regular recordings of location until the signal is lost, or the bat returns to the roost. Due to the nature of the tracking, locations are not recorded at regular intervals, and instead when the signal is found.

A study was conducted at Greater Horseshoe bat roosts in Devon to identify roosts used in the area. 12 bats were fit with radio tags and studied over 24 nights. Due to a limited number of workers and limited battery life on the tags, bats were not tracked every night. Four day roosts were used by bats in the study, with some bats using different roosts on different days. The roost used by each bat was identified for 2/3 of bat-nights. For this analysis, only  the data from nights when a bat's roost was known was used, since the movement away from the roost is important. A total of 322 bat locations were used for this analysis, and these are shown along with the location of day roosts in \fig{fig:radiotrack_locations}. As the data is not recorded at regular time intervals, locations were linearly interpolated at intervals of $\Delta t = 200$ seconds. The interpolated trajectories for bat 1 are shown in \fig{fig:bat1} along with the two day roosts used by bat 1 over the survey. The trajectories show that bat 1 took a different path each night and always remained within 3km of the roost.
%
\begin{figure} [h]
    \centering
        \includegraphics[width=\textwidth]{track_locations.pdf}
        \caption{Locations of day roosts and bat recordings found during the radio tracking survey}
    \label{fig:radiotrack_locations}
\end{figure}
%
\begin{figure} [h]
    \centering
        \includegraphics[width=\textwidth]{bat1_locations.pdf}
        \caption{The locations recorded for bat 1 during the study. Bat 1 was only recorded on 6 nights. The locations were interpolated for each night to obtain trajectories.}
    \label{fig:bat1}
\end{figure}
%
To gain more information about the movement of bats, the mean-squared distance (MSD) was plotted over the course of the night, shown in \fig{fig:MSD}. The results show an initial straight line segment up to $t = 1.6$ hours from sunset, followed by a quadratic decrease for the rest of the night until a sharp decrease as the bats return to the roost.
%
\begin{figure} [h]
    \centering
        \includegraphics[width=\textwidth]{RadioTrack_simulation.pdf}
        \caption{CHANGE THIS TO MSD}
    \label{fig:radiotrack_fit}
\end{figure}
%
\section{A diffusion model for bat movement}
%
Diffusion models are widely used to model animal movement for a number of species \cite{Ovaskainen2016}. In this case, a 2D diffusion model is used to describe the movement of bats flying away from the roost.
%
If the roost is at $(x_0,y_0)$ and bats leave the roost at time $t =0$,
the 2D diffusion equation describes the probability density $\phi(x,y,t)$ of
finding a bat at position $(x,y)$ at a later time $t$,
%
\begin{equation}
  \D{\phi(x,y,t)}{t} = D \nabla^2 \phi(x,y,t) ,
  \label{eqn:diffusion_cartesian}
\end{equation}
%
where the diffusion coefficient $D$ quantifies the speed with which bats diffuse. The diffusion equation can also be written in polar coordinates:
%
\begin{equation}
\D{ \phi(r,t)}{t} = \frac{D}{r} \D{}{ r} \left( r \D{\phi(r,t)}{r} \right),
\label{eqn:diffusion_polar}
\end{equation}
%
where $r$ is the distance from the roost, given by $r=\sqrt{(x-x_0)^2 + (y-y_0)^2}$. As diffusion is symmetric, $\phi$ is only dependent on $r$ and not on the angle.
%
The initial condition
%
\begin{equation}
\phi(r,t=0) = \delta(r=0)
\label{eqn:IC}
\end{equation}
%
specifies that all bats begin the night at the roost. The boundary conditions
%
\begin{equation}
\phi(r=\infty,t) = 0
\label{eqn:IC}
\end{equation}
%
and
%
\begin{equation}
\D{\phi(r=\infty,t)}{t} = 0
\label{eqn:IC}
\end{equation}
%
{\huge Something about how it's flat at infinity??}
%
Motion is determined by the parameter $D$, the diffusion coefficient. The diffusion coefficient determines the speed of motion away from the origin and can be calculated using the mean squared distance (MSD) $\mathbb{E}[r^2]$. The expected MSD at time $t$ can be calculated using the diffusion equation,

\begin{equation}
\mathbb{E}[r^2] = \int_{\infty}r^2 \phi(r,t) d\Omega ,
\label{eqn:MSD_int}
\end{equation}
%
where the integral is over all space.
%
Differentiating \eqn{eqn:MSD_int} with respect to time and substituting $\D{\phi}{t}$ from \eqn{eqn:diffusion_polar},
%
\begin{equation}
\begin{split}
\frac{d}{dt} \left<r^2\right> &= \frac{d}{dt}\int_{\infty}r^2 \phi(r,t) d\Omega \\
                            &= \frac{d}{dt} \int_0^{2\pi}\int_0^{\infty} r^2 \phi(r,t) r dr d\theta \\
                           &= \int_0^{2\pi} \int_0^{\infty} r^3 \D{\phi}{t} dr d\theta \\
                            &= \int_0^{2\pi} \int_0^{\infty} r^3 \frac{D}{r} \D{}{r } \left( r \D{ \phi}{ r}\right) dr d\theta \\
                            &= \int_0^{2\pi} \left( \left[ D r^2 \left( r \D{ \phi}{ r}\right) \right]_0^{\infty} - \int_0^{\infty} 2rD \left(r \frac{\partial \phi}{\partial r} \right) dr \right) d\theta. \\
\label{eqn:diffusion_1}
\end{split}
\end{equation}
%
$\D{\phi}{r} = 0$ and $\phi(r=\infty)=0$
%
\begin{equation}
\begin{split}
\frac{d}{dt} \left<r^2\right> &= \int_0^{2\pi} \int_0^{\infty} -2r^2D \D{ \phi}{r}dr d\theta \\
                            &= \int_0^{2\pi} \left( \left[-2r^2D \phi \right]_0^{\infty} + \int_0^{\infty} 4rD \phi dr \right)d\theta \\
                            &= \int_0^{2\pi} \int_0^{\infty} 4rD\phi dr d\theta \\
                            &= 4D \int_{\infty} \phi d\Omega \\
\frac{d}{dt} \left<r^2\right>  &= 4D \\
\label{eqn:diffusion_1}
\end{split}
\end{equation}
Integrating with respect to time gives
%
\begin{equation}
\left<r^2\right> = 4Dt ,
\label{eqn:D_msd}
\end{equation}
%
and therefore MSD is directly proportional to time for a simple diffusion model.

The diffusion equation in cartesian coordinates, given by \eqn{eqn:diffusion_cartesian}, can be written as

{\huge What do i want from this}

 solved using a Fourier transform in the spatial dimensions. Writing the Fourier transform of $\phi_0(x,y,t)$ as $\mathscr{F}\left[ \phi_0(x,y,t) \right] = \tilde{\phi_0}(k_x,k_y,t)$, each term in equation

\begin{equation}
\begin{split}
  \mathscr{F}\left[ \DD{\phi_0(x,y,t)}{x} \right] &= \int \int \DD{\phi_0(x,y,t)}{x} \mathrm{e}^{-2\pi i(xk_x + yk_y)} dx dy \\
  &= -4\pi^2k_x^2 \tilde{\phi_0}(k_x,k_y,t), \\
  \mathscr{F}\left[ \DD{\phi_0(x,y,t)}{y} \right] &= \int \int \DD{\phi_0(x,y,t)}{y} \mathrm{e}^{-2\pi i(xk_x + yk_y)} dx dy \\
  &= -4\pi^2k_y^2 \tilde{\phi_0}(k_x,k_y,t) \\
  \mathscr{F}\left[\DD{\phi_0(x,y,t)}{t}\right] &= \DD{\tilde{\phi_0}(k_x,k_y,t)}{t} \\
  \mathscr{F}\left[\phi_0(x,y,0)\right] &= \mathscr{F}\left[\delta(0)\right] \\
  &= 1
\end{split}
\end{equation}


Substituting into \eqn{eqn:cartesian_diffusion}

\begin{equation}
  \DD{\tilde{\phi_0}(k_x,k_y,t)}{t} = -4\pi^2D (k_x^2 + k_y^2) \tilde{\phi_0}(k_x,k_y,t).
\end{equation}

Integrating with respect to time gives

\begin{equation}
\begin{split}
  \tilde{\phi_0}(k_x,k_y,t) &= \int -4\pi^2D (k_x^2 + k_y^2) \tilde{\phi_0}(k_x,k_y,t) dt \\
  &= \tilde{\phi_0}(k_x,k_y,0)\mathrm{e}^{-4\pi^2D(k_x^2 + k_y^2)t} \\
\end{split}
\end{equation}

IC $ \tilde{\phi_0}(k_x,k_y,0) = 1$ and so

\begin{equation}
  \tilde{\phi_0}(k_x,k_y,t) = \mathrm{e}^{-4\pi^2D(k_x^2 + k_y^2)t} \\
\end{equation}

Then inverse fourier transform to recover $\phi_0(x,y,t)$

\begin{equation}
  \phi_0(x,y,t) = \mathscr{F}^{-1}\left[\tilde{\phi_0}(k_x,k_y,t)\right] = \frac{1}{4\pi Dt}\mathrm{e}^{-\frac{x^2 + y^2}{4Dt}} \\
\end{equation}

Transforming back to polar coordinates gives

\begin{equation}
  \phi_0(x,y,t) = \frac{1}{4\pi Dt}\mathrm{e}^{-\frac{r^2}{4Dt}} .\\
  \label{eqn:phi_0}
\end{equation}


\section{Fitting model to data}

\begin{figure} [h]
    \centering
        \includegraphics[width=\textwidth]{RadioTrack_simulation.pdf}
        \caption{Mean squared distance from the roost for radio tracked bats in Devon, UK. The initial straight line segment corresponds to diffusion with $D = 65 ms^{-s^2}$. Diffusion/leapfrog simulation fit with $r^2 = 0.92$}
    \label{fig:radiotrack_fit}
\end{figure}

Stochastic differential equation:

Diffusion:
\begin{equation}
X_{t+1} = X_t + N(0,D)
\end{equation}

Diffusion with drift:






Area is topologically complex: mix of favourable habitat: patches of woodland, high hedgerows and steep cliffs. Not clear how generalisable the data is to other landscapes. Could try ABC with detector data and see?

Interacting particles:

\begin{equation}
dx_i(t) = (m_i(t) - \frac{\kappa x_i(t)}{||x_i(t)||}) dt + \sigma dB(t)
\end{equation}

\begin{equation}
m_i(t) = \sum_{j \neq i} \frac{(x_j(t) - x_i(t))}{||x_j(t) - x_i(t)||} * h(||x_j(t) - x_i(t)||).
\end{equation}

\begin{equation}
h(x) = 1/(\epsilon + x^2)
\end{equation}

Then talk about the drift thingy:
show that quadratic fit is ok
Drift diffusion equation
explain why that doesn't work with multiple bats at many distances: add example with a few bats and the avg of them

Leapfrogging: write out SDE?
