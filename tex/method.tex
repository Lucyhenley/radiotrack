\section{Radoiotracking data}

Shortly after sunset, bats emerge from the roost and fly around to forage. Studying this movement is difficult as bats they are nocturnal, elusive and difficult to track. GPS tracking devices are unsuitable, as they are large enough that they are likely to impede bat movement. Instead, radio transmitters are often used as these are smaller and less likely to affect foraging behaviour. Data from radio tracking surveys are widely used to identify habitat use of bats \cite{Bontadina2002, Encarnacao2005}.  The signal from transmitters is picked up using scanning radio-receivers, and the position of the bat is estimated. Field workers can then follow the bat and attempt to maintain contact, taking regular recordings of location until the signal is lost, or the bat returns to the roost. Due to the nature of the tracking, locations are not recorded at regular intervals, and instead when the signal is found.

A study was conducted at Greater Horseshoe bat roosts in Devon to identify roosts used in the area. 12 bats were fit with radio tags and studied over 24 nights. Due to a limited number of workers and limited battery life on the tags, bats were not tracked every night. Four day roosts were used by bats in the study, with some bats using different roosts on different days. The roost used by each bat was identified for 2/3 of bat-nights. For this analysis, only  the data from nights when a bat's roost was known was used, since the dispersal from the roost is important. A total of 322 bat locations were used for this analysis, and these are shown along with the location of day roosts in \fig{fig:radiotrack_locations}. Since the data is recorded at irregular time intervals locations were linearly interpolated between recordings at intervals of $\Delta t = 200$ seconds. The interpolated trajectories for bat 1 are shown in \fig{fig:bat1} along with the two day roosts used by bat 1 during the survey. These trajectories indicate that the bat travelled in different directions each night whilst foraging and remained within 3km of the roost at all times {\huge Check the max distance from the roost?}
%
\begin{figure} [h]
    \centering
        \includegraphics[width=\textwidth]{track_locations.pdf}
        \caption{Locations of day roosts and tracked bat locations recorded during the radio tracking survey.}
    \label{fig:radiotrack_locations}
\end{figure}
%
\begin{figure} [h]
    \centering
        \includegraphics[width=\textwidth]{bat1_locations.pdf}
        \caption{The locations recorded for bat 1 over 6 nights during the study. The locations were interpolated for each night to obtain trajectories.}
    \label{fig:bat1}
\end{figure}
%
The mean-squared distance (MSD) from the roost was calculated using the interpolated positions and is shown in \fig{fig:MSD}. The results show an initial straight line segment up to $t = 1.6$ hours from sunset, followed by a quadratic decrease for the rest of the night until a sharp decrease as the bats return to the roost before sunrise.
%
\begin{figure} [h]
    \centering
        \includegraphics[width=\textwidth]{RadioTrack_MSD.pdf}
        \caption{The mean-sqared distance (MSD) for all radio tracked bats. The standard error is shown as a ribbon.}
    \label{fig:MSD}
\end{figure}
%
\section{A diffusion model for bat movement}
%
Partial differential equations (PDEs) are often used in ecology to describe spatial processes such as animal movement \cite{Holmes1994}. The diffusion model is commonly used to model dispersal \cite{Ovaskainen2016}. In this case, a 2D diffusion model is used to describe the dispersal of bats flying away from the roost.
%
All bats are assumed to leave the roost at position $(x_0,y_0)$ at time $t=0$. The 2D diffusion equation describes the probability density $\phi(x,y,t)$ of
bats at position $(x,y)$ at a later time $t$,
%
\begin{equation}
  \D{\phi(x,y,t)}{t} = D \nabla^2 \phi(x,y,t) ,
  \label{eqn:diffusion_cartesian}
\end{equation}
%
where the diffusion coefficient $D$ quantifies the dispersal speed. Since bats are assumed to move The diffusion equation can also be written in polar coordinates:
%
\begin{equation}
\D{ \phi(r,t)}{t} = \frac{D}{r} \D{}{ r} \left( r \D{\phi(r,t)}{r} \right),
\label{eqn:diffusion_polar}
\end{equation}
%
where $r$ is the distance from the roost, given by $r=\sqrt{(x-x_0)^2 + (y-y_0)^2}$. As diffusion is symmetric, $\phi$ is only dependent on $r$ and not on the angle.
%
The initial condition
%
\begin{equation}
\phi(r,t=0) = \delta(r=0)
\label{eqn:IC}
\end{equation}
%
specifies that all bats begin the night at the roost. The boundary conditions
%
\begin{equation}
\phi(r=\infty,t) = 0
\label{eqn:IC}
\end{equation}
%
and
%
\begin{equation}
\D{\phi(r=\infty,t)}{t} = 0
\label{eqn:IC}
\end{equation}
%
{\huge Something about how it's flat at infinity??}
%
Motion is determined by the parameter $D$, the diffusion coefficient. The diffusion coefficient determines the speed of motion away from the origin and can be calculated using the mean squared distance (MSD) $\mathbb{E}[r^2]$. The expected MSD at time $t$ can be calculated using the diffusion equation,

\begin{equation}
\mathbb{E}[r^2] = \int_{\infty}r^2 \phi(r,t) d\Omega ,
\label{eqn:MSD_int}
\end{equation}
%
where the integral is over all space.
%
Differentiating \eqn{eqn:MSD_int} with respect to time and substituting $\D{\phi}{t}$ from \eqn{eqn:diffusion_polar},
%
\begin{equation}
\begin{split}
\frac{d}{dt} \left<r^2\right> &= \frac{d}{dt}\int_{\infty}r^2 \phi(r,t) d\Omega \\
                            &= \frac{d}{dt} \int_0^{2\pi}\int_0^{\infty} r^2 \phi(r,t) r dr d\theta \\
                           &= \int_0^{2\pi} \int_0^{\infty} r^3 \D{\phi}{t} dr d\theta \\
                            &= \int_0^{2\pi} \int_0^{\infty} r^3 \frac{D}{r} \D{}{r } \left( r \D{ \phi}{ r}\right) dr d\theta \\
                            &= \int_0^{2\pi} \left( \left[ D r^2 \left( r \D{ \phi}{ r}\right) \right]_0^{\infty} - \int_0^{\infty} 2rD \left(r \frac{\partial \phi}{\partial r} \right) dr \right) d\theta. \\
\label{eqn:diffusion_1}
\end{split}
\end{equation}
%
$\D{\phi}{r} = 0$ and $\phi(r=\infty)=0$
%
\begin{equation}
\begin{split}
\frac{d}{dt} \left<r^2\right> &= \int_0^{2\pi} \int_0^{\infty} -2r^2D \D{ \phi}{r}dr d\theta \\
                            &= \int_0^{2\pi} \left( \left[-2r^2D \phi \right]_0^{\infty} + \int_0^{\infty} 4rD \phi dr \right)d\theta \\
                            &= \int_0^{2\pi} \int_0^{\infty} 4rD\phi dr d\theta \\
                            &= 4D \int_{\infty} \phi d\Omega \\
\frac{d}{dt} \left<r^2\right>  &= 4D \\
\label{eqn:diffusion_1}
\end{split}
\end{equation}
Integrating with respect to time gives
%
\begin{equation}
\left<r^2\right> = 4Dt ,
\label{eqn:D_msd}
\end{equation}
%
and therefore MSD is directly proportional to time for a simple diffusion model.



\section{Fitting model to data}



\begin{figure} [h]
    \centering
        \includegraphics[width=\textwidth]{RadioTrack_simulation.pdf}
        \caption{Mean squared distance from the roost for radio tracked bats in Devon, UK. The initial straight line segment corresponds to diffusion with $D = 65 ms^{-s^2}$. Diffusion/leapfrog simulation fit with $r^2 = 0.92$}
    \label{fig:radiotrack_fit}
\end{figure}

Diffusion is a stochastic process and can therefore be described in terms of a Markov process. Let the location of bat $i$ at time $t$ be $\bm{X_{i,t}}$. The initial condition
%
\begin{equation}
X_{i,0} = (0,0)         \forall i
\end{equation}
%
specifies that all bats begin the night at the roost. The diffusion model can then be described by the stochastic difference equation,
%
\begin{equation}
X_{t+1} = X_t + N(0,D) .
\end{equation}



\section{Movement back to the roost}

Convection-diffusion doesn't work: wrong shape because particles close to the origin return faster than those far away and avg is wrong.
Plot of multiple bats SD and MSD?

Leapfrogging: write out SDE or write as algorithm/pseudocode?
