\section{Modelling methods}
%
Deterministic partial differential equation models are used to model the movement of bats foraging. As discussed in \sect{sect:radiotrack}, the movement is in two distinct phases, a dispersal followed by a return to the roost. In the next section various models for each phase will be compared.

\subsection{Phase one: dispersal} \label{phase1}

Diffusion models are widely used to model animal movement, specifically dispersal, for a number of
species \cite{Ovaskainen2016}. In this case, a diffusion model is used to
describe dispersal during phase one of movement as bats fly away from the roost. Since it is commonly accepted that bats tend
to forage within the Core Sustenance Zone and remain within a certain distance
 of the roost, a diffusion model on a bounded domain is considered here. As discussed in \chap{chap:intro}, Greater Horseshoe bats tend to move along linear features whilst foraging, and as such we will consider a one dimensional diffusion model in which bats are confined to moving along lines, as well as in two dimensions such that bats are able to travel anywhere within the domain.

 \subsubsection{A diffusion model in one dimension}

First we will first consider diffusion on a one  dimensional domain, $\Omega \subset \mathbb{R}$, where the spatial coordinate is bounded, $x \in [0,R]$.  The probability density $\phi(x,t)$ of finding a bat at position $x$ at time $t$ is given by
 %
 \begin{equation}
   \D{\phi(x,t)}{t} = D \DD{\phi(x,t)}{x} ,
   \label{eqn:diffusion_cartesian}
 \end{equation}
 %
 where the diffusion coefficient $D$ quantifies the rate of spread. The boundary conditions,
 %
\begin{align}
\D{\phi(x=0,t)}{x} &= 0, \\
\D{\phi(x=R,t)}{x} &= 0,
\label{eqn:BC1d}
\end{align}
%
specify zero-flux across the boundary. The initial condition,
%
\begin{equation}
\phi(x = 0) = \delta(0),
\label{eqn:IC1d}
\end{equation}
%
specifies that all bats begin the night at the roost before moving away to begin foraging.
%
 \subsubsection{A Discretised Diffusion Model} \label{1ddiscrete}

 The diffusion equation on a bounded domain can be solved using a discretised ODE description \cite{woolley2011stochastic}. The domain can be discretised into $N$ boxes, each of length
 $h\Delta x=R/N$. The probability density in each box $i$ at $x=x_i$ is denoted by $\phi_i$, and evolves over time according to the diffusion process. A diagram of the motion is shown in \fig{fig:diffusion_diagram1d}. A finite difference approximation is used to describe the movement of probability density between boxes. The central difference approximation to the second order derivative at box $i$ is given by
 %
 \begin{equation}
  \at{\DD{\phi}{x}}{x=x_i}= \frac{\phi_{i-1}-2\phi_i+\phi_{i+1}}{h^2}, \label{eqn:diffusion_discrete1di}
 \end{equation}
 %
 where $\phi_{i-1} = \phi_i - h$ and $\phi_{i+1} = \phi_i + h$.

 \begin{figure} [t]
     \centering
         \includegraphics[width=0.5\textwidth]{1d_diffusion_diag.pdf}
         \caption{A diagram to illustrate the movement of probability density between boxes in the discretised diffusion model. Diffusion between boxes is represented by $d$ and the probability density in each box $i$ is denoted by $\phi_i$.}
     \label{fig:diffusion_diagram1d}
 \end{figure}

 The equation for box 1 at $x = 0$ is
 \begin{equation}
 \frac{d\phi_1}{dt} = \frac{D}{h^2}(\phi_{0}-2\phi_1 +\phi_{2}) + \frac{D}{2h^2} (\phi_{2}-\phi_{1}).
 \end{equation}
 %
 However, due to the reflective boundary condition in \eqn{eqn:BC1d}, any bats in a trajectory that would pass through the boundary are reflected back in the direction of the roost. Bats that would pass from box 1 to an imaginary box 0 are instead reflected back, and therefore the value of $\phi$ in the imaginary box 0 is the same as in box 1, $\phi_0$ = $\phi_1$, and
 %
 \begin{equation}
 \frac{d\phi_1}{dt} = \frac{D}{h^2}(\phi_{2}- \phi_1) + \frac{D}{2h^2} (\phi_{2}-\phi_{1}) .
         \label{eqn:box_1}
 \end{equation}
%
 Similarly, from \eqn{eqn:diffusion_discrete1di}, for strip $n$ at $r=R$,
 %
 \begin{equation}
 \frac{d\phi_n}{dt} = \frac{D}{h^2}(\phi_{n-1}-2\phi_n +\phi_{n+1}) + \frac{D}{2h^2} (\phi_{n+1}-\phi_{n}) .
 \end{equation}
 %
 Due to the reflective boundary condition between strip $n$ and strip $n+1$, $\phi_{n+1} = \phi_n$, and
 \begin{equation}
 \frac{d\phi_n}{dt} = \frac{D}{h^2}(\phi_{n-1}-\phi_n) + \frac{D}{2h^2} (\phi_{n+1}-\phi_{n-1}) .
         \label{eqn:strip_n1d}
 \end{equation}

 Gathering up \eqnto{eqn:diffusion_discrete1di}{eqn:strip_n1d} and substituting $d = \frac{D}{h^2}$, the set of equations describing the full system is
 %
 \begin{equation}
 \frac{d\phi_i}{dt} = \begin{cases}
 		d(\phi_i - \phi_{i+1}), & \text{for } i = 1, \\
 		d(\phi_{i-1}-2\phi_i +\phi_{i+1}), & \text{for } 2 \leq i \leq N-1, \\
 		d(\phi_{i-1}-\phi_i), & \text{for } i = N ,
 		\end{cases}
         \label{eqn:discrete_diffusion}
 \end{equation}
 %
 where the discretised diffusion coefficient is given by
 $d = D/h^2$.
  The initial condition corresponding to \eqn{eqn:IC1d} means that probability density is concentrated in the first box,
 %
 \begin{equation}
 \phi_i(0) = \begin{cases}
 		\frac{1}{h}, & \text{for } i = 1, \\
 		0, & \text{for } 2 \leq i \leq N. \\
 		\end{cases}
         \label{eqn:discrete_diffusion_IC}
 \end{equation}
 %

 %
 A diagram illustrating the spread of probability density due to the diffusion process is shown in \fig{fig:diffusion_diagram1d}. This generates a system of $N$ ODEs describing motion over
 the domain at each timestep and which can be solved using a numerical ODE solver. The equations for $i=1$ and $i=N$ correspond to reflective, zero-flux boundary conditions. The discretised model was simulated using DifferentialEquations.jl, a package for solving differential equations in Julia \cite{DifferentialEquations}. The result of a simulation with $N = 100$ boxes in a domain of length $R = 1000$m and diffusion coefficient $D = 100\mathrm{ms^{-2}}$ is shown in \fig{fig:discretised_phi}. The diffusion process spreads the probability density out from the left side of the domain, where $\phi$ is high, and tends to homogenise the probability density across the domain over time. After 5 hours, the probability density $\phi$ is evenly spread throughout the domain and the probability distribution is eventually uniform.

  \begin{figure} [t]
      \centering
          \includegraphics[width=0.8\textwidth]{discretised_phi.png}
          \caption{The value of $\phi$ for a discretised diffusion simulation with parameters $N = 100$, $R = 2000\mathrm{m}$ and $D = 100\mathrm{ms^{-2}}$ and initial condition given by \eqn{eqn:discrete_diffusion_IC} after $t = $ 1 hour, 2 hours and 5 hours.}
      \label{fig:discretised_phi}
  \end{figure}

\subsubsection{Diffusion in two dimensions}

Next we will consider a diffusion model in two dimensions to
describe dispersal during phase one of movement. The third dimension is not
included as height is not measured in the radio tracking survey, and it is not
needed to describe landscape use.

If the roost is at $(x_0,y_0)$ and bats leave the roost at time $t =0$,
the 2D diffusion equation describes the probability density $\phi(x,y,t)$ of
finding a bat at position $(x,y)$ at time $t$,
%
\begin{equation}
  \D{\phi(x,y,t)}{t} = D \nabla^2 \phi(x,y,t) ,
  \label{eqn:diffusion_cartesian2d}
\end{equation}
%
where $D$ is the diffusion coefficient, a positive constant that quantifies the
 rate of spread. The Core Sustenance Zone is denoted by $\Omega \subset \mathbb{R}^2$ and modelled as a disk of radius $R$ centred around the roost. Therefore we will consider the diffusion equation in polar coordinates,
 %
 \begin{equation}
 \D{ \phi(r,t)}{t} = \frac{D}{r} \D{}{ r} \left( r \D{\phi(r,t)}{r} \right),
 \label{eqn:diffusion_polar2d}
 \end{equation}
 %
 where $r$ is the distance from the roost, given by $r=\sqrt{(x-x_0)^2 +
 (y-y_0)^2}$. Since the domain is symmetric, $\phi$ is only dependent on $r$ and not
  on the angle. The initial condition,
%
 \begin{equation}
 \phi(r = 0) = \delta(0),
 \label{eqn:IC2d}
 \end{equation}
%
specifies that all bats begin the night at the roost before moving away at time $t$ to begin foraging. The boundary condition,
%
\begin{equation}
\D{\phi(r=R,t)}{r} = 0,
\label{eqn:BC}
\end{equation}
%
specifies zero-flux across the boundary.

\subsubsection{Mean squared displacement in the two dimensional diffusion model} \label{msd2d}
%
The relationship between the expected mean squared displacement (MSD) and time $t$
can be calculated using moments of the probability density $\phi$,
%
\begin{equation}
\left<r^2\right> = \int_{\Omega}r^2 \phi(r,t) d\omega ,
\label{eqn:MSD_expectation}
\end{equation}
%
where $\omega = (r,\theta) \subset \Omega$. Taking the time derivative of both
sides and substituting
\begin{equation}
    \D{\phi(r,t)}{t}
\end{equation} from \eqn{eqn:diffusion_polar2d},
%
\begin{align}
\frac{d}{dt} \left<r^2\right> &= \frac{d}{dt}\int_{\Omega}r^2 \phi(r,t) d\omega ,\\
                           &= \int_0^{2\pi} \int_0^{R} r^3 \D{\phi}{t} dr d\theta ,\\
                            &= \int_0^{2\pi} \int_0^{R} r^3 \frac{D}{r} \D{}{r } \left( r \D{ \phi}{ r}\right) dr d\theta , \\
                            &= \int_0^{2\pi} \left( \left[ D r^2 \left( r \D{ \phi}{ r}\right) \right]_0^{R} - \int_0^{R} 2rD \left(r \frac{\partial \phi}{\partial r} \right) dr \right) d\theta , \\
                            &= \int_0^{2\pi} \int_0^{R} -2r^2D \D{ \phi}{r}dr d\theta , \\
                            &= \int_0^{2\pi} \left( \left[-2r^2D \phi \right]_0^{R} + \int_0^{R} 4rD \phi dr \right)d\theta , \\
                            &= - 4\pi R^2D \phi(R,t) + 4D \int_{\Omega} \phi d\omega ,
\label{eqn:diffusion_1}
\end{align}
%
and therefore,
\begin{equation}
\frac{d}{dt} \left<r^2\right>  = 4D( 1- \pi R^2 \phi(R,t)) .
\end{equation}
%
Integrating with respect to time gives
%
\begin{equation}
\left<r^2\right> = 4D \left( t - \pi R^2 \int_0^t \phi(R,\tau) d \tau \right).
\label{eqn:diffusion_msd}
\end{equation}
%
Over short timescales, $\phi(R,t) \approx 0$, since the probability of a reaching the boundary over a short period of time is small due to the initial condition. Therefore, over a short timescale, the expected MSD for diffusion is directly proportional to time,
%
\begin{equation}
\left<r^2\right> \approx 4Dt.
\label{eqn:diffusion_short}
\end{equation}
%
An expression for mean-squared displacement over a long timescale can also be derived using \eqn{eqn:MSD_expectation}. Over long timescales, we expect the probability density to be uniformly spread across the domain,
%
\begin{equation}
\phi(r,t) = \frac{1}{\pi R^2} .
\end{equation}
%
Substituting this into \eqn{eqn:MSD_expectation} gives
%
\begin{align}
\left<r^2\right> &= \frac{1}{\phi R^2}\int_{0}^{2\pi} \int_{0}^{R} r^2 dr d\theta \\
&= \frac{1}{2} R^2,
\end{align}
%
and therefore the mean squared displacement is constant over long timescales.

\subsection{A discretised ODE solution to the two dimensional diffusion model}

The two dimensional diffusion equation in a bounded domain can be solved using the same method as in \sect{1ddiscrete}. For a circular domain in polar coordinates $\Omega = [0,R] \times [0, 2\pi]$, the domain can be discretised into $N$ annuli, each of
width $h=R/N$. Due to the assumption that the domain is symmetric, we will not consider angular movement around the annuli. The probability density in each annulus $i$ is denoted by $\phi_i$, and evolves over time according to the diffusion process.  The distance from the origin to the inner edge of annulus $i$ is given by $r_i$. A diagram of the motion is shown in \fig{fig:2d_diffusion_diag}. A central difference approximation to \eqn{eqn:diffusion_polar2d} is used to describe the movement of probability density between boxes.

\begin{figure}
\centering
    \includegraphics[width=0.5\textwidth]{figs/2d_diffusion_diag.pdf}
    \caption{A diagram of the discretised diffusion process. Each box represents an annular region around the roost centre. The probability density shifts between boxes due to the diffusion process, denoted by $d$. Reflective boundary conditions ensure that probability density never leaves the domain.}
\label{fig:2d_diffusion_diag}
\end{figure}

The central difference approximation to the second order derivative at annulus $i$ is given by
%
\begin{equation}
 \at{\DD{\phi}{r}}{r=r_i}= \frac{\phi(r_i-h)-2\phi(r_i)+\phi(r_i+h)}{h^2}. .\label{eqn:diffusion_discretei}
\end{equation}
%
The central difference approximation to \eqn{eqn:diffusion_polar2d} is then
%
\begin{equation}
 \at{\D{\phi}{t}}{r=r_i} = D \left(\frac{\phi(r-h)-2\phi(r)+\phi(r+h)}{h^2} + \frac{1}{r} \frac{\phi(r+h)-\phi(r-h)}{2 h} \right).
 \label{eqn:diffusion_polar_cd}
\end{equation}
%
The distance $r_i$ at strip $i$ is given by
\begin{equation}
r = i h ,
\end{equation}
%
and \eqn{eqn:diffusion_polar_cd} can therefore be written as
%
\begin{equation}
\D{\phi(r)}{t} = \frac{D}{h^2} \left(\phi(r-h)-2\phi(r)+\phi(r+h)\right) + \frac{D}{2ih^2}\left(\phi(r+h)-\phi(r-h) \right).
\label{eqn:diffusion_polar_final}
\end{equation}
%
Changing notation for the discretised version gives
%
\begin{equation}
\frac{d\phi_i}{dt} = \frac{D}{h^2}(\phi_{i-1}-2\phi_i +\phi_{i+1}) + \frac{D}{2ih^2} (\phi_{i+1}-\phi_{i}).
        \label{eqn:discrete_diffusion_i}
\end{equation}
%
The equation for strip 1 at $r=0$ is
%
\begin{equation}
\frac{d\phi_1}{dt} = \frac{D}{h^2}(\phi_{0}-2\phi_1 +\phi_{2}) + \frac{D}{2h^2} (\phi_{2}-\phi_{1}).
\end{equation}
%
However, due to the reflective boundary condition in \eqn{eqn:BC}, any bats in a trajectory that would pass through the boundary are reflected back in the direction of the roost. Bats that would pass from strip 1 to an imaginary strip 0 are instead reflected back, and therefore the value of $\phi$ in the imaginary strip 0 is the same as in strip 1, and $\phi_0$ = $\phi_1$.
%
\begin{equation}
\frac{d\phi_1}{dt} = \frac{D}{h^2}(\phi_{2}- \phi_1) + \frac{D}{2h^2} (\phi_{2}-\phi_{1}) .
        \label{eqn:strip_1}
\end{equation}

Similarly, from \eqn{eqn:diffusion_discrete1}, for strip $n$ at $r=R$,
%
\begin{equation}
\frac{d\phi_n}{dt} = \frac{D}{h^2}(\phi_{n-1}-2\phi_n +\phi_{n+1}) + \frac{D}{2h^2} (\phi_{n+1}-\phi_{n}) .
\end{equation}
%
Due to the reflective boundary condition between strip $n$ and strip $n+1$, $\phi_{n+1} = \phi_n$, and
\begin{equation}
\frac{d\phi_n}{dt} = \frac{D}{h^2}(\phi_{n-1}-\phi_n) + \frac{D}{2h^2} (\phi_{n+1}-\phi_{n-1}) .
        \label{eqn:strip_n}
\end{equation}

Gathering up \eqnto{eqn:discrete_diffusion_i}{eqn:strip_n}, the set of equations describing the full system is
%
  \begin{equation}
  \frac{d\phi_i}{dt} = \begin{cases}
  		\frac{D}{h^2}(\phi_{i+1} - \phi_{i}) + \frac{D}{2h^2} (\phi_{i+1}-\phi_i), & \text{for } i = 1, \\
  		\frac{D}{h^2}(\phi_{i-1}-2\phi_i +\phi_{i+1}) + \frac{D}{2ih^2} (\phi_{i+1}-\phi_{i}), & \text{for } 2 \leq i \leq N-1, \\
  		\frac{D}{h^2}(\phi_{i-1}-\phi_i) - \frac{D}{2ih^2} (\phi_{i}-\phi_{i-1}), & \text{for } i = N .
  		\end{cases}
          \label{eqn:discrete_diffusion2d}
  \end{equation}

 A 2D discretised diffusion simulation with $N=1000$ particles and diffusion coefficient $D=65$ms\textsuperscript{-2} was run using DifferentialEquations.jl \cite{DifferentialEquations} to validate the model. The mean squared displacement is shown in \fig{fig:2d_diffusion}. The mean squared displacement for the discretised simulation is initially linear and equal to $4Dt$ before the curve flattens and tends towards a constant value. This is consistent with the mean squared displacement expected for a bounded diffusion model, calculated in \ref{msd2d}.

\begin{figure}
\centering
    \includegraphics[width=0.8\textwidth]{figs/2d_diffusion.png}
    \caption{The mean squared displacement for a discretised diffusion simulation with $N=1000$ particles in a bounded 2D domain for $D=65$ms\textsuperscript{-2} compared to the analytical result of $4Dt$ from \sect{msd2d}.}
\label{fig:2d_diffusion}
\end{figure}
%
\subsection{Comparison of one and two dimensional diffusion models}
%
\begin{figure}
\centering
    \includegraphics[width=0.8\textwidth]{figs/2d_diffusion.png}
    \caption{The mean squared displacement for a discretised diffusion simulation with $N=1000$ particles in a bounded 2D domain for $D=65$ms\textsuperscript{-2} compared to the analytical result of $4Dt$ from \sect{msd2d}.}
\label{fig:1d2d_diffusion}
\end{figure}
%
