
\subsection{Phase two: return to roost}

The diffusion model described in \sect{phase1} explains the initial dispersal in phase 1, however it cannot explain the decrease in MSD for phase 2. For the second phase of movement, two different models are considered, a convection-diffusion diffusion model and a diffusion model on a shrinking domain. Convection-diffusion models are commonly used to model animal movement, however we will show here that a shrinking domain model provides a more accurate description of bat movement whilst foraging.

\subsubsection{A convection-diffusion model in two dimensions}

 First, we will consider a convection-diffusion model to describe the drift of bats back towards the roost. The 2D symmetric convection-diffusion equation, in polar coordinates, is
%
\begin{equation}
  \D{\phi(r,t)}{t} = \frac{D}{r} \D{}{r}\left(r \D{\phi(r,t)}{r} \right) - \chi \D{\phi(r,t)}{r},
  \label{eqn:convection}
\end{equation}
%
where $D$ is the diffusion coefficient and $\chi$ is the convection coefficient. In this case, bats are returning to a point, the roost location $(x_0,y_0)$, where $r=0$, and therefore the convection component of \eqn{eqn:convection} describes a drift towards $r=0$. As bats undergo diffusive movement whilst dispersing from the roost for a time $T$ before their behaviour changes, $\chi$ is time dependent,
%
\begin{equation}
\chi (t) =  \begin{cases}
    0, & \text{for } t < T, \\
    \chi_0, & \text{for } t \leq T,
  \end{cases}
  \label{eqn:time_dependentchi}
\end{equation}
%
where $\chi_0$ is a positive constant. When $t < T$, the convection term in \eqn{eqn:convection} is zero, and the equation reduces to a polar diffusion equation, as in \eqn{eqn:diffusion_polar2d}. For $t > T$, \eqn{eqn:convection} becomes
%
\begin{equation}
  \D{\phi(r,t)}{t} = \frac{D}{r} \D{}{r}\left(r \D{\phi(r,t)}{r} \right) - \chi_0 \D{\phi(r,t)}{r}.
\end{equation}

\subsubsection{A discretised convection-diffusion model}

 \begin{figure} [ht]
     \centering
         \includegraphics[width=0.5\textwidth]{diffusion_convection.pdf}
         \caption{A diagram to illustrate the movement of probability density between annuli in the discretised convection-diffusion model. Diffusion between annuli is represented by $d$ and the drift due to convection is represented by $c$. The probability density in each annulus $i$ is denoted by $\phi_i$.}
     \label{fig:convection_diffusion_diag}
 \end{figure}

The convection-diffusion model will be solved using a discretised ODE model, as with the diffusion model in \sect{phase1}. The domain $\Omega$ of length $R$ is discretised into $N$ annuli, each of length $h = R/N$. The probability density in each annulus $i$ is denoted by $\phi_i$ and evolves over time according to the diffusion and convection processes, as illustrated in \fig{fig:convection_diffusion_diag}. The diffusion component is unchanged from \sect{phase1}, however this time there is an additional convection component pushing bats away from the domain boundary and back towards the roost at $r=0$. The convection process shifts probability density towards the left, towards annulus $i = 1$, and the discretised equations are
%
\begin{equation}
\frac{d\phi_i}{dt} = \begin{cases}
		d(\phi_{i+1} - \phi_i) + \frac{d}{2} (\phi_{i+1}-\phi_i) - c \phi_{i+1}, & \text{for } i = 1, \\
		d(\phi_{i-1}-2\phi_i +\phi_{i+1}) + \frac{d}{2i} (\phi_{i+1}-\phi_{i-1}) - \frac{c}{i}(\phi_{i+1}-\phi_{i}), & \text{for } 2 \leq i \leq N-1, \\
		d(\phi_{i-1}-\phi_i)  + \frac{d}{2i} (\phi_{i}-\phi_{i-1}) - \frac{c}{i}\phi_{i}, & \text{for } i = N.
		\end{cases}
        \label{eqn:discrete_convection}
\end{equation}
%
The discretised diffusion coefficient is $d = D/h^2$ as before and $c=\chi/h$ is the discretised convection coefficient. The initial condition is the state of the system after diffusion for time $T$.

The model was simulated using \texttt{DifferentialEquations.jl} \cite{DifferentialEquations}, using a time dependent convection coefficient as in \eqn{eqn:time_dependentchi} with $T = 4000$ seconds. The
simulation was run with a domain of length $R = 2000$m, split into $N = 100$
annuli. The diffusion coefficient was $D = 65\mathrm{m^2s^{-1}}$, and the convection coefficient
was $\chi =  - 15\mathrm{ms^{-1}}$. The results
of this simulation are shown in \fig{fig:convection2d}. The probability density $\phi$ is shown in \fig{fig:convection2dphi}. The convection process pushes $\phi$ uniformly in the direction of the drift, towards the left side of the domain, whereas the diffusion tends to spread $\phi$ across the domain. The boundary acts as a barrier stopping $\phi$ from moving any further and $\phi$ collects at the boundary. The movement towards the edge slows as $t$ increases because it is unable to move any further, and eventually the system reaches a steady state when diffusion and convection are balanced.

The MSD for the same convection-diffusion simulation is shown in \fig{fig:convection2dmsd}. The convection-diffusion simulation shows the initial rapid dispersal expected from the diffusion model, however the shape of the curve for phase 2 is clearly inconsistent with the radio tracking data. The convection-diffusion simulation yields a convex curve, and the decrease in MSD slows with time as bats return to the roost and stop contributing to the movement. The MSD never reaches 0 because the diffusion term acts to spread bats out whilst the convection term is pushing them back towards the roost, and these eventually balance, without the colony returning to the roost.

\begin{figure}
     \centering
     \begin{subfigure}[b]{0.48\textwidth}
         \centering
         \includegraphics[width=\textwidth]{convection_diffusion_phi2dchi_15.png}
         \caption{The probability density $\phi$ after $t = 1$, 1.5, 2 and 5 hours.}
         \label{fig:convection2dphi}
     \end{subfigure}
     \hfill
     \begin{subfigure}[b]{0.48\textwidth}
         \centering
         \includegraphics[width=\textwidth]{convection_diffusion_MSD2dchi_15.png}
         \caption{The mean squared displacement for $0 \leq t \leq 8$ hours. }
         \label{fig:convection2dmsd}
     \end{subfigure}
     \caption{The results of a convection-diffusion simulation with parameters are $R = 2000$m, $N = 100$,
     $D = 100\mathrm{m^2s^{-1}}$, $\chi_0 = - 15\mathrm{ms^{-1}}$ and $T = 4000$ seconds.}
     \label{fig:convection2d}
     \end{figure}

We have also tested a convection-diffusion model in which the convection coefficient is spatially dependent as well as time dependent such that bats drift back to the roost at a rate dependent on their distance from the roost,
%
\begin{equation}
\chi (r,t) =  \begin{cases}
    0, & \text{for } t < T, \\
    r^{\beta}\chi_0, & \text{for } t \leq T,
  \end{cases}
  \label{eqn:r_dependentchi}
\end{equation}
%
where $\beta$ is a constant. The results of simulations with $ -2 \leq \beta \leq 2$ are shown in \fig{fig:rdependentphi}. The plots show that for each value of the exponent $\beta$, the curves are convex rather than concave and diffusion eventually balances convection and bats stop moving once they reach the roost.
%
\begin{figure} [h]
    \centering
        \includegraphics[width=0.7\textwidth]{rdependentphi.png}
        \caption{The mean squared displacement for spatially dependent convection-diffusion models with convection coefficient $\chi$ of the form given in \eqn{eqn:r_dependentchi}. The simulation parameters are $R = 2000$m, $N = 100$,
        $D = 100\mathrm{m^2s^{-1}}$ and $T = 4000$ seconds.}
    \label{fig:rdependentphi}
\end{figure}
%

 In order to produce a concave curve consistent with the radio tracking data, the model cannot push all bats towards the roost at the same time, as those closest to the roost will always reach the roost first and stop moving. Next we will consider a model that solves this problem by selecting only the bats furthest from the roost to drift back towards the roost

\subsubsection{Diffusion on a shrinking domain} \label{shrink}
%NOTE Leapfrogging?
 Next we will consider diffusion on a shrinking domain, assuming that bats disperse at the beginning of the night, but tend to move back towards the roost, narrowing the area in which they forage as the night goes on. To model diffusion on a shrinking domain, we can consider a frame of reference that moves with a flow caused by the domain shrinking \cite{crampin1999reaction}. Considering an elemental volume $w(t)$, the velocity field $\bm{a}$ of the flow at position $\bm{X}$ is
%
\begin{equation}
\bm{a}(\bm{X},t) = \frac{d\bm{X}}{dt}.
\end{equation}
%
\begin{figure} [h]
    \centering
        \includegraphics[width=0.5\textwidth]{shrink_diagram.pdf}
        \caption{A diagram showing the mapping of the stationary to the shrinking domain frame of reference. }
    \label{fig:shrink_diagram}
\end{figure}
%
Moving from the stationary to the shrinking domain frame of reference requires a Lagrangian description of the domain which maps each point in the domain from the stationary frame to the shrinking frame. If $\bm{X} = (r(t=0),\theta(t=0))$ is the initial location of an element $w(t)$ and $\bm{x} = (r(t),\theta(t))$ is the location of the element at time $t$ then the mapping is defined by the function $\Gamma(\bm{X},t) $ as $\bm{x}(t) =\Gamma(\bm{X},t) $. A diagram showing the mapping is displayed in \fig{fig:shrink_diagram}. For a growth rate $l(t)$, the mapping function is
%
\begin{equation}
\Gamma(X,t) = Xl(t).
\label{eqn:mapping}
\end{equation}
%
Then, the velocity field is defined by
%
\begin{equation}
\bm{a}(\bm{X},t) = \frac{d\bm{\Gamma}(\bm{X},t)}{dt} .
\label{dgammadt}
\end{equation}
%
Using the chain rule to expand \eqn{dgammadt} gives
%
\begin{equation}
    \Dd{\Gamma_i}{t}{X_k} = \sum_{j=1}^{3}\D{a_i}{x_j}\D{\Gamma_j}{X_k},
\end{equation}
%
where $\Gamma_i$ and $a_i$ are the $i^{th}$ components of $\bm{\Gamma}$ and $\bm{a}$ \cite{crampinnonuniform}.
Considering the mapping function in \eqn{eqn:mapping}, a stationary element is mapped onto the shrinking domain with $\bm{x}(t) = \bm{X}l(t)$. The diffusion equation in polar coordinates, \eqn{eqn:diffusion_polar2d}, can then be transformed from the shrinking domain variables by mapping the derivatives to the new domain using
%
\begin{equation}
\at{\D{}{t}}{x} = \at{\D{}{t}}{X} - X \frac{\dot{l}(t)}{l(t)} \D{}{X}
\end{equation}
%
and
%
\begin{equation}
\D{}{x} = \frac{1}{l(t)}\D{}{X}.
\end{equation}
%
The scaled diffusion equation in the stationary frame of reference is then given by
%
\begin{equation}
\D{\phi}{t} = \frac{D}{Xl(t)^2}\D{}{X}\left(X \D{\phi}{X} \right) + X \frac{\dot{l(t)}}{l(t)}\D{\phi}{X}.
\end{equation}
%
\begin{figure} [h]
    \centering
        \includegraphics[width=0.5\textwidth]{apical.pdf}
        \caption{A diagram showing a domain shrinking apically, such that shrinking is restricted to the edge of the domain.}
    \label{fig:apical_diagram}
\end{figure}
%
We will consider a domain shrinking apically, in which shrinking is restricted to a region of width $\delta$ at the tip of domain. The shrinking rate $\rho$ is zero everywhere except at the edge of the domain,
%
\begin{equation}
\D{a}{r} = \rho = \begin{cases}
		0, & 0 \leq r \leq R(t) - \delta, \\
		S_{tip}(t), & R(t) - \delta \leq r \leq R(t), \\
		\end{cases}
        \label{eqn:s_tip}
\end{equation}
%
where $S_{tip}(t)$ is the shrinking rate in the element at the edge of the boundary. A diagram illustrating \eqn{eqn:s_tip} is shown in \fig{fig:apical_diagram}. Therefore, the size of the domain $R(t)$ is given by $R(t) = 1 + \delta \int_0^t S_{tip}(t')dt'$. Considering the shrinking region to be much smaller than the domain size, the system can be reduced to an Eulerian moving boundary problem.
Assuming also that the growth rate $l$ is small, the equation reduces to the diffusion equation
%
\begin{equation}
\D{\phi}{t} = \frac{D}{Xl(t)^2}\D{}{X}\left(X \D{\phi}{X} \right).
\end{equation}
%
If the diffusion rate is larger than the rate at which
 the domain changes size, the solution should approximate steady state diffusion. As shown in \sect{phase1}, the solution to the diffusion equation on a bounded domain tends to uniformity. On a circular domain, as $t\rightarrow \infty$,
 %
  \begin{equation}
  \phi \rightarrow \frac{1}{\pi R^2}.
  \label{eqn:uniform_circle}
  \end{equation}
  %
Thus, we expect the probability distribution to remain
 approximately uniform over the domain and the expected MSD at time $t$ can be calculated using this probability distribution,
 %
 \begin{align}
 \left<r^2\right> 	&= \int_{\Omega}r^2 \phi(r,t) d\Omega , \nonumber\\
                 	&= \int_0^{2\pi}\int_0^{R(t)} \frac{r^3}{\pi R(t)^2} dr d\theta, \nonumber \\
	                &= \frac{R(t)^2}{2} .
 \label{eqn:shrink_domain}
 \end{align}
 %
   \begin{figure} [h]
       \centering
           \includegraphics[width=0.5\textwidth]{shrinking_diagram.pdf}
           \caption{A diagram illustrating a discretised diffusion process on a shrinking domain. The diffusion process is denoted by $d$ and the probability density in annulus $i$ is denoted by $\phi_i$}
       \label{fig:shrinking_diagram}
   \end{figure}
%
\subsubsection{A discretised diffusion simulation on a shrinking domain}
%NOTE timescale?
The diffusion process on a shrinking domain can be simulated using the diffusion model described in \sect{phase1}. The simulation consists of two separate stages at each timestep, illustrated in \fig{fig:shrinking_diagram}. First, the diffusion process moves probability density between annuli. Then, the size of the domain is reduced by adding the concentration in annulus $N$ to annulus $N-1$ and removing annulus $N$.

Since the MSD for this phase is a negative parabola, the shrinking rate is chosen to give
%
 \begin{equation}
 \left<r^2\right> \propto a - t^2.
 \label{eqn:MSD_shrink}
 \end{equation}
%
From \eqn{eqn:shrink_domain}, a time dependent domain size $R(t)$ of
%
\begin{equation}
R(t) = \sqrt{2(R_0^2 - \alpha t^2)}
\label{eqn:Rt}
\end{equation}
%
gives an expected MSD of
 \begin{equation}
 \left<r^2\right> = {R_0}^2 -\alpha t^2.
 \end{equation}

The result of a simulation of diffusion on a domain shrinking with rate $R(t)$ given by \eqn{eqn:Rt} is shown in \fig{fig:shrink_phi}. The probability density spreads across the domain due to the diffusion process, however the shrinking causes the probability density to increase at the right edge of the domain. In this case, the rate of diffusion is slower than the rate at which the domain shrinks, and therefore the probability density does not spread evenly across the domain.
%NOTE Add another plot to show both extremes: slow shrink rate, high diffusion. high shrink rate, slow diffusion.
    \begin{figure} [h]
        \centering
            \includegraphics[width=0.8\textwidth]{shrinkphi.png}
            \caption{The probability density $\phi$ for diffusion simulation on a shrinking domain at $t = 1$, 4 and 6 hours. The initial condition is a delta function at $r = 0$ and the
             parameters for this simulation are $N = 100$, $D = 100\mathrm{m^2s^{-1}}$, $R_0 = 1800$m, $t_s = 1000$ seconds and $\alpha  = 3.24 \mathrm{m^2s^{-2}}$.}
        \label{fig:shrink_phi}
    \end{figure}
