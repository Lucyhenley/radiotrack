\section{A diffusion model for bat movement} \label{diffusion}

Diffusion models are widely used to model animal movement for a number of
species \cite{Ovaskainen2016}. In this case, a 2D diffusion model is used to
describe dispersal as bats fly away from the roost. The third dimension is not
included as height is not measured in the radio tracking survey, and it is not
needed to describe landscape use.

If the roost is at $(x_0,y_0)$ and bats leave the roost at time $t =0$,
the 2D diffusion equation describes the probability density $\phi(x,y,t)$ of
finding a bat at position $(x,y)$ at time $t$,
%
\begin{equation}
  \D{\phi(x,y,t)}{t} = D \nabla^2 \phi(x,y,t) ,
  \label{eqn:diffusion_cartesian}
\end{equation}
%
where $D$ is the diffusion coefficient, a positive constant that quantifies the rate of spread.
%
The diffusion equation can also be written in polar coordinates,
%
\begin{equation}
\D{ \phi(r,t)}{t} = \frac{D}{r} \D{}{ r} \left( r \D{\phi(r,t)}{r} \right),
\label{eqn:diffusion_polar}
\end{equation}
%
where $r$ is the distance from the roost, given by $r=\sqrt{(x-x_0)^2 +
(y-y_0)^2}$. As diffusion is symmetric, $\phi$ is only dependent on $r$ and not
 on the angle.

The initial condition,
%
\begin{equation}
\phi(x_0,y_0,0) = \delta(x_0,y_0),
\label{eqn:IC}
\end{equation}
%
specifies that all bats begin the night at the roost. For diffusion on an
infinite domain, the boundary conditions

{\huge no idea also make these equations better i think not sure}
%
%\begin{equation}
%    \lim_{\r\to\infty} \phi(r) = 0
%\end{equation}
%
and
%
\begin{equation}
\D{\phi(r=\infty,t)}{t} = 0
\end{equation}
%
specify that the probability density decays as $r\rightarrow\infty$.
The relationship between the expected mean squared displacement (MSD) and time $t$
can be calculated using the probability density $\phi$,
%
{\huge define a domain :(, maybe write this as bounded diffusion first}
\begin{equation}
\left<r^2\right> = \int_{r=0}^{r=\infty}r^2 \phi(r,t) d\Omega ,
\end{equation}
%
where the integral is over all space. Taking the time derivative of both
sides and substituting
\begin{equation}
    \D{\phi(r,t)}{ t}
\end{equation} from \eqn{eqn:diffusion_polar},
%
\begin{equation}
\begin{split}
\frac{d}{dt} \left<r^2\right> &= \frac{d}{dt}\int_{\infty}r^2 \phi(r,t) d\Omega ,\\
                           &= \int_0^{2\pi} \int_0^{\infty} r^3 \D{\phi}{t} dr d\theta ,\\
                            &= \int_0^{2\pi} \int_0^{\infty} r^3 \frac{D}{r} \D{}{r } \left( r \D{ \phi}{ r}\right) dr d\theta , \\
                            &= \int_0^{2\pi} \left( \left[ D r^2 \left( r \D{ \phi}{ r}\right) \right]_0^{\infty} - \int_0^{\infty} 2rD \left(r \frac{\partial \phi}{\partial r} \right) dr \right) d\theta , \\
                            &= \int_0^{2\pi} \int_0^{\infty} -2r^2D \D{ \phi}{r}dr d\theta , \\
                            &= \int_0^{2\pi} \left( \left[-2r^2D \phi \right]_0^{\infty} + \int_0^{\infty} 4rD \phi dr \right)d\theta , \\
                            &= \int_0^{2\pi} \int_0^{\infty} 4rD\phi dr d\theta , \\
                            &= 4D \int_{\infty} \phi d\Omega , \\
\frac{d}{dt} \left<r^2\right>  &= 4D .\\
\label{eqn:diffusion_1}
\end{split}
\end{equation}
%
Integrating with respect to time gives
%
\begin{equation}
\left<r^2\right> = 4Dt ,
\label{eqn:diffusion_msd}
\end{equation}
and therefore the expected MSD for diffusion is directly proportional to time.


\begin{equation}
  \D{\phi_0(x,y,t)}{t} = D \left( \DD{\phi_0(x,y,t)}{x}+\DD{\phi_0(x,y,t)}{y}\right),
  \label{eqn:cartesian_diffusion}
\end{equation}

and using a Fourier transform in the spatial dimensions. Writing the Fourier transform of $\phi_0(x,y,t)$ as $\mathscr{F}\left[ \phi_0(x,y,t) \right] = \tilde{\phi_0}(k_x,k_y,t)$, each term in equation

\begin{equation}
\begin{split}
  \mathscr{F}\left[ \DD{\phi_0(x,y,t)}{x} \right] &= \int \int \DD{\phi_0(x,y,t)}{x} \mathrm{e}^{-2\pi i(xk_x + yk_y)} dx dy \\
  &= -4\pi^2k_x^2 \tilde{\phi_0}(k_x,k_y,t), \\
  \mathscr{F}\left[ \DD{\phi_0(x,y,t)}{y} \right] &= \int \int \DD{\phi_0(x,y,t)}{y} \mathrm{e}^{-2\pi i(xk_x + yk_y)} dx dy \\
  &= -4\pi^2k_y^2 \tilde{\phi_0}(k_x,k_y,t) \\
  \mathscr{F}\left[\DD{\phi_0(x,y,t)}{t}\right] &= \DD{\tilde{\phi_0}(k_x,k_y,t)}{t} \\
  \mathscr{F}\left[\phi_0(x,y,0)\right] &= \mathscr{F}\left[\delta(0)\right] \\
  &= 1
\end{split}
\end{equation}


Substituting into \eqn{eqn:cartesian_diffusion}

\begin{equation}
  \DD{\tilde{\phi_0}(k_x,k_y,t)}{t} = -4\pi^2D (k_x^2 + k_y^2) \tilde{\phi_0}(k_x,k_y,t).
\end{equation}

Integrating with respect to time gives

\begin{equation}
\begin{split}
  \tilde{\phi_0}(k_x,k_y,t) &= \int -4\pi^2D (k_x^2 + k_y^2) \tilde{\phi_0}(k_x,k_y,t) dt \\
  &= \tilde{\phi_0}(k_x,k_y,0)\mathrm{e}^{-4\pi^2D(k_x^2 + k_y^2)t} \\
\end{split}
\end{equation}

IC $ \tilde{\phi_0}(k_x,k_y,0) = 1$ and so

\begin{equation}
  \tilde{\phi_0}(k_x,k_y,t) = \mathrm{e}^{-4\pi^2D(k_x^2 + k_y^2)t} \\
\end{equation}

Then inverse fourier transform to recover $\phi_0(x,y,t)$

\begin{equation}
  \phi_0(x,y,t) = \mathscr{F}^{-1}\left[\tilde{\phi_0}(k_x,k_y,t)\right] = \frac{1}{4\pi Dt}\mathrm{e}^{-\frac{x^2 + y^2}{4Dt}} \\
\end{equation}

Transforming back to polar coordinates gives

\begin{equation}
  \phi_0(x,y,t) = \frac{1}{4\pi Dt}\mathrm{e}^{-\frac{r^2}{4Dt}} \\
  \label{eqn:phi_0}
\end{equation}


\subsection{A discretised diffusion model} \label{discretised_model}


You need to add Crampin's work in before this to demonstrate how we get to this approximation. Firstly, you should note that the solution to the non-growing case tends to uniformity. This can be done solving the equation in Fourier terms and demonstrating that as t\rightarrow\infty then \phi \rightarrow 1/(\pi R^2_0).

Then, moving into a growing domain frame of reference we assume that the growth is really slow. Eliminating the growth term as small in the PDE we see that the uniform distribution satisfies it.

The diffusion model in a domain can be solved using a discretised ODE
description of the diffusion equation \cite{woolley2011stochastic}. For a domain
$\Omega$ of length $R_0$, the domain is discretised into $N$ boxes $u_i$ each of
length $\Delta x=R_0/N$. The diffusion process is described by the finite
difference approximations
%
\begin{equation}
\frac{du_i}{dt} = \begin{cases}
		d(u_i - u_{i+1}), & \text{for } i = 1, \\
		d(u_{i-1}-2u_i +u_{i+1}), & \text{for } 2 \leq i \leq N-1, \\
		d(u_{i-1}-u_i), & \text{for } i = N .
		\end{cases}
\end{equation}
%
where $d = D/(\Delta x)^2$ is the discretised diffusion coefficient.
%
This generates a system of $N$ ODEs for the domain at each timestep which can be
 solved using a numerical ODE solver. The equations for $i=1$ and $i=N$
 correspond to reflective, zero flux boundary conditions.

 \subsection{Diffusion on a growing or shrinking domain} \label{shrink}

Bats tend to remain close to the roost during foraging, within a Core Sustenance Zone \cite{Mathews2009}. After the initial dispersal, bats are assumed to be uniformly distributed
 around a circular domain of radius $R_0$. The probability distribution within this domain is
 uniform,
 %
 \begin{equation}
 \phi(r,t) = \frac{1}{\pi R_0^2}.
 \end{equation}
 %
 For a domain that is shrinking or growing, the radius of the domain at time $t$
 is given by $R(t)$. If the rate of diffusion is larger than the rate at which
 the domain changes size, the probability distribution should remain
 approximately uniform over the domain,
%
\begin{equation}
\phi(r,t) = \frac{1}{\pi R(t)^2}.
\end{equation}
%
 Then the expected MSD at time $t$ can be calculated using this probability distribution,
 %
 \begin{equation}
 \begin{split}
 \left<r^2\right> 	&= \int_{\Omega}r^2 \phi(r,t) d\Omega ,\\
                 	&= \int_0^{2\pi}\int_0^{\infty} \frac{r^3}{\pi R(t)^2} dr d\theta, \\
 \left<r^2\right>	&= \frac{R(t)^2}{2} .\\
 \label{eqn:shrink_domain}
 \end{split}
 \end{equation}

 To simulate diffusion on a shrinking domain, the diffusion model described in
 \sect{discretised_model} was used. To simulate the domain shrinking, the
 concentration of bats in boxes at the right boundary is added to the
 concentration in the next box, and $N$ is decreased so that the boundary is moved
 to the left.

 \section{abc etc}


 In this case, the parameters $\theta$ are $R_0$ and $t_s$, and $\bm{Y}$ is the
 MSD at each point in time. The posterior
 distribution $ p(\theta \mid \bm{Y})$ will be a distribution describing the
 probability of each set of possible parameters $\theta$. As there is initially no information about parameters $\theta$, the prior distribution $p(\theta)$
 is assumed to be uniform over plausible values.

 In this case uniform priors were used since there was no prior knowledge of the parameter values.


 In this case, the observations $\bm{Y}$ are the values of the MSD at each time
 point from the radio tracking data and $\bm{X}$ is the expected MSD at each time
 point for parameters $\theta' = (R_0',t_s') $, calculated using the model for
 diffusion on a shrinking domain described in  \sect{shrink}. The
 posterior distribution $p(\theta \mid \bm{Y})$ for set of parameters will be
 given by $\bm{\theta}$, where $\bm{\theta_i}$ is the $i$-th accepted sample.

\section{Model validation using radio tracking data}


The diffusion coefficient for the initial, linear dispersal was calculated from
the gradient of the line using \eqn{eqn:diffusion_msd} as $D = 63 m^2s^{-1}$.
For the later decrease in MSD, the model described in  \sect{shrink} for diffusion
on a shrinking domain was used. The shrinking rate was chosen to give a
quadratically decreasing MSD,
%
\begin{equation}
R(t) = R_0 - \sqrt{\alpha t^2}.
\end{equation}
%
The expected MSD for this $R(t)$ can be calculated using \eqn{eqn:shrink_domain}
as
%
 \begin{equation}
 \left<r^2\right>	= {R_0}^2 - 2R_0\sqrt{\alpha t} -\alpha t^2.
 \label{eqn:MSD_shrink}
 \end{equation}

The parameters to fit are $\alpha$, the rate at which the domain shrinks,
$t_s$, the time at which the domain begins to shrink and $R_0$, the initial size
of the domain. Since bats return to the roost at sunrise, the MSD is 0 at the end of the night and $\alpha$ was chosen from $R_0$ to ensure the domain size shrinks to 0 by sunrise. The parameters were fit using Approximate Bayesian Computation (ABC).

\subsection{Results}

The ABC algorithm was run for a sample size of $n = 10000$, and $\epsilon$ was chosen such that the best 1\% of parameter values were added to the posterior. The MSD for estimates of the true parameters is shown in \fig{fig:fit}. The curve shows a good fit with the radio tracking data, within the standard error for the majority of the night. The coefficient of determination ($r^2$ value) was calculated to compare the deterministic model and the radio tracking data, giving $r^2 = 0.904$, suggesting that the model provides a good description of bat movement during foraging.

\begin{figure} [h]
    \centering
        \includegraphics[width=\textwidth]{shrinking_bound.png}
        \caption{The MSD for a deterministic diffusion model on a shrinking domain of size $R(t) = R_0 - \sqrt{\alpha t^2}$. }
    \label{fig:fit}
\end{figure}
