
\subsection{Solution to Diffusion Equation (FOURIER SOLUTION ONLY POSSIBLE ON INFINITE DOMAIN!!!)}


derive:


Equating the rate of change of concentration in the elemental volume with the flux in and out gives the conservation form of the system,
%
\begin{equation}
\frac{d}{dt} \int_w(t) \phi d\bm{r}  = \int_w(t) - \nabla \cdot \bm{J} d \bm{r},
\end{equation}
%
where $\bm{J}$ is the diffusive flux. Applying Reynolds' Transport Theorem \cite{acheson1990elementary} to the left hand side,
%
\begin{equation}
\frac{d}{dt} \int_w(t) \phi d\bm{r} = \int_w(t) \frac{d\phi}{dt} + \nabla \cdot (\bm{a}\phi) d \bm{r}.
\end{equation}
%
Fick's Law of Diffusion,
%
\begin{equation}
\bm{J} = - D\nabla \phi_i
\end{equation}


The diffusion equation in Cartesian coordinates,
\begin{equation}
  \D{\phi(x,y,t)}{t} = D \nabla^2 \phi(x,y,t) ,
  \label{eqn:diffusion_cartesian}
\end{equation}
 can be solved using a Fourier transform in the spatial dimensions. Writing the Fourier transforms of $x$ as $\mathscr{F}\left[x \right] = k_x$, $y$ as $\mathscr{F}\left[y \right] = k_y$, and $\phi(x,y,t)$ as $\mathscr{F}\left[ \phi(x,y,t) \right] = \tilde{\phi}(k_x,k_y,t)$, each term in \eqn{eqn:diffusion_cartesian} can be Fourier transformed as
 %
\begin{align}
  \mathscr{F}\left[ \DD{\phi(x,y,t)}{x} \right] &= \int \int \DD{\phi(x,y,t)}{x} \mathrm{e}^{-2\pi i(xk_x + yk_y)} dx dy \\
  &= -4\pi^2k_x^2 \tilde{\phi}(k_x,k_y,t), \\
  \mathscr{F}\left[ \DD{\phi(x,y,t)}{y} \right] &= \int \int \DD{\phi(x,y,t)}{y} \mathrm{e}^{-2\pi i(xk_x + yk_y)} dx dy \\
  &= -4\pi^2k_y^2 \tilde{\phi}(k_x,k_y,t), \\
  \mathscr{F}\left[\D{\phi(x,y,t)}{t}\right] &= \D{\tilde{\phi}(k_x,k_y,t)}{t},\\
  \mathscr{F}\left[\phi(x,y,0)\right] &= \mathscr{F}\left[\delta(0)\right]  \\
  &= 1 .
\end{align}
%
The Fourier transform of \eqn{eqn:diffusion_cartesian} is then
%
\begin{equation}
  \D{\tilde{\phi}(k_x,k_y,t)}{t} = -4\pi^2D (k_x^2 + k_y^2) \tilde{\phi}(k_x,k_y,t).
\end{equation}
%
Integrating with respect to time,
%
\begin{align}
  \tilde{\phi}(k_x,k_y,t) &= \int -4\pi^2D (k_x^2 + k_y^2) \tilde{\phi}(k_x,k_y,t) dt, \\
  &= \tilde{\phi}(k_x,k_y,0)\mathrm{e}^{-4\pi^2D(k_x^2 + k_y^2)t}.
  \label{eqn:fourier_diffusion}
\end{align}
%
Writing the Fourier transform of the initial condition in \eqn{eqn:IC},
%
\begin{equation}
\mathscr{F}\left[\phi_0(x,y,0)\right] = \tilde{\phi_0}(k_x,k_y,0) = 1,
\end{equation}
%
and substituting into \eqn{eqn:fourier_diffusion} gives
%
\begin{equation}
  \tilde{\phi}(k_x,k_y,t) = \mathrm{e}^{-4\pi^2D(k_x^2 + k_y^2)t}. \\
\end{equation}
%
Taking the inverse Fourier transform to recover $\phi(x,y,t)$,
%
\begin{equation}
  \phi(x,y,t) = \mathscr{F}^{-1}\left[\tilde{\phi}(k_x,k_y,t)\right] = \frac{1}{4\pi Dt}\mathrm{e}^{-\frac{x^2 + y^2}{4Dt}}, \\
\end{equation}
%
and transforming back to polar coordinates gives
\begin{equation}
  \phi(r,t) = \frac{1}{4\pi Dt}\mathrm{e}^{-\frac{r^2}{4Dt}} .\\
  \label{eqn:diffusion_solution}
\end{equation}
