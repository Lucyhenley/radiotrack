
\section{A diffusion model for bat movement}

Diffusion models are widely used to model animal movement for a number of species \cite{Ovaskainen2016}. In this case, a 2D diffusion model is used to describe the movement of bats flying away from the roost.

If the roost is at $(x_0,y_0)$ and bats leave the roost at time $t =0$,
the 2D diffusion equation describes the probability density $\phi(x,y,t)$ of
finding a bat at position $(x,y)$ at time $t$,

\begin{equation}
  \D{\phi(x,y,t)}{ t} = D \nabla^2 \phi(x,y,t) ,
  \nonumber
\end{equation}
%
where $D$ is the diffusion coefficient and quantifies the speed with which bats diffuse.

The diffusion equation can also be written in polar coordinates:

\begin{equation}
\D{ \phi(r,t)}{t} = \frac{D}{r} \D{}{ r} \left( r \D{\phi(r,t)}{r} \right),
\end{equation}
%
where $r$ is the distance from the roost, given by $r=\sqrt{(x-x_0)^2 + (y-y_0)^2}$. As diffusion is symmetric, $\phi$ is only dependent on $r$ and not on the angle.

The initial condition

\begin{equation}
\phi(x_0,y_0,0) = \delta(x_0,y_0)
\label{eqn:IC}
\end{equation}
%
specifies that all bats begin the night at the roost.

Motion is determined by the parameter $D$, the diffusion coefficient. The diffusion coefficient determines the speed of motion away from the origin and can be calculated using the mean squared distance (MSD) $\mathbb{E}[r^2]$. The expected MSD at time $t$ can be calculated using the diffusion equation:

\begin{equation}
\left<r^2\right> = \int_{\infty}r^2 \phi(r,t) d\Omega ,
\end{equation}
%
where the integral is over all space.

The rate of change of the MSD with time can be used to calculate the relationship between MSD and time:

\begin{equation}
\begin{split}
\frac{d}{dt} \left<r^2\right> &= \frac{d}{dt}\int_{\infty}r^2 \phi(r,t) d\Omega \\
                            &= \frac{d}{dt} \int_0^{2\pi}\int_0^{\infty} r^2 \phi(r,t) r dr d\theta \\
                           &= \int_0^{2\pi} \int_0^{\infty} r^3 \D{\phi}{l t} dr d\theta \\
                            &= \int_0^{2\pi} \int_0^{\infty} r^3 \frac{D}{r} \D{}{r } \left( r \D{ \phi}{ r}\right) dr d\theta \\
                            &= \int_0^{2\pi} \left( \left[ D r^2 \left( r \D{ \phi}{ r}\right) \right]_0^{\infty} - \int_0^{\infty} 2rD \left(r \frac{\partial \phi}{\partial r} \right) dr \right) d\theta \\
                            &= \int_0^{2\pi} \int_0^{\infty} -2r^2D \D{ \phi}{r}dr d\theta \\
                            &= \int_0^{2\pi} \left( \left[-2r^2D \phi \right]_0^{\infty} + \int_0^{\infty} 4rD \phi dr \right)d\theta \\
                            &= \int_0^{2\pi} \int_0^{\infty} 4rD\phi dr d\theta \\
                            &= 4D \int_{\infty} \phi d\Omega \\
\frac{d}{dt} \left<r^2\right>  &= 4D \\
\label{eqn:diffusion_1}
\end{split}
\end{equation}
%
Integrating with respect to time gives
%
\begin{equation}
\left<r^2\right> = 4Dt ,
\label{eqn:D_msd}
\end{equation}
%
and therefore MSD is directly proportional to time.
%
\subsection{Advection-Diffusion model}
%
Motion is not diffusive for the whole night, as bats must return back to the roost by the morning. An advection-diffusion model adds an extra term to the simple diffusion equation that specifies a drift towards a given location. In this case, the drift will be towards the roost. In polar coordinates, the advection-diffusion equation is
%
\begin{equation}
  \D{\phi(r,t)}{t} = \frac{D}{r} \D{}{r}\left(r \D{\phi(r,t)}{r} \right) - \chi \D{\phi(r,t)}{r},
  \nonumber
\end{equation}
%
where $\chi$ is the advection coefficient, the strength of the drift back towards the roost.
%
The MSD for the advection-diffusion model in terms of $D$ and $\chi$ can be calculated using the same method as for the simple diffusion model:
%
\begin{equation}
\begin{split}
\frac{d}{dt} \left<r^2\right> &= \frac{d}{dt}\int_{\infty}r^2 \phi(r,t) d\Omega \\
                              &= \frac{d}{dt} \int_0^{2\pi}\int_0^{\infty} r^2 \phi(r,t) r dr d\theta \\
                              &= \int_0^{2\pi} \int_0^{\infty} r^3 \D{\phi}{t} dr d\theta \\
                              &= \int_0^{2\pi} \int_0^{\infty} r^3 \frac{D}{r} \D{}{r}\left(r \D{\phi}{r}\right)- r^3\chi\D{\phi}{r} dr d\theta \\
\end{split}
\end{equation}
%
Substituting the first term from \eqn{eqn:diffusion_1}:
%
\begin{equation}
\begin{split}
                              &= 4D - \chi \int_0^{2\pi} \int_0^{\infty} r^3 \D{\phi}{r} dr d\theta \\
                              &= 4D - \chi \int_0^{2\pi} \left[ r^3 \phi \right]_0^{\infty} - \int_0^{\infty}  3 r^2 \phi dr d\theta \\
                              &= 4D + 3\chi \int_0^{2\pi} \int_0^{\infty}  r^2 \phi dr d\theta  \\
                              &= 4D + 3\chi \int_{\infty} r \phi d\Omega \\
                              &= 4D + 3\chi \left<r\right>
\nonumber
\end{split}
\end{equation}
%
The mean distance $ \left<r\right>$ can be calculated in the same way:
%
\begin{equation}
\begin{split}
\frac{d}{dt} \left<r\right> &= \frac{d}{dt}\int_{\infty}r \phi(r,t) d\Omega \\
                              &= \frac{d}{dt} \int_0^{2\pi}\int_0^{\infty} r \phi(r,t)r dr d\theta \\
                              &= \int_0^{2\pi} \int_0^{\infty} r^2 \D{\phi}{t} dr d\theta \\
                              &= \int_0^{2\pi} \int_0^{\infty} r^2 \left[ \frac{D}{r} \D{}{r}\left(r \D{\phi}{r}\right)-\chi\D{\phi}{r} \right] dr d\theta \\
                              &= \int_0^{2\pi} \int_0^{\infty} rD  \D{}{r}\left(r \D{\phi}{r}\right) -  \chi r^2 \D{\phi}{r} dr d\theta \\
\nonumber
\end{split}
\end{equation}
%
\begin{equation}
\begin{split}
\frac{d}{dt} \left<r\right> &= \sqrt{\frac{D}{t}} + \int_0^{2\pi} \int_0^{\infty} -  \chi r^2 \D{\phi}{r} dr d\theta \\
                            &= \sqrt{\frac{D}{t}} + \int_0^{2\pi} - \left[ \chi r^2 \phi \right]_0^{\infty} + \int_0^{\infty} 2 \chi r \phi dr d\theta \\
                            &= \sqrt{\frac{D}{t}} + \int_0^{2\pi} \int_0^{\infty} 2 \chi r \phi dr d\theta \\
                            &= \sqrt{\frac{D}{t}} + \int_{\infty} 2 \chi \phi d\omega \\
                            &= \sqrt{\frac{D}{t}} + 2\chi
\nonumber
\end{split}
\end{equation}
%
Integrating with respect to time gives
%
\begin{equation}
\left<r\right> = 2\chi t .
\nonumber
\end{equation}
%
Substituting into the equation for $\left<r^2\right>$ gives
%
\begin{equation}
\begin{split}
\frac{d}{dt} \left<r^2\right> &= 4D + 6\chi\sqrt{Dt}  + 6\chi^2t \\
\left<r^2\right>              &= 4Dt + 12\chi\sqrt{Dt^3} + 3\chi^2t^2 .
\label{eqn:AD_msd}
\end{split}
\end{equation}
%
\subsection{Fitting the model to radio tracking data}
%
The diffusion model was fit to radio tracking data from 11 Greater Horseshoe bats, surveyed over the course of 26 nights. The data was recorded by placing a radio collar on a bat, and attempting to follow it over a night. Following bats in order to take measurements is challenging as they move quickly and are able to fly over a variety of terrains which are inaccessible to humans (for example, across privately owned or fenced off land). Therefore, measurements are not taken at regular intervals, and instead when a signal is detected the coordinates and times are recorded. In order to generate trajectories, the data was interpolated at evenly spaced time intervals. An example of the trajectories generated is shown in \fig{fig:Diffusion_eg} for four bats. The trajectories show what appears to be a random walk, with bats starting the night at their roost. The squared displacement from the roost was calculated for each bat on each survey night at time intervals of 200 seconds, and the mean at each time was taken to give the MSD. The results of this are shown in \fig{fig:RadioTrackD_SD}. These results show an initial straight line segment for $0 < t < 4 \times 10^3 s$, consistent with a diffusion model. The later section shows that bats drift back towards the roost, and all return to the roost by the end of the night. To calculate the diffusion coefficient $D$, the initial straight line segment for $0 < t < 4 \times 10^3$ was fit to \eqn{eqn:D_msd}. The estimate calculated was $D = 64.5 \pm 30 m^2s^{-1}$.

\begin{figure} [h]
\centering
      \includegraphics[width=0.4\textwidth]{Diffusion_eg.pdf}
      \caption{Trajectories for four bats over one night each. These were created by linearly interpolating between points where bats were recorded during radio tracking.}
      \label{fig:Diffusion_eg}
\end{figure}

\begin{figure} [h]
\centering
      \includegraphics[width=0.6\textwidth]{RadioTrack_SD.pdf}
      \caption{Estimates for MSD interpolated over evenly spaced time points. The filled area represents $\pm 1$ standard error for each estimate.}
      \label{fig:RadioTrackD_SD}
\end{figure}

\begin{figure} [h]
\centering
      \includegraphics[width=0.6\textwidth]{RadioTrack_fit.pdf}
      \caption{Estimates for MSD interpolated over evenly spaced time points. The filled area represents $\pm 1$ standard error for each estimate.
      A diffusion model with $D = 64.5 m^2s^{-1}$ was fit for $0 < t < 4 \times 10^3 s$, and an advection-diffusion model with $\chi = 0.00077 ms^{-1}$ and $D = 2.21 m^2s^{-1}$ was fit for $t > 4 \times 10^3 s$.}
      \label{fig:RadioTrackFit}
\end{figure}


\subsection{Using the diffusion model with static detector data}

For the parameters found using radio tracking data, motion for the first first hour after sunset approximates simple diffusion, with no bounded domain. As a result, the model used in locating roosts will use simple diffusion, and only use data from the first hour after sunset. The model will use the same initial condition as before, with all bats beginning the night at the roost. If the roost is at the origin and bats leave the roost at time $t =0$, the 2D diffusion equation describes the probability density $\phi(x,y,t)$ of finding a bat at position $(x,y)$ at time $t$,

\begin{equation}
  \frac{\partial \phi(x,y,t)}{\partial t} = D \nabla^2 \phi(x,y,t) .
  \nonumber
\end{equation}
%
The value for $D$ found using radio tracking surveys will be used here, $D = 75.5$ m\textsuperscript{2}s\textsuperscript{-1}. Solving for $\phi(x,y,t)$ gives

\begin{equation}
  \phi(x,y,t) = \frac{1}{4 \pi D t} e^{-\frac{x^2 + y^2}{4Dt}}. \cite{Ovaskainen2016}
  \label{eqn:phi}
\end{equation}


Integrating $\phi(x,y,t)$ over the range $R$ of a detector $i$ at $(x_i, y_i)$ gives
the expected density of bats in range of detector $i$ at time $t$. The
integral is approximated using a quadratic Taylor expansion around $(x_i, y_i)$. A 2D Taylor expansion of $f(x,y)$ about point $(x_i, y_i)$ is given by

\begin{equation}
\begin{split}
f(x,y) &= f(x_i,y_i) + f_x(x_i,y_i)\Delta x + f_y(x_i,y_i)\Delta y \\
       &+ \frac{1}{2}\left[ f_{xx}(x_i,y_i) \Delta x^2 + 2 f_{xy}(x_i,y_i) \Delta x \Delta y + f_{yy}(x_i,y_i) \Delta y^2 \right] + O(\Delta x^3 + \Delta y^3) .
\end{split}
\label{eqn:taylor}
\end{equation}
%
where $\Delta x = x-x_i$ and $\Delta y = y - y_i$. For $\phi$ in \eqn{eqn:phi}, the derivatives are

\begin{equation}
\begin{split}
\phi_x(x_i,y_i,t) &= \frac{-x_i}{2Dt} \phi(x_i,y_i,t) \\
\phi_y(x_i,y_i,t) &= \frac{-y_i}{2Dt} \phi(x_i,y_i,t) \\
\phi_{xx}(x_i,y_i,t) &= \phi(x_i,y_i,t) \left[ \frac{-1}{2Dt} + \frac{x_i^2}{4D^2t^2}\right] \\
\phi_{yy}(x_i,y_i,t) &= \phi(x_i,y_i,t) \left[ \frac{-1}{2Dt} + \frac{y_i^2}{4D^2t^2}\right] \\
\phi_{xy}(x_i,y_i,t) &= \frac{x_iy_i}{4D^2t^2}\phi(x_i,y_i,t)
\end{split}
\nonumber
\end{equation}

Substituting derivatives into \eqn{eqn:taylor} gives

\begin{multline}
\phi(x,y,t) = \phi(x_i,y_i,t) \left[1 - \frac{x_0}{2Dt}\Delta x - \frac{y_i}{2Dt} \Delta y + \left( \frac{-1}{4Dt} + \frac{x_i^2}{8D^2t^2} \right)\Delta x^2  \right. \\
\left. + \left( \frac{-1}{4Dt} + \frac{y_i^2}{8D^2t^2} \right) \Delta y^2 + \frac{x_iy_i}{4D^2t^2}\Delta x \Delta y \right] + O(\Delta x^3 + \Delta y ^3)
\end{multline}

Writing $\phi(x_i,y_i,t)$ as $\phi_i(t)$ and changing variables to $\Delta x = r\cos{\theta}$ and $\Delta y = r\sin{\theta}$:

\begin{multline}
\phi(r,\theta,t) \approx \phi_i(t) \left[ 1 - \frac{x_i}{2Dt} r\cos{\theta} - \frac{y_i}{2Dt} r\sin{\theta} +  r^2 \cos^2{\theta} \left(\frac{-1}{4Dt} + \frac{x_iy_i^2}{8D^2t^2} \right) \right. \\
\left. + r^2\sin^2{\theta}\left(\frac{-1}{4Dt} + \frac{y_i^2}{8D^2t^2} \right) + \frac{x_i y_i}{4D^2t^2} r^2\sin{\theta}\cos{\theta} \right]
\label{eqn:phi_polar}
\end{multline}

The expected density $N_i(t)$ of bats in the range of detector $i$ at point $(x_i,y_i)$ at time $t$ is found by integrating over the circle of radius $R$ around the detector:

\begin{multline}
N_i(t) = \int_0^R \int_0^{2\pi} \phi(r,\theta,t) r d\theta dr \\
= \phi_i(t) \int_0^R \left[ 2\pi + \frac{y_i}{2Dt}r + \pi r^2 \left(\frac{-1}{4Dt} + \frac{x_i^2}{8D^2t^2}  \right) + \pi r^2 \left( \frac{-1}{4Dt} + \frac{y_i^2}{8D^2t^2} \right) \right. \\
\left. - \frac{r^2}{2} \frac{x_0y_0}{4Dt}  - r \frac{y_i}{2Dt}  + \frac{r^2}{2} \frac{x_iy_i}{4D^2t^2} \right] r dr \\
= \phi_i(t) \int_0^R 2\pi r + \pi r^3 \left( \frac{-1}{2Dt} + \frac{x_i^2+y_i^2}{8D^2t^2} \right)  dr \\
= \phi_i(t) \left[ \pi r^2 + \frac{r^4\pi}{4}\left( \frac{-1}{2Dt} + \frac{x_i^2 + y_i^2}{8D^2t^2} \right) \right]_{r=0}^{r=R} \\
= \phi_i(t) \left[ \pi R^2 + \frac{\pi R^4}{4} \left( \frac{x_i^2 + y_i^2}{8D^2t^2} - \frac{1}{2Dt} \right) \right] \\ .
\nonumber
\end{multline}
%
Substituting in $\phi_i(t)$ from \eqn{eqn:phi_polar} gives

\begin{equation}
  N_i(t) = \frac{1}{4Dt} e^{-\frac{x_i^2 + y_i^2}{4Dt}} \left[ R^2 + \frac{R^4}{4} \left( \frac{x_i^2 + y_i^2}{8D^2t^2} - \frac{1}{2Dt} \right)\right] .
\end{equation}

The total number of hits expected at each
detector for $0 < t < T$ is calculated by integrating over time,

\begin{equation}
  \tilde{N_i}(T) = \int_{t=0}^{t=T} N_i(t) dt.
\end{equation}
%
At present, this is approximated using a Riemann sum at intervals $\Delta t$:

\begin{equation}
  \tilde{N_i}(T) \simeq \sum_{t=0}^{t=T} N_i(t) \Delta t.
  \label{eqn:expect_hits}
\end{equation}
%
In future, this approximation will be improved by using a better numerical
integration method. The number of hits expected at each detector $\tilde{N_i}(T)$ for a possible roost location $\bm{\theta'}$ will be compared to the number of hits found in bat surveys to estimate the probability that the true roost is at $\bm{\theta'}$.

%This model assumes infinite space and there are no boundary conditions, so that bats can travel any distance away from the roost. However, it can be shown that the probability of a bat travelling further than the maximum foraging radius $R_f =$ 3000m calculated using $\phi$ is typically small. Integrating $\phi$ using a numerical integration algorithm \cite{Berntsen1991} over the first hour  after sunset and over space for $x^2 + y^2 > R_f^2$ gives a proability that a bat will exceed $R_f$ of $p = 4.5 \times 10^{-6}$ and therefore the effects of the boundary are negligible in this case.
