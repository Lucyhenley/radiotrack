
\section{Radio Tracking Survey} \label{sect:radiotrack}

A radio tracking study was conducted at 3 Greater Horseshoe bat roosts in Devon to
study the usage of land surrounding the roosts \cite{Mathews2009}. 12 bats were fitted with
radio tags and studied over 24 nights. Due to a limited number of workers and
 limited battery life on the tags, each bat was not tracked every night. Four day
 roosts were used by bats in the study, with some bats using different roosts on
 different days. The roost used by each bat was identified for 2/3 of nights.
 For this analysis, only the
data from nights when a bat's roost was known was used, since the dispersal
from the roost is important.
A total of 322 bat locations were used for this analysis. The trajectory for bat
one over the 6 nights it was tracked is shown in \fig{fig:bat1}. The bat visits different areas whilst foraging, and takes a different route on each night. Each trajectory resembles a random walk. The number of
locations recorded varies each night because the signal was lost during the night
 on some nights. The recorded locations of all bats over the study period are displayed in a polar histogram in \fig{fig:polar_histogram}. The plot shows that most records are located close to the roost, and are distributed over all angles.

\begin{figure} [h]
    \centering
        \includegraphics[width=0.8\textwidth]{bat1_locations.png}
        \caption{The locations of bat 1 over 6 nights of the survey.}
    \label{fig:bat1}
\end{figure}

\begin{figure}
\centering
    \includegraphics[width=0.8\textwidth]{figs/polar_histogram.pdf}
    \caption{Polar histogram of bat locations over the study period. The density for each segment is calculated as the number of locations recorded in the segment divided by the area of the segment.}
\label{fig:polar_histogram}
\end{figure}


Since the data is recorded at irregular time intervals, locations were linearly
interpolated between recordings at intervals of $\Delta t = 200$ seconds.
The mean-squared distance (MSD) from the roost was calculated using the
interpolated positions and is shown in \fig{fig:MSD}. The standard error is given by $\sigma_{\bar{x}} = \sigma/\sqrt{n}$, where
$\sigma$ is the standard deviation and $n$ is the number of observations, and is
shown as a ribbon. The data indicates two movement phases, an initial dispersal from the roosts, followed by a gradual return whilst bats are foraging.

During phase 1, for $0 \leq t < 1.6$ hours, the mean squared distance seems to increase linearly as bats are dispersing. The standard error grows during this phase because the number of recordings $n$ increases as tagged bats are found and the signal is picked up. During phase 2, for $1.6 \leq t < 8$ hours, the mean squared distance decreases at an increasing rate as bats move back towards the roost, shrinking to zero at $t \approx 8$ hours. The variation shrinks to zero during this phase because the number of recordings $n$ reduces later
in the night when the signal is lost.
%
\begin{figure} [h]
    \centering
        \includegraphics[width=\textwidth]{RadioTrack_MSD.png}
        \caption{The mean-squared distance (MSD) for all radio tracked bats.
        The red dots are the averaged values over 56 trajectories and the standard error is shown as a ribbon.}
    \label{fig:MSD}
\end{figure}
