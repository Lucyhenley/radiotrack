
	\chapter{Introduction}
	\label{chapter:intro}


Bats play an important role in the UK ecosystem as they control insect
populations \cite{Kunz2011} and act as ecological indicators \cite{Jones2009}.
However, they are susceptible to human impacts due to their sensitivity to
light, noise and temperature, especially during hibernation \cite{Jones2009}. As
a result, many species are under threat, and bats are now protected by law under
the EUROBATS agreement \cite{Eurobats}. The breeding sites or resting places of
bats are known as roosts, and can be found in trees and buildings as well as
underground in caves and mines \cite{Eurobats}. Some species are highly site
faithful, returning to the same roosts year after year and are therefore reliant
on the suitability of the surrounding habitat \cite{Lewis1995}. Locating roosts
is important for conservation, in ensuring the roosts as well as the surrounding
habitat is preserved. Locating roosts is also important for research: monitoring
bat populations tends to be challenging as most species are small, nocturnal and
elusive, and roost surveys provide a useful way of studying populations
\cite{Flaquer2007}.

\section{Objectives}

Bat roosts tend to be difficult to find as they are generally located in rural
areas. The maximum foraging radius, $R_f$, is the maximum straight line distance from
the roost bats will travel. This varies between species, and for most species in the
UK is between 2km and 4km \cite{CSZ}. Detecting a bat therefore indicates that
there is a roost within distance $R_f$ of that point. For $R_f$=3km, this equates to
an area of approximately 30km\textsuperscript{2} to be searched, a task that is
both expensive and labour intensive.

The aim of this project is to develop a mathematical model to estimate roost
locations. The model should narrow down the search area in order to decrease the
time taken to locate roosts.


\section{Existing methods to locate roosts}

Radio tracking surveys are often used to directly trail bats back to their
roost. Unfortunately, radio tracking is challenging as it involves catching a
bat in order to collar it, and then following the bat in order to keep the radio
signal \cite{Lewis1995}.

Geographic profiling is a method used in criminology to identify areas where a
criminal is likely to live based on a series of related crimes
\cite{Rossmo1999}. The method creates a probability surface of the area using
the distance between crimes and each grid square. This has been applied to
locating bat roosts \cite{Comber2006} using foraging sites instead of crimes.
The disadvantage of this method is that it requires knowledge of foraging sites
used by the bats, which are often as difficult to discover as the roosts themselves.
