
\chapter{Proposed methodology}
\label{chapter:methods}

Three different methods have been used to estimate the locations of bat roosts, each requiring data from acoustic surveys. Two use a diffusion model for bat movement, fitting the model to acoustic detector data to estimate the most likely roost location. Approximate Bayesian Computation uses Bayesian inference to compare the model to the data, and generalised method of moments estimation generates a point estimate by minimising a function of the parameters and data. Geographic profiling does not use the diffusion model, and instead generates a probability surface of the search area using the distance from each grid square to each bat detection.

\section{Bat surveys}

Many bats use echolocation to navigate and catch prey
\cite{schnitzler2003}, and acoustic surveys using bat detectors have proved a
useful and cost-effective method to study populations \cite{Walters2012}.
Acoustic bat detectors are microphones calibrated to record the high-frequency
sound of bat calls and allow for long-term, autonomous surveying.

Surveys are conducted by placing static acoustic detectors within a search
area which record the calls of bats passing through their range. The coordinates
of each detector are noted and a series of recordings is produced. The species
for each recording can be identified using the sound of the call and a list of
times at which a bat of the species in question has been recorded at each
detector is produced.

%
\section{A Diffusion-Convection model for bat movement}

%Partial differential equations are widely used to model a variety of phenomena, a

The diffusion-convection equation is a combination of diffusion and convection
\begin{equation}
  \D{\phi(r,t)}{t} = D\nabla^2\phi(x,y,t) - \chi \nabla\phi(x,y,t),
\end{equation}

$r \geq 0$

where $D$ is the diffusion coefficient, and $\chi$ is the convection coefficient. The initial condition

\begin{equation}
  \phi(r,0) = \delta(\theta),
\end{equation}

where $\theta$ is the location of the roost, specifies that all bats start the night in the roost.

The diffusion-convection equation can also be written in polar coordinates, noting that since diffusion is symmetric,  $\D{\phi}{\theta} = 0$.

\begin{equation}
  \D{\phi(r,t)}{t} = \frac{D}{r} \D{}{r}\left(r \D{\phi(r,t)}{r} \right) - \chi \D{\phi(r,t)}{r}.
\end{equation}

Approximate solutions can be found using perturbations, assuming that either convection or diffusion dominates. The solution $\phi(r,t)$ can be written as a superposition of two solutions with a perturbation,

\begin{equation}
  \phi(r,t) \approx \phi_0(r,t) + \epsilon \phi_1(r,t),
  \label{eqn:phi_perturbation}
\end{equation}

where $\epsilon$ is an arbitrary constant describing the strength of the perturbation. Since $\epsilon$ is arbitrary, the initial condition means that

\begin{equation}
  \phi(r,0) = \phi_0(r,0) = \phi_1(r,0),
\end{equation}

When diffusion dominates, the convection term acts as a perturbation, and the equation becomes

\begin{equation}
  \D{\phi(r,t)}{t} = \frac{D}{r} \D{}{r}\left(r \D{\phi(r,t)}{r} \right) - \epsilon\chi \D{\phi(r,t)}{r}.
  \label{eqn:convection_perturbation}
\end{equation}

Substituting \eqn{eqn:phi_perturbation} into \eqn{eqn:convection_perturbation} gives

\begin{equation}
\begin{split}
  \D{\phi_0(r,t)}{t} + \epsilon\D{\phi_1(r,t)}{t} &= \frac{D}{r} \left( \D{\phi_0(r,t)}{r} + r \DD{\phi_0(r,t)}{r} + \epsilon\D{\phi_1(r,t)}{r} + r\epsilon \DD{\phi_1(r,t)}{r}\right)  \\
  &- \epsilon\chi \left( \D{\phi_0(r,t)}{r} + \epsilon \D{\phi_1(r,t)}{r} \right)
  \end{split}
\end{equation}

An equation for $\phi_0(r,t)$ can be found by collecting coefficients of $\epsilon^0$,

\begin{equation}
  \D{\phi_0(r,t)}{t} = \frac{D}{r} \left(\D{\phi_0)(r,t)}{r} + r\DD{\phi_0(r,t)}{r}\right) - \epsilon\chi \D{\phi_0(r,t)}{r}.
  \label{eqn:diffusion_eps0}
\end{equation}

This is the diffusion equation in polar coordinates and can be solved by first converting to Cartesian coordinates,

\begin{equation}
  \D{\phi_0(x,y,t)}{t} = D \left( \DD{\phi_0(x,y,t)}{x}+\DD{\phi_0(x,y,t)}{y}\right),
  \label{eqn:cartesian_diffusion}
\end{equation}

and using a Fourier transform in the spatial dimensions. Writing the Fourier transform of $\phi_0(x,y,t)$ as $\mathscr{F}\left[ \phi_0(x,y,t) \right] = \tilde{\phi_0}(k_x,k_y,t)$, each term in equation

\begin{equation}
\begin{split}
  \mathscr{F}\left[ \DD{\phi_0(x,y,t)}{x} \right] &= \int \int \DD{\phi_0(x,y,t)}{x} \mathrm{e}^{-2\pi i(xk_x + yk_y)} dx dy \\
  &= -4\pi^2k_x^2 \tilde{\phi_0}(k_x,k_y,t), \\
  \mathscr{F}\left[ \DD{\phi_0(x,y,t)}{y} \right] &= \int \int \DD{\phi_0(x,y,t)}{y} \mathrm{e}^{-2\pi i(xk_x + yk_y)} dx dy \\
  &= -4\pi^2k_y^2 \tilde{\phi_0}(k_x,k_y,t) \\
  \mathscr{F}\left[\DD{\phi_0(x,y,t)}{t}\right] &= \DD{\tilde{\phi_0}(k_x,k_y,t)}{t} \\
  \mathscr{F}\left[\phi_0(x,y,0)\right] &= \mathscr{F}\left[\delta(0)\right] \\
  &= 1
\end{split}
\end{equation}


Substituting into \eqn{eqn:cartesian_diffusion}

\begin{equation}
  \DD{\tilde{\phi_0}(k_x,k_y,t)}{t} = -4\pi^2D (k_x^2 + k_y^2) \tilde{\phi_0}(k_x,k_y,t).
\end{equation}

Integrating with respect to time gives

\begin{equation}
\begin{split}
  \tilde{\phi_0}(k_x,k_y,t) &= \int -4\pi^2D (k_x^2 + k_y^2) \tilde{\phi_0}(k_x,k_y,t) dt \\
  &= \tilde{\phi_0}(k_x,k_y,0)\mathrm{e}^{-4\pi^2D(k_x^2 + k_y^2)t} \\
\end{split}
\end{equation}

IC $ \tilde{\phi_0}(k_x,k_y,0) = 1$ and so

\begin{equation}
  \tilde{\phi_0}(k_x,k_y,t) = \mathrm{e}^{-4\pi^2D(k_x^2 + k_y^2)t} \\
\end{equation}

Then inverse fourier transform to recover $\phi_0(x,y,t)$

\begin{equation}
  \phi_0(x,y,t) = \mathscr{F}^{-1}\left[\tilde{\phi_0}(k_x,k_y,t)\right] = \frac{1}{4\pi Dt}\mathrm{e}^{-\frac{x^2 + y^2}{4Dt}} \\
\end{equation}

Transforming back to polar coordinates gives

\begin{equation}
  \phi_0(x,y,t) = \frac{1}{4\pi Dt}\mathrm{e}^{-\frac{r^2}{4Dt}} \\
  \label{eqn:phi_0}
\end{equation}

Then gathering coefficients of $\epsilon^1$:

\begin{equation}
\D{\phi_1(x,y,t)}{t} = \frac{D}{r}\left(\D{\phi_1(x,y,t)} + r \DD{\phi_1(x,y,t)}{r}\right) - \chi\D{\phi_0}{r}
\label{eqn:eps1}
\end{equation}

and for $\epsilon^2$:

\begin{equation}
\begin{split}
0 &= -\chi\D{\phi_1(x,y,t)}{r} \\
\D{\phi_1(x,y,t)}{r} &= 0
\end{split}
\end{equation}

Substituting this into \eqn{eqn:eps1} gives

\begin{equation}
  \D{\phi_1(x,y,t)}{t} = - \chi\D{\phi_0(x,y,t)}{r} = - \frac{\chi r}{2Dt} \phi_0(x,y,t)
\end{equation}

%
\section{A diffusion model for bat movement}

Diffusion models are widely used to model animal movement for a number of species \cite{Ovaskainen2016}. In this case, a 2D diffusion model is used to describe the movement of bats flying away from the roost.

If the roost is at $(x_0,y_0)$ and bats leave the roost at time $t =0$,
the 2D diffusion equation describes the probability density $\phi(x,y,t)$ of
finding a bat at position $(x,y)$ at time $t$,

\begin{equation}
  \D{\phi(x,y,t)}{ t} = D \nabla^2 \phi(x,y,t) ,
  \nonumber
\end{equation}
%
where $D$ is the diffusion coefficient and quantifies the speed with which bats diffuse.

The diffusion equation can also be written in polar coordinates:

\begin{equation}
\D{ \phi(r,t)}{t} = \frac{D}{r} \D{}{ r} \left( r \D{\phi(r,t)}{r} \right),
\end{equation}
%
where $r$ is the distance from the roost, given by $r=\sqrt{(x-x_0)^2 + (y-y_0)^2}$. As diffusion is symmetric, $\phi$ is only dependent on $r$ and not on the angle.

The initial condition

\begin{equation}
\phi(x_0,y_0,0) = \delta(x_0,y_0)
\label{eqn:IC}
\end{equation}
%
specifies that all bats begin the night at the roost.

Motion is determined by the parameter $D$, the diffusion coefficient. The diffusion coefficient determines the speed of motion away from the origin and can be calculated using the mean squared distance (MSD) $\mathbb{E}[r^2]$. The expected MSD at time $t$ can be calculated using the diffusion equation:

\begin{equation}
\left<r^2\right> = \int_{\infty}r^2 \phi(r,t) d\Omega ,
\end{equation}
%
where the integral is over all space.

The rate of change of the MSD with time can be used to calculate the relationship between MSD and time:

\begin{equation}
\begin{split}
\frac{d}{dt} \left<r^2\right> &= \frac{d}{dt}\int_{\infty}r^2 \phi(r,t) d\Omega \\
                            &= \frac{d}{dt} \int_0^{2\pi}\int_0^{\infty} r^2 \phi(r,t) r dr d\theta \\
                           &= \int_0^{2\pi} \int_0^{\infty} r^3 \D{\phi}{l t} dr d\theta \\
                            &= \int_0^{2\pi} \int_0^{\infty} r^3 \frac{D}{r} \D{}{r } \left( r \D{ \phi}{ r}\right) dr d\theta \\
                            &= \int_0^{2\pi} \left( \left[ D r^2 \left( r \D{ \phi}{ r}\right) \right]_0^{\infty} - \int_0^{\infty} 2rD \left(r \frac{\partial \phi}{\partial r} \right) dr \right) d\theta \\
                            &= \int_0^{2\pi} \int_0^{\infty} -2r^2D \D{ \phi}{r}dr d\theta \\
                            &= \int_0^{2\pi} \left( \left[-2r^2D \phi \right]_0^{\infty} + \int_0^{\infty} 4rD \phi dr \right)d\theta \\
                            &= \int_0^{2\pi} \int_0^{\infty} 4rD\phi dr d\theta \\
                            &= 4D \int_{\infty} \phi d\Omega \\
\frac{d}{dt} \left<r^2\right>  &= 4D \\
\label{eqn:diffusion_1}
\end{split}
\end{equation}
%
Integrating with respect to time gives
%
\begin{equation}
\left<r^2\right> = 4Dt ,
\label{eqn:D_msd}
\end{equation}
%
and therefore MSD is directly proportional to time.
%
\subsection{Advection-Diffusion model}
%
Motion is not diffusive for the whole night, as bats must return back to the roost by the morning. An advection-diffusion model adds an extra term to the simple diffusion equation that specifies a drift towards a given location. In this case, the drift will be towards the roost. In polar coordinates, the advection-diffusion equation is
%
\begin{equation}
  \D{\phi(r,t)}{t} = \frac{D}{r} \D{}{r}\left(r \D{\phi(r,t)}{r} \right) - \chi \D{\phi(r,t)}{r},
  \nonumber
\end{equation}
%
where $\chi$ is the advection coefficient, the strength of the drift back towards the roost.
%
The MSD for the advection-diffusion model in terms of $D$ and $\chi$ can be calculated using the same method as for the simple diffusion model:
%
\begin{equation}
\begin{split}
\frac{d}{dt} \left<r^2\right> &= \frac{d}{dt}\int_{\infty}r^2 \phi(r,t) d\Omega \\
                              &= \frac{d}{dt} \int_0^{2\pi}\int_0^{\infty} r^2 \phi(r,t) r dr d\theta \\
                              &= \int_0^{2\pi} \int_0^{\infty} r^3 \D{\phi}{t} dr d\theta \\
                              &= \int_0^{2\pi} \int_0^{\infty} r^3 \frac{D}{r} \D{}{r}\left(r \D{\phi}{r}\right)- r^3\chi\D{\phi}{r} dr d\theta \\
\end{split}
\end{equation}
%
Substituting the first term from \eqn{eqn:diffusion_1}:
%
\begin{equation}
\begin{split}
                              &= 4D - \chi \int_0^{2\pi} \int_0^{\infty} r^3 \D{\phi}{r} dr d\theta \\
                              &= 4D - \chi \int_0^{2\pi} \left[ r^3 \phi \right]_0^{\infty} - \int_0^{\infty}  3 r^2 \phi dr d\theta \\
                              &= 4D + 3\chi \int_0^{2\pi} \int_0^{\infty}  r^2 \phi dr d\theta  \\
                              &= 4D + 3\chi \int_{\infty} r \phi d\Omega \\
                              &= 4D + 3\chi \left<r\right>
\nonumber
\end{split}
\end{equation}
%
The mean distance $ \left<r\right>$ can be calculated in the same way:
%
\begin{equation}
\begin{split}
\frac{d}{dt} \left<r\right> &= \frac{d}{dt}\int_{\infty}r \phi(r,t) d\Omega \\
                              &= \frac{d}{dt} \int_0^{2\pi}\int_0^{\infty} r \phi(r,t)r dr d\theta \\
                              &= \int_0^{2\pi} \int_0^{\infty} r^2 \D{\phi}{t} dr d\theta \\
                              &= \int_0^{2\pi} \int_0^{\infty} r^2 \left[ \frac{D}{r} \D{}{r}\left(r \D{\phi}{r}\right)-\chi\D{\phi}{r} \right] dr d\theta \\
                              &= \int_0^{2\pi} \int_0^{\infty} rD  \D{}{r}\left(r \D{\phi}{r}\right) -  \chi r^2 \D{\phi}{r} dr d\theta \\
\nonumber
\end{split}
\end{equation}
%
\begin{equation}
\begin{split}
\frac{d}{dt} \left<r\right> &= \sqrt{\frac{D}{t}} + \int_0^{2\pi} \int_0^{\infty} -  \chi r^2 \D{\phi}{r} dr d\theta \\
                            &= \sqrt{\frac{D}{t}} + \int_0^{2\pi} - \left[ \chi r^2 \phi \right]_0^{\infty} + \int_0^{\infty} 2 \chi r \phi dr d\theta \\
                            &= \sqrt{\frac{D}{t}} + \int_0^{2\pi} \int_0^{\infty} 2 \chi r \phi dr d\theta \\
                            &= \sqrt{\frac{D}{t}} + \int_{\infty} 2 \chi \phi d\omega \\
                            &= \sqrt{\frac{D}{t}} + 2\chi
\nonumber
\end{split}
\end{equation}
%
Integrating with respect to time gives
%
\begin{equation}
\left<r\right> = 2\chi t .
\nonumber
\end{equation}
%
Substituting into the equation for $\left<r^2\right>$ gives
%
\begin{equation}
\begin{split}
\frac{d}{dt} \left<r^2\right> &= 4D + 6\chi\sqrt{Dt}  + 6\chi^2t \\
\left<r^2\right>              &= 4Dt + 12\chi\sqrt{Dt^3} + 3\chi^2t^2 .
\label{eqn:AD_msd}
\end{split}
\end{equation}
%
\subsection{Fitting the model to radio tracking data}
%
The diffusion model was fit to radio tracking data from 11 Greater Horseshoe bats, surveyed over the course of 26 nights. The data was recorded by placing a radio collar on a bat, and attempting to follow it over a night. Following bats in order to take measurements is challenging as they move quickly and are able to fly over a variety of terrains which are inaccessible to humans (for example, across privately owned or fenced off land). Therefore, measurements are not taken at regular intervals, and instead when a signal is detected the coordinates and times are recorded. In order to generate trajectories, the data was interpolated at evenly spaced time intervals. An example of the trajectories generated is shown in \fig{fig:Diffusion_eg} for four bats. The trajectories show what appears to be a random walk, with bats starting the night at their roost. The squared displacement from the roost was calculated for each bat on each survey night at time intervals of 200 seconds, and the mean at each time was taken to give the MSD. The results of this are shown in \fig{fig:RadioTrackD_SD}. These results show an initial straight line segment for $0 < t < 4 \times 10^3 s$, consistent with a diffusion model. The later section shows that bats drift back towards the roost, and all return to the roost by the end of the night. To calculate the diffusion coefficient $D$, the initial straight line segment for $0 < t < 4 \times 10^3$ was fit to \eqn{eqn:D_msd}. The estimate calculated was $D = 64.5 \pm 30 m^2s^{-1}$.

\begin{figure} [h]
\centering
      \includegraphics[width=0.4\textwidth]{Diffusion_eg.pdf}
      \caption{Trajectories for four bats over one night each. These were created by linearly interpolating between points where bats were recorded during radio tracking.}
      \label{fig:Diffusion_eg}
\end{figure}

\begin{figure} [h]
\centering
      \includegraphics[width=0.6\textwidth]{RadioTrack_SD.pdf}
      \caption{Estimates for MSD interpolated over evenly spaced time points. The filled area represents $\pm 1$ standard error for each estimate.}
      \label{fig:RadioTrackD_SD}
\end{figure}

\begin{figure} [h]
\centering
      \includegraphics[width=0.6\textwidth]{RadioTrack_fit.pdf}
      \caption{Estimates for MSD interpolated over evenly spaced time points. The filled area represents $\pm 1$ standard error for each estimate.
      A diffusion model with $D = 64.5 m^2s^{-1}$ was fit for $0 < t < 4 \times 10^3 s$, and an advection-diffusion model with $\chi = 0.00077 ms^{-1}$ and $D = 2.21 m^2s^{-1}$ was fit for $t > 4 \times 10^3 s$.}
      \label{fig:RadioTrackFit}
\end{figure}


\subsection{Using the diffusion model with static detector data}

For the parameters found using radio tracking data, motion for the first first hour after sunset approximates simple diffusion, with no bounded domain. As a result, the model used in locating roosts will use simple diffusion, and only use data from the first hour after sunset. The model will use the same initial condition as before, with all bats beginning the night at the roost. If the roost is at the origin and bats leave the roost at time $t =0$, the 2D diffusion equation describes the probability density $\phi(x,y,t)$ of finding a bat at position $(x,y)$ at time $t$,

\begin{equation}
  \frac{\partial \phi(x,y,t)}{\partial t} = D \nabla^2 \phi(x,y,t) .
  \nonumber
\end{equation}
%
The value for $D$ found using radio tracking surveys will be used here, $D = 75.5$ m\textsuperscript{2}s\textsuperscript{-1}. Solving for $\phi(x,y,t)$ gives

\begin{equation}
  \phi(x,y,t) = \frac{1}{4 \pi D t} e^{-\frac{x^2 + y^2}{4Dt}}. \cite{Ovaskainen2016}
  \label{eqn:phi}
\end{equation}


Integrating $\phi(x,y,t)$ over the range $R$ of a detector $i$ at $(x_i, y_i)$ gives
the expected density of bats in range of detector $i$ at time $t$. The
integral is approximated using a quadratic Taylor expansion around $(x_i, y_i)$. A 2D Taylor expansion of $f(x,y)$ about point $(x_i, y_i)$ is given by

\begin{equation}
\begin{split}
f(x,y) &= f(x_i,y_i) + f_x(x_i,y_i)\Delta x + f_y(x_i,y_i)\Delta y \\
       &+ \frac{1}{2}\left[ f_{xx}(x_i,y_i) \Delta x^2 + 2 f_{xy}(x_i,y_i) \Delta x \Delta y + f_{yy}(x_i,y_i) \Delta y^2 \right] + O(\Delta x^3 + \Delta y^3) .
\end{split}
\label{eqn:taylor}
\end{equation}
%
where $\Delta x = x-x_i$ and $\Delta y = y - y_i$. For $\phi$ in \eqn{eqn:phi}, the derivatives are

\begin{equation}
\begin{split}
\phi_x(x_i,y_i,t) &= \frac{-x_i}{2Dt} \phi(x_i,y_i,t) \\
\phi_y(x_i,y_i,t) &= \frac{-y_i}{2Dt} \phi(x_i,y_i,t) \\
\phi_{xx}(x_i,y_i,t) &= \phi(x_i,y_i,t) \left[ \frac{-1}{2Dt} + \frac{x_i^2}{4D^2t^2}\right] \\
\phi_{yy}(x_i,y_i,t) &= \phi(x_i,y_i,t) \left[ \frac{-1}{2Dt} + \frac{y_i^2}{4D^2t^2}\right] \\
\phi_{xy}(x_i,y_i,t) &= \frac{x_iy_i}{4D^2t^2}\phi(x_i,y_i,t)
\end{split}
\nonumber
\end{equation}

Substituting derivatives into \eqn{eqn:taylor} gives

\begin{multline}
\phi(x,y,t) = \phi(x_i,y_i,t) \left[1 - \frac{x_0}{2Dt}\Delta x - \frac{y_i}{2Dt} \Delta y + \left( \frac{-1}{4Dt} + \frac{x_i^2}{8D^2t^2} \right)\Delta x^2  \right. \\
\left. + \left( \frac{-1}{4Dt} + \frac{y_i^2}{8D^2t^2} \right) \Delta y^2 + \frac{x_iy_i}{4D^2t^2}\Delta x \Delta y \right] + O(\Delta x^3 + \Delta y ^3)
\end{multline}

Writing $\phi(x_i,y_i,t)$ as $\phi_i(t)$ and changing variables to $\Delta x = r\cos{\theta}$ and $\Delta y = r\sin{\theta}$:

\begin{multline}
\phi(r,\theta,t) \approx \phi_i(t) \left[ 1 - \frac{x_i}{2Dt} r\cos{\theta} - \frac{y_i}{2Dt} r\sin{\theta} +  r^2 \cos^2{\theta} \left(\frac{-1}{4Dt} + \frac{x_iy_i^2}{8D^2t^2} \right) \right. \\
\left. + r^2\sin^2{\theta}\left(\frac{-1}{4Dt} + \frac{y_i^2}{8D^2t^2} \right) + \frac{x_i y_i}{4D^2t^2} r^2\sin{\theta}\cos{\theta} \right]
\label{eqn:phi_polar}
\end{multline}

The expected density $N_i(t)$ of bats in the range of detector $i$ at point $(x_i,y_i)$ at time $t$ is found by integrating over the circle of radius $R$ around the detector:

\begin{multline}
N_i(t) = \int_0^R \int_0^{2\pi} \phi(r,\theta,t) r d\theta dr \\
= \phi_i(t) \int_0^R \left[ 2\pi + \frac{y_i}{2Dt}r + \pi r^2 \left(\frac{-1}{4Dt} + \frac{x_i^2}{8D^2t^2}  \right) + \pi r^2 \left( \frac{-1}{4Dt} + \frac{y_i^2}{8D^2t^2} \right) \right. \\
\left. - \frac{r^2}{2} \frac{x_0y_0}{4Dt}  - r \frac{y_i}{2Dt}  + \frac{r^2}{2} \frac{x_iy_i}{4D^2t^2} \right] r dr \\
= \phi_i(t) \int_0^R 2\pi r + \pi r^3 \left( \frac{-1}{2Dt} + \frac{x_i^2+y_i^2}{8D^2t^2} \right)  dr \\
= \phi_i(t) \left[ \pi r^2 + \frac{r^4\pi}{4}\left( \frac{-1}{2Dt} + \frac{x_i^2 + y_i^2}{8D^2t^2} \right) \right]_{r=0}^{r=R} \\
= \phi_i(t) \left[ \pi R^2 + \frac{\pi R^4}{4} \left( \frac{x_i^2 + y_i^2}{8D^2t^2} - \frac{1}{2Dt} \right) \right] \\ .
\nonumber
\end{multline}
%
Substituting in $\phi_i(t)$ from \eqn{eqn:phi_polar} gives

\begin{equation}
  N_i(t) = \frac{1}{4Dt} e^{-\frac{x_i^2 + y_i^2}{4Dt}} \left[ R^2 + \frac{R^4}{4} \left( \frac{x_i^2 + y_i^2}{8D^2t^2} - \frac{1}{2Dt} \right)\right] .
\end{equation}

The total number of hits expected at each
detector for $0 < t < T$ is calculated by integrating over time,

\begin{equation}
  \tilde{N_i}(T) = \int_{t=0}^{t=T} N_i(t) dt.
\end{equation}
%
At present, this is approximated using a Riemann sum at intervals $\Delta t$:

\begin{equation}
  \tilde{N_i}(T) \simeq \sum_{t=0}^{t=T} N_i(t) \Delta t.
  \label{eqn:expect_hits}
\end{equation}
%
In future, this approximation will be improved by using a better numerical
integration method. The number of hits expected at each detector $\tilde{N_i}(T)$ for a possible roost location $\bm{\theta'}$ will be compared to the number of hits found in bat surveys to estimate the probability that the true roost is at $\bm{\theta'}$.

%This model assumes infinite space and there are no boundary conditions, so that bats can travel any distance away from the roost. However, it can be shown that the probability of a bat travelling further than the maximum foraging radius $R_f =$ 3000m calculated using $\phi$ is typically small. Integrating $\phi$ using a numerical integration algorithm \cite{Berntsen1991} over the first hour  after sunset and over space for $x^2 + y^2 > R_f^2$ gives a proability that a bat will exceed $R_f$ of $p = 4.5 \times 10^{-6}$ and therefore the effects of the boundary are negligible in this case.

%
\section{Bayesian statistics}

Bayesian statistics is a statistical paradigm for updating knowledge about a
parameter given related data \cite{Gelman2013}. The probability distribution of a parameter
$\theta$ conditioned on observations $\bm{Y} = \{ Y_1, Y_2, ..., Y_n \}$ is
given by Bayes' Theorem:

\begin{equation}
  p(\theta \mid \bm{Y}) \propto p(\bm{Y} \mid \theta) p(\theta)
  \nonumber
\end{equation}
%
where $p(\theta \mid Y)$ is the posterior probability distribution, formally
describing the probability that the parameter value is $\theta$ given
observations $\bm{Y}$. The likelihood is given by $p(\bm{Y} \mid \theta)$ and
is the probability of observing $\bm{Y}$ if the parameter value is $\theta$. The
prior distribution is $p(\theta)$, describing the initial knowledge of possible
parameter values.

In our case, the parameter $\theta$ is the roost location, and $\bm{Y}$ is the
number of hits at each detector found in the bat surveys. The posterior
distribution $ p(\theta \mid \bm{Y})$ will be a distribution describing the
probability of possible roost locations. The prior distribution $p(\theta)$
covers the initial search area, and is assumed to be uniform over the original
search area (the circle of radius 3km centred around the original detected bat).
The likelihood $p(\bm{Y} \mid \theta)$ is problematic as the dependencies
between components of $\bm{Y}$ are unclear, and therefore the probability of
recording a given $\bm{Y}$ is difficult to estimate.

\section{Approximate Bayesian Computation (ABC)}
\label{section:ABC}

Approximate Bayesian Compuation (ABC) is an approach to Bayesian inference using
simulation \cite{Beaumont2002, Sisson2010}. ABC replaces the calculation
of the likelihood function $p(\bm{Y} \mid \theta)$ with simulation of the model
using a specific parameter value $\theta'$ to produce an artificial dataset
$\bm{X}$. Then, some distance metric $\rho (\bm{X}, \bm{Y}) $, usually defined
as a distance between summary statistics of $\bm{X}$ and $\bm{Y}$, is used to
compare simulated data $\bm{X}$ to observations $\bm{Y}$. If $\rho (\bm{X},
\bm{Y}) $ is smaller than some threshold value $\epsilon$, the simulated data is
close enough to observations that the candidate parameter $\theta'$ has some
nonzero probability of being in the posterior distribution $ p(\theta \mid
\bm{Y})$, and the sample $\theta'$ is accepted into the simulated posterior
distribution. This is repeated until the desired sample size is reached. For
small $\epsilon$, the simulated posterior distribution produced approximates the
true posterior $ p(\theta \mid \bm{Y})$.

\subsection{The ABC Algorithm}

First, the threshold parameter $\epsilon$ and sample size $n$ are fixed and the
prior distribution $p(\theta)$ is defined as uniform over the circle of 3km
radius around the original detected bat, $p(\theta) = \frac{1}{3^2 \pi}$.

Observations $\bm{Y}$ are given by the number of hits at each detector from bat
surveys and dataset $\bm{X}$ is the expected number of hits at each detector for
a roost at $\theta'$ given by Equation \ref{eqn:expect_hits}. The posterior
distribution $p(\theta \mid \bm{Y})$ for roost locations will be given by
$\bm{\theta}$, where $\bm{\theta_i}$ is the $i$-th accepted sample.

The pseudocode is as follows:

\begin{algorithmic}
  \While {$i < n$}
    \State  Sample $\theta'$ from $p(\theta)$
    \State Calculate $\bm{X}$ using Equation \ref{eqn:expect_hits}
    \State $\rho \gets \mid \bm{X} - \bm{Y} \mid$
    \If {$\bar{\rho} < \epsilon$}
      \State $\bm{\theta_i} \gets \theta'$
    	\State $i \gets i + 1$
    \EndIf
  \EndWhile
\end{algorithmic}

The probability distribution function of the roost search area is then given by $\bm{\theta}$, and the true location of the
roost can be estimated by taking the mean of the posterior $\bm{\theta}$.

%
\section{Generalised Method of Moments}

Generalised method of moments (GMM) estimation is an alternative parameter estimation method, and is commonly used in econometrics \cite{Hansen1982, Hall1993}. It provides a point estimate by minimising a function of the parameters and the data, and is usually a more efficient method than a random search as in ABC. GMM is based on the specificication of certain moment conditions, functions of a model's parameters $\theta$ and a set of observations $Y$,
%
\begin{equation}
g(\theta) = \mathbb{E}[f(Y, \theta)] ,
\label{eqn:moment}
\end{equation}
%
where $f$ is a set of functions, one for each observation in $Y$. The functions $f$ are specified such that the expectation is zero at the true parameter value $\theta_0$,
%
\begin{equation}
g(\theta_0) = \mathbb{E}[f(Y, \theta_0)] = 0.
\label{eqn:moment}
\end{equation}

Calculating the expectation for a given sample $t$ is often impossible, so this can be replaced with sample averages to obtain sample moments,
%
\begin{equation}
g_T(\theta) = \frac{1}{T} \sum_{t=1}^{T} f(Y_t, \theta) .
\end{equation}
%
The moment condition becomes
%
\begin{equation}
g_T(\theta_0) = \frac{1}{T} \sum_{t=1}^T f(Y_t, \theta_0) .
\end{equation}

In the case where there are more functions $f$ than parameters $\theta$, the problem is over-identified and there is no solution to $g_T(\theta_0) = 0$. Instead, the distance $g_T(\theta) - 0$ is minimized. This distance is measured by the quadratic form
%
\begin{equation}
Q_T(\theta) = g_T(\theta)^T W_T g_T(\theta),
\label{eqn:GMM_distance}
\end{equation}
%
where $W_T$ is a symmetric and positive weight matrix and $g_T(\theta)^T$ denotes the transpose of $g_T(\theta)$. The GMM estimator is then the minimiser of \eqn{eqn:GMM_distance},
%
\begin{equation}
\begin{split}
\theta_0 &= \arg \min_{\theta}(g_T(\theta)^T W_Tg_T(\theta)) \\
         &= \arg \min_{\theta} \left[ \left( \frac{1}{T}\sum_{t=1}^T f(Y_t, \theta_0) \right)^T W_T \left( \frac{1}{T}\sum_{t=1}^T f(Y_t, \theta_0) \right) \right] .
\end{split}
\label{eqn:GMM}
\end{equation}

Usually, the data is collected with successive measurements, with mean values $Y$. The variance $\sigma^2$ of each entry of $Y$ is used to determine an optimal weighting matrix which should provide the most efficient search,
%
\begin{equation}
W_T = \textrm{diag} \left(\frac{1}{\sigma^2} \right) .
\end{equation}

The most likely parameter value $\theta_0$ is generally found using numerical optimisation to minimise \eqn{eqn:GMM}.

\subsection{Applying GMM to locating roosts}

The distances $g_T(\theta)$ are calculated as the difference between $X$, the expected number of hits calculated using \eqn{eqn:expect_hits} and $Y$, the number of hits at each detector in the bat surveys,
%
\begin{equation}
g_T(\theta) = \mathbb{E}X(\theta) - Y .
\end{equation}
%
However, the weighting matrix is more problematic as the variance of successive samples is not always calculable. Detectors are placed out for multiple nights, but sometimes fail as they can be damaged by animals. Therefore, some detectors record only one night's data and so there is no variance in these measurements. Instead, since the time between recordings is analagous to waiting times we can assume that the time between detector hits are poisson distributed. The variance of a poisson distribution is equal to its mean, so the mean number of hits for each detector is used instead of variance for the weighting matrix, $\sigma^2 \approx Y$. Some detectors record no hits and $Y=0$ in this case, so the value is shifted to avoid dividing by zero, $\sigma^2 \approx Y+1$. The weighting matrix is therefore

\begin{equation}
W_T = \textrm{diag} \left(\frac{1}{1+Y} \right) .
\label{eqn:weighting}
\end{equation}

The GMM optimiser is then given by

\begin{equation}
\theta_0 = \arg \min_{\theta} \left( (\mathbb{E}X(\theta) - Y)^T\textrm{diag} \left(\frac{1}{1+Y} \right) .(\mathbb{E}X(\theta) - Y) \right) .
\label{eqn:roost_gmm}
\end{equation}

The minimum can be found using a numerical optimiser. In this case, the function given by \eqn{eqn:roost_gmm} has many local minima so a global optimiser is needed. The adaptive differential evolution optimiser \textproc{adaptive\_de\_rand\_1\_bin} found in Julia package BlackBoxOptim.jl \cite{Feldt2018} is used.

%
\section{Geographic profiling}
Geographic profiling is a method used in criminology to identify areas where a criminal is likely to live based on a series of related crimes
\cite{Rossmo1999}. The method creates a probability surface of the area using the distance between crimes and each grid square. The method assumes that crimes are most likely to occur close to the offenders residence. However, it also includes an area of low probability in the offenders immediate neighbourhood, termed the buffer zone. This is because criminals are less likely to offend very close to their own home, due to to low anonymity and the risk of being seen by their neighbours. The model splits the area where crimes have occurred into a grid, and for each grid square calculates a score function which describes the probability density of the offenders residence being in that square. The score function is given by

\begin{equation}
p_{ij} = \sum_{n=1}^{C}\left[ \frac{\phi}{\left(|x_i - x_n| + |y_i - y_n|\right)^f} + \frac{(1-\phi)(B^{g-f})}{\left(2B - |x_i - x_n| - |y_i - y_n|\right)^g}    \right]  ,
\label{eqn:geo}
\end{equation}
%
where $B$ is the radius of the buffer zone, $C$ is the number of foraging sites, $f$ and $g$ are empirically determined exponents, chosen to give the most efficient search procedure. The coordinates of point $(i,j)$ are $(x_i,y_j)$ and $(x_n,y_n)$ are the coordinates of the $n$th site. $\phi$ is a weighting factor set to $0$ for sites within the buffer zone and $1$ for sites outside the buffer zone. The result of the model is a 3D probability surface, where the higher the grid square, the greater likelihood of offender residence. This surface provides an optimal search process based on searching the locations with the highest probability density first.

Geographic profiling has been successfully applied to locating pipistrelle bat roosts using the distribution of foraging sites instead of crimes \cite{Comber2006}. The size of the buffer zone was set to the mean distance pipistrelles travel from the roost, found to be 1.8km \cite{Racey1985}.

The disadvantage of this method is that it requires knowledge of foraging sites
used by the bats, which are often as difficult to locate as the roosts themselves. Instead, the method can be adapted to use use detector hits rather than foraging sites. As bats start the night at the roost, there will most likely be the most hits close to the roost and therefore the buffer zone was removed. Additionally, bats travel in straight lines rather than following a grid system and so the distance measure was changed from Manhattan distance to Euclidean distance. The equation was also modified to sum over detectors rather than hits. The new score function is given by

\begin{equation}
p_{ij} = \sum_{n=1}^{C}\left[ \frac{N_n}{\left((x_i - x_n)^2 + (y_i - y_n)^2\right) ^{\frac{f}{2}}   }   \right] ,
\label{eqn:geo2}
\end{equation}
%
where $(x_n,y_n)$ is now the coordinates of the $n$th detector and $N_n$ is the number of hits recorded at the $n$th detector.

