
\chapter{Preliminary Results}
\label{chapter:results}

In order to test the validity of the model, data from greater horseshoe bat
roosts in Devon, UK has been analysed. The results are presented in
\fig{fig:fixed_D}. The diffusion coefficient $D = 75.5$
m\textsuperscript{2}s\textsuperscript{-1} was calculated using data from radio
tracking surveys. Each set of data was collected by distributing detectors near
a variety of different landscape features (in hedgerows, near roads etc), marked
by blue crosses around Greater Horseshoe roost sites, marked in cyan in \fig{fig:fixed_D}. In
\fig{fig:buckfastleigh_250716}, the posterior distribution shown successfully
predicted the location of the roost, as the true roost location falls inside the
posterior distribution. The error between the true roost and the estimate,
shown in red, is 77m. For this set of data, the search area required would have
been narrowed down significantly, saving valuable time and resources in locating
the roost.

However, \fig{fig:gunnislake_080816} shows that the model is less successful for
the second roost studied. Whilst the posterior distribution is close to the true
roost, the roost still falls outside the posterior. The distance from the
estimated roost location and the true roost is 795m, significantly higher than
that calculated for roost 1.

There could be a number of reasons for this error due to the differences in the
data from each roost. Firstly, roost 1 has a larger population than roost 2, and
therefore the data collected is likely to be more representative of the
movements of the population. The landscape is also likely to play an important
part in bat movement, as bats prefer to forage along linear features such as
hedgerows and rivers and avoid noisy or brightly lit areas \cite{Stone2009}. As the model assumes
bats move completely homogeneously, it does not take this preference into
account. As detectors were placed near different landscape features for both
roosts, those close to suitable foraging locations are likely to have seen
proportionately more bats than those close to unsuitable habitat at the same
distance from the roost. The diffusion model also predicts a high density of
bats around the origin for all $t$, which may be inaccurate as bats tend to fly
away from the roost to forage. This therefore may lead to an overestimate in the
number of hits expected at detectors close to the roost, skewing the results
of the model by rejecting possible roost sites because they are close to a
detector.


\begin{figure} [h]
    \centering
    \begin{subfigure}[b]{\ttp}
        \includegraphics[width=\textwidth]{buckfastleigh_250716.pdf}
        \caption{Roost 1: Buckfastleigh, 25/07/2016}
        \label{fig:buckfastleigh_250716}
    \end{subfigure}
    ~ %add desired spacing between images, e. g. ~, \quad, \qquad, \hfill etc.
      %(or a blank line to force the subfigure onto a new line)
    \begin{subfigure}[b]{\ttp}
        \includegraphics[width=\textwidth]{gunnislake_080816.pdf}
        \caption{Roost 2: Gunnislake, 08/08/2016}
        \label{fig:gunnislake_080816}
    \end{subfigure}
    \caption{The results for surveys at two roosts. Each was implemented using the algorithm set out in section \sect{section:ABC} with $n=10^5$ and $\epsilon$ was chosen to give an acceptance rate of 3\%. The posterior distribution is shown as a heatmap.}\label{fig:fixed_D}
\end{figure}


\section{Future improvements to the model}

The model would be improved by including landscape features in the model. The
diffusion coefficient $D$ could be varied over space to account for the
avoidance of lights/roads and preferential foraging close to hedgerows and
rivers. The distance measure $\rho(\bm{X}, \bm{Y})$ used to compare the
simulated dataset to the results of the bat surveys could be improved by
including the number of hits over multiple time intervals.

The survey process could be improved by investigating if there is some optimum
detector placement to provide the best estimate. For example, does a regular
grid of detectors provide a better estimate than a random array, or is it best
to place all of the detectors near the same type of landscape?

%The error could be due to differing landscape features around the
%roost which are likely to affect bat movement. To investigate this, the
%diffusion coefficient $D$ was added as a second parameter to be varied. The
%prior was given by $D \sim N[\mu, \mu/2]$, where $\mu =75.5$, the value
%calculated from radio tracking surveys. The results are shown in
%\fig{fig:vary_D}. The results show more variation in the posterior distributions
%leading to a larger search area for both roosts. The results for roost 2 have
%improved as the posterior distribution now contains the roost. The prior and
%posterior distributions for $D$ shown in \fig{fig:D_dists} suggest that the
%value of $D = 75.5$ m\textsuperscript{2}s\textsuperscript{-1} calculated using
%the radio tracking survey may be an underestimate for both roosts.


%This could be due to the date of the survey: later in the summer,
%%once pups are old enough to survive the night alone, mothers will often sleep
%outside of the main roost, which could contribute to noise in the data. There
%could also be effects from the location of detectors; if a detector was placed
%close to a popular foraging site, it would be likely to detect a
%disproportionately large number of bats.

%Three surveys were conducted at roost 1 in Buckfastleigh, Devon during summer
%2016, in June, July and September. The results of the model are shown in Figure
%\ref{fig:buckfastleigh}. The results show that the model works well for survey
%2, in July, successfully locating the roost and narrowing down the search area.
%However, the model was less successful for surveys 1 and 3 in June and
%September: the posterior distributions do not include the roost.

%To improve roost estimates, the diffusion coefficient $D$ was added to the model
%as a second parameter to be varied. The prior is given by $D \sim N[\mu,
%\mu/2]$, where $\mu =75.5$, the value calculated from radio tracking surveys.
%The results are shown in Figure \ref{fig:buckfastleigh_varyD}. These results show
%that the posterior for $D$ is very different from the prior; the mean has
%shifted to $D = 118$ m\textsuperscript{2}s\textsuperscript{-1}.  The model is
%improved with varying $D$ for some sites, shown in Table \ref{table: results}.


%This is possibly due to the time of year; roosting and
%foraging behaviour varies over the summer, with some bats tending to roost in
%'day roosts' outside of the main roost later on in the summer. Also there are no
%detectors close to the roost for the other surveys.
