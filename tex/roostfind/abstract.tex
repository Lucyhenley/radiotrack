\section{Abstract}

Locating bat roosts is vital for both conservation and research purposes, as it
allows biologists to study bat populations and the impact of human behaviour on
populations. However, it is usually both difficult and labour intensive to find
roosts using traditional search methods. I propose a novel approach for locating
greater horseshoe bat roosts using data from static acoustic detectors and a
mathematical model of bat movement using diffusion modelling. The model is
fitted using Bayesian statistics. This method gives an actionable estimate of
the roost location and narrows the search area.



Bats populations are declining due to many reasons, including loss of habitat from human activity, thus, all bat species are fully protected by law. As bats play an important role in the UK ecosystem a reduction in the population could have catastrophic effects on biodiversity. As a result, ecological surveys are legally required when undertaking large-scale building work to locate breeding or resting places (roosts) in the area and determine whether the work will negatively impact on the local bat populations. However, locating roosts is generally a difficult, labour intensive task, requiring many hours of manually searching a vast area, and as a result surveys can be expensive. In collaboration with ecological experts I propose a novel approach for locating roosts using data from static acoustic detectors and a mathematical model using diffusive movement, fitted using Bayesian statistics. This provides an estimate of the roost location and a credible area in which the roost is likely to be. The method has been successful in locating Greater Horseshoe bat roosts, estimating the location to within 250m of the true roost and reducing the area to be searched by up to 90%. The use of this method in locating roosts could save surveyors countless hours of searching and significantly reduce the cost for developers. Future model developments will include effects such as roads, sound pollution and light pollution, allowing us to develop evidence for policy makers.
