%\section{Landscape effects}

%$ D = \alpha_0 e^{\alpha_1H + \alpha_2L + \alpha_3R}$

%Where $\alpha_0$ = base diffusion constant where there are no roads, hedgerows
%or lights, $\alpha_1, \alpha_2, \alpha_3$ are the diffusion constants for
%hedgerows, lights and roads respectively. $H, L, R$ are occupancy values for
%hedgerows, lights and roads respectively (eg $H = 1$ in grid squares containing
%hedgerows, $H = 0$ in grid squares not containing hedgerows)?.


\section{Lightscape} 

\begin{figure} [h]
    \centering
    \includegraphics[width = 0.9\linewidth]{landscape_diffusion/log_point_irradiance.png}
    \label{fig:point_irradiance}
    \caption{Point irradiance (log scale) at each square on the grid.}
\end{figure}

\subsection{Binary point irradiance}
Light is either present or not: grid squares with light above a certain
threshold $\epsilon$ are 'light', all others are 'dark'. The model is fit using
ABC with 2 parameters: $\epsilon$, the threshold, and $\alpha$, the diffusion
coefficient multiplier for 'light' grid squares.

The diffusion coefficient for each square is given by:

$$D = D_0 e^{\alpha l} $$ 

where $D_0 = 75.5$, the base diffusion coefficient when no light is present, and
$l_{i,j} = 0$ for 'dark' grid squares and $l_{i,j} = 1$ for 'light' grid
squares. A uniform prior was used for $\alpha$, $-2 < \alpha < 3$. $\epsilon$
was chosen from a uniform distribution of the percentiles of non-zero lightscape
elements. 



\begin{figure} [h]
    \centering
    \includegraphics[width = 0.9\linewidth]{landscape_diffusion/lightscape_threshold_diffusion_threshold_percentile.png}
    \label{fig:epsilon_threshold}
    \caption{Prior and posterior for $\epsilon$}
\end{figure}


\begin{figure} [h]
    \centering
    \includegraphics[width = 0.9\linewidth]{landscape_diffusion/lightscape_threshold_diffusion_alpha.png}
    \label{fig:alpha_threshold}
    \caption{Prior and posterior for $\alpha$}
\end{figure}

\begin{figure} [h]
    \centering
    \includegraphics[width = 0.9\linewidth]{landscape_diffusion/lightscape_threshold_diffusion_alpa_threshold.png}
    \label{fig:}
    \caption{Posteriors for $\alpha$ and $\epsilon$}
\end{figure}

\begin{figure} [h]
    \centering
    \includegraphics[width = 0.9\linewidth]{landscape_diffusion/lightscape_intensity_bar.png}
    \label{fig:nnz_frequency}
    \caption{Frequency of different levels of light intensity (for non-zero grid squares).}


\end{figure}

Peak at 0: include picture


\pagebreak
\pagebreak
\pagebreak

\subsection{Log scaled point irradiance}

Non-zero elements log scaled and then scaled between 0 and 1.  
The diffusion coefficient for each square is given by:

$$D = D_0 e^{\alpha l} $$ 

\begin{figure} [h]
    \centering
    \includegraphics[width = 0.9\linewidth]{landscape_diffusion/lightscape_log_diffusion_alpha.png}
    \label{fig:log_diffusion}
    \caption{Prior and posterior for $\alpha$ with log scaled irradiance scores.}
\end{figure}   


%\subsection{Rejection}

%log or no log 

%$$ P_{accept} = \alpha l $$


\section{Roads}

Use distance to any roads and do the same threshold/alpha business as with
lights. Do we have any info on the traffic on these roads?

However, minor roads will have much less of an effect: unlikely to have someone
driving down a minor road at night? So try with minimum distance to A/B roads
and dual carriageways too. 