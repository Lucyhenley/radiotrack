
\section{A Diffusion-Convection model for bat movement}

%Partial differential equations are widely used to model a variety of phenomena, a

The diffusion-convection equation is a combination of diffusion and convection
\begin{equation}
  \D{\phi(r,t)}{t} = D\nabla^2\phi(x,y,t) - \chi \nabla\phi(x,y,t),
\end{equation}

$r \geq 0$

where $D$ is the diffusion coefficient, and $\chi$ is the convection coefficient. The initial condition

\begin{equation}
  \phi(r,0) = \delta(\theta),
\end{equation}

where $\theta$ is the location of the roost, specifies that all bats start the night in the roost.

The diffusion-convection equation can also be written in polar coordinates, noting that since diffusion is symmetric,  $\D{\phi}{\theta} = 0$.

\begin{equation}
  \D{\phi(r,t)}{t} = \frac{D}{r} \D{}{r}\left(r \D{\phi(r,t)}{r} \right) - \chi \D{\phi(r,t)}{r}.
\end{equation}

Approximate solutions can be found using perturbations, assuming that either convection or diffusion dominates. The solution $\phi(r,t)$ can be written as a superposition of two solutions with a perturbation,

\begin{equation}
  \phi(r,t) \approx \phi_0(r,t) + \epsilon \phi_1(r,t),
  \label{eqn:phi_perturbation}
\end{equation}

where $\epsilon$ is an arbitrary constant describing the strength of the perturbation. Since $\epsilon$ is arbitrary, the initial condition means that

\begin{equation}
  \phi(r,0) = \phi_0(r,0) = \phi_1(r,0),
\end{equation}

When diffusion dominates, the convection term acts as a perturbation, and the equation becomes

\begin{equation}
  \D{\phi(r,t)}{t} = \frac{D}{r} \D{}{r}\left(r \D{\phi(r,t)}{r} \right) - \epsilon\chi \D{\phi(r,t)}{r}.
  \label{eqn:convection_perturbation}
\end{equation}

Substituting \eqn{eqn:phi_perturbation} into \eqn{eqn:convection_perturbation} gives

\begin{equation}
\begin{split}
  \D{\phi_0(r,t)}{t} + \epsilon\D{\phi_1(r,t)}{t} &= \frac{D}{r} \left( \D{\phi_0(r,t)}{r} + r \DD{\phi_0(r,t)}{r} + \epsilon\D{\phi_1(r,t)}{r} + r\epsilon \DD{\phi_1(r,t)}{r}\right)  \\
  &- \epsilon\chi \left( \D{\phi_0(r,t)}{r} + \epsilon \D{\phi_1(r,t)}{r} \right)
  \end{split}
\end{equation}

An equation for $\phi_0(r,t)$ can be found by collecting coefficients of $\epsilon^0$,

\begin{equation}
  \D{\phi_0(r,t)}{t} = \frac{D}{r} \left(\D{\phi_0)(r,t)}{r} + r\DD{\phi_0(r,t)}{r}\right) - \epsilon\chi \D{\phi_0(r,t)}{r}.
  \label{eqn:diffusion_eps0}
\end{equation}

This is the diffusion equation in polar coordinates and can be solved by first converting to Cartesian coordinates,

\begin{equation}
  \D{\phi_0(x,y,t)}{t} = D \left( \DD{\phi_0(x,y,t)}{x}+\DD{\phi_0(x,y,t)}{y}\right),
  \label{eqn:cartesian_diffusion}
\end{equation}

and using a Fourier transform in the spatial dimensions. Writing the Fourier transform of $\phi_0(x,y,t)$ as $\mathscr{F}\left[ \phi_0(x,y,t) \right] = \tilde{\phi_0}(k_x,k_y,t)$, each term in equation

\begin{equation}
\begin{split}
  \mathscr{F}\left[ \DD{\phi_0(x,y,t)}{x} \right] &= \int \int \DD{\phi_0(x,y,t)}{x} \mathrm{e}^{-2\pi i(xk_x + yk_y)} dx dy \\
  &= -4\pi^2k_x^2 \tilde{\phi_0}(k_x,k_y,t), \\
  \mathscr{F}\left[ \DD{\phi_0(x,y,t)}{y} \right] &= \int \int \DD{\phi_0(x,y,t)}{y} \mathrm{e}^{-2\pi i(xk_x + yk_y)} dx dy \\
  &= -4\pi^2k_y^2 \tilde{\phi_0}(k_x,k_y,t) \\
  \mathscr{F}\left[\DD{\phi_0(x,y,t)}{t}\right] &= \DD{\tilde{\phi_0}(k_x,k_y,t)}{t} \\
  \mathscr{F}\left[\phi_0(x,y,0)\right] &= \mathscr{F}\left[\delta(0)\right] \\
  &= 1
\end{split}
\end{equation}


Substituting into \eqn{eqn:cartesian_diffusion}

\begin{equation}
  \DD{\tilde{\phi_0}(k_x,k_y,t)}{t} = -4\pi^2D (k_x^2 + k_y^2) \tilde{\phi_0}(k_x,k_y,t).
\end{equation}

Integrating with respect to time gives

\begin{equation}
\begin{split}
  \tilde{\phi_0}(k_x,k_y,t) &= \int -4\pi^2D (k_x^2 + k_y^2) \tilde{\phi_0}(k_x,k_y,t) dt \\
  &= \tilde{\phi_0}(k_x,k_y,0)\mathrm{e}^{-4\pi^2D(k_x^2 + k_y^2)t} \\
\end{split}
\end{equation}

IC $ \tilde{\phi_0}(k_x,k_y,0) = 1$ and so

\begin{equation}
  \tilde{\phi_0}(k_x,k_y,t) = \mathrm{e}^{-4\pi^2D(k_x^2 + k_y^2)t} \\
\end{equation}

Then inverse fourier transform to recover $\phi_0(x,y,t)$

\begin{equation}
  \phi_0(x,y,t) = \mathscr{F}^{-1}\left[\tilde{\phi_0}(k_x,k_y,t)\right] = \frac{1}{4\pi Dt}\mathrm{e}^{-\frac{x^2 + y^2}{4Dt}} \\
\end{equation}

Transforming back to polar coordinates gives

\begin{equation}
  \phi_0(x,y,t) = \frac{1}{4\pi Dt}\mathrm{e}^{-\frac{r^2}{4Dt}} \\
  \label{eqn:phi_0}
\end{equation}

Then gathering coefficients of $\epsilon^1$:

\begin{equation}
\D{\phi_1(x,y,t)}{t} = \frac{D}{r}\left(\D{\phi_1(x,y,t)} + r \DD{\phi_1(x,y,t)}{r}\right) - \chi\D{\phi_0}{r}
\label{eqn:eps1}
\end{equation}

and for $\epsilon^2$:

\begin{equation}
\begin{split}
0 &= -\chi\D{\phi_1(x,y,t)}{r} \\
\D{\phi_1(x,y,t)}{r} &= 0
\end{split}
\end{equation}

Substituting this into \eqn{eqn:eps1} gives

\begin{equation}
  \D{\phi_1(x,y,t)}{t} = - \chi\D{\phi_0(x,y,t)}{r} = - \frac{\chi r}{2Dt} \phi_0(x,y,t)
\end{equation}
