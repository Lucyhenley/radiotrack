
\chapter{Stochastic models for bat movement}

So far, only deterministic partial differential equation models have been
considered. These models provide a good description of average behaviour with a
population large enough to treat as a continuum. However, bat colonies tend to
contain a finite number of individuals. Greater Horseshoe bat colonies typically
have 50 - 200 individuals, although some colonies can number up to 600
\cite{greenaway1990field}. As a result, it is important to consider stochastic
effects in order to provide a realistic model for bat movement. Here we will
consider a stochastic description of the diffusion model describing dispersal
from the roost in two dimensions, and a model for the return to roost.

	\section{A stochastic diffusion model}

	For simplicity, we will first derive a stochastic diffusion model one
	dimension, before extending to two dimensions.

	Solution to diffusion equation in 1D with IC and BC describes the probability
	density after time $t$.

	\begin{equation}
\phi(x,t) = \frac{1}{\sqrt{4\pi Dt}}\exp \left(\frac{-x^2}{4Dt} \right) ,
\label{eqn:diffusion_solution1d}
	\end{equation}
%
equivalent to a normal distribution with mean $\mu = 0$ and variance $\sigma^2 =
2Dt$.

To convert to a stochastic diffusion model, we can convert the deterministic
diffusion equation to a stochastic differential equation with discrete time and
continuous space \cite{ghoniem1985grid,roberts2002langevin}. We assume that bats
are able to move freely through space, with the location recorded at regular
timesteps, with interval $\tau$. If all particles are initially located at the
origin (IC), $x(t=0)=0$, the probability density after one timestep $\tau$ must
be
%
\begin{equation}
\phi(x,t) = \frac{1}{\sqrt{4\pi D\tau}}\exp \left(\frac{-x^2}{4D\tau} \right) ,
\end{equation}
%
equivalent to a normal distribution with mean $\mu = 0$ and variance $\sigma^2 =
2D\tau$. Therefore, for the first timestep, each particle takes a step $dx$
drawn at random from this distribution, $dx \sim \mathcal{N}(0,2D\tau)$. If each
particle then takes a second step, drawn from the same distribution, the
probability density at $x$ and $t = 2\tau$ is given by the sum of the
probabilities that the particle reaches $x$ in 2 steps, the integral
%
\begin{equation}
\phi(x,t=2\tau) = \int_{-\infty}^{\infty} \frac{1}{4\pi D \tau} \exp \left(\frac{-\eta^2}{4D\tau} \right) \exp \left(\frac{-(x-\eta)^2}{4D\tau}\right).
\end{equation}
The result of this integral is given by
%
\begin{equation}
\phi(x,t=2\tau) = \frac{1}{\sqrt{8\pi D\tau}} \exp \left(\frac{-y^2}{8\pi D\tau} \right).
\end{equation}
%
By induction, after $k$ steps, at time $k\tau$, each chosen from the same distribution, $dx \sim \mathcal{N}(0,2D\tau)$, the probability density at $x$ is
%
\begin{equation}
\phi(x,t=k\tau) = \frac{1}{\sqrt{4k\pi D\tau}} \exp \left(\frac{-y^2}{4k\pi D\tau} \right),
\end{equation}
%
equivalent to the distribution given by the solution to the diffusion equation at time $t=k\tau$, \eqn{eqn:diffusion_solution1d}.
%
We can write this process as a stochastic differential equation,
%
\begin{equation}
dx_i = \rho_i,
\label{eqn:SDE}
\end{equation}
%
where $\{\rho_i\}$ is a set of random numbers chosen from the normal distribution with zero mean and standard deviation $\sqrt{2D\tau}$, such that $\rho_i \sim \mathcal{N}(0,2D\tau)$. The expression for $x_{i+1}$ can be written as
%
\begin{equation}
x_{i+1} = x_i + \rho_i,
\end{equation}
%
and an expression for $x_n$ can be written as a sum of random numbers,
\begin{equation}
x_{n} = x_0 + \sum_{i=1}^{n} \rho_i,
\end{equation}
%
where $x_0$ is the initial position at time $t=0$.

\subsection{Extending to two dimensions}

To extend the stochastic differential equation from one to two dimensions, we can consider the solution to the two dimensional diffusion equation,
%
\begin{equation}
\phi(x,y,t) = \frac{1}{4\pi Dt}\exp \left(\frac{-(x^2+y^2)}{4Dt} \right) .
\label{eqn:diffusion_solution2d}
\end{equation}
%
By separating this into $x$ and $y$ directions, we see that the solution to the two dimensional diffusion model is simply two one dimensional diffusion solutions multiplied together,
\begin{equation}
\phi(x,y,t) = \frac{1}{\sqrt{4\pi Dt}}\exp \left(\frac{-x^2}{4Dt} \right) \frac{1}{\sqrt{4\pi Dt}}\exp \left(\frac{-y^2}{4Dt} \right) .
\end{equation}
%
By treating the $x$ and $y$ directions separately, we use \eqn{eqn:SDE} to generate two stochastic differential equations for movement in each direction,
\begin{align}
%
dx_i &= \rho_i, \\
dy_i &= \lambda_i,
\end{align}
%
where $\{\rho_i\}$ and $\{\lambda_i\}$ are both sets of random numbers chosen from the normal distribution with zero mean and standard deviation $\sqrt{2D\tau}$, such that $\rho_i \sim \mathcal{N}(0,2D\tau)$ and $\lambda_i \sim \mathcal{N}(0,2D\tau)$.

	\section{Leapfrog model: return to roost}

\section{Results}

\begin{figure} [h]
    \centering
        \includegraphics[width=0.8\textwidth]{leapfrog_ABC.png}
        \caption{Leapfrog ABC prior/posterior        }
    \label{fig:leapfrog_abc}
\end{figure}

\begin{figure} [h]
    \centering
        \includegraphics[width=0.8\textwidth]{leapfrog_fit.png}
        \caption{MSD for leapfrog model compared to radio tracking data.
        }
    \label{fig:leapfrog_fit}
\end{figure}
